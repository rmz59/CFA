\documentclass{article}
\usepackage[top=1.2in, bottom=1in, left=1in, right = 1in]{geometry}
%\usepackage[font=footnotesize,width = 0.8\textwidth]{caption}
\usepackage{titling}
\usepackage{fancyhdr}
\usepackage{float}
\usepackage{graphicx}
\usepackage{subcaption}
\usepackage{indentfirst}
\usepackage{lastpage}
\usepackage{amsmath}
\usepackage{bm}
\usepackage{wrapfig}
\usepackage[inline]{enumitem}   % This allows me to enumerate things inline.
\usepackage{hyperref}

\hypersetup{
    pdftitle={CFA II Notes},
    pdfauthor={Runmin Zhang},
    %pdfsubject={Your subject here},
    %pdfkeywords={keyword1, keyword2},
    bookmarksnumbered=true,     
    bookmarksopen=true,         
    bookmarksopenlevel=1,       
    colorlinks=true,            
    pdfstartview=Fit,           
    pdfpagemode=UseOutlines,    % this is the option you were lookin for
    pdfpagelayout=TwoPageRight
}


\newcommand{\be}{\begin{enumerate}}
\newcommand{\ee}{\end{enumerate}}


\setlength\parindent{0pt}
\pagestyle{fancy}
\fancyhf{}
\fancyhead[L]{CFA II Notes}
\fancyhead[R]{Runmin Zhang}
\fancyfoot[R]{\thepage}
\begin{document}
\tableofcontents
\newpage
\section{Reading 9: Correlation and Regressions}
\subsection{Sample covar and sample correlation coefficients}
Sample covariance: $cov_{x,y}=\sum_i \frac{(X_i-\bar X)(Y_i-\bar Y)}{n-1}$
\\Sample correlation coeff: 
$r_{x,y}=\frac{cov_{x,y}}{s_x s_y}$, where $s_x$ is the sample dev of X.
\subsection{Limtations to correlations analysis}
Outliers: The results will be affected by extreme data points.(outliers)\\
Spurious correlation: There might be some non-zero corrlation coeff, but 
acutally they have no correlation at all.\\
Nonlinear relationships: Correlation only describe the linear relastions.
\subsection{Hypothesis: determine if the population 
correlation coefficient is zero}
Two-tailed hypothesis test:
$$
H_0: \rho=0, H_a: \rho \neq 0
$$
Assume that the two populations are {\bf normally} distrubited, then we 
can use t-test:
$$
t=\frac{r\sqrt{n-2}}{1-r^2}
$$:
Reject $H_0$ if  $t>+t_{critical}$ or $t<-t_{critical}$. Here,$r$ is the
sample correlation. Remember, you need to check t-table to find the t-value.
\subsection{Determine dependent/indepedent variables in a linear regression}
{\bf Simple linear regression}: Explain the variation in a dependent variable 
in terms of the variabltion in a single indepedent variable.
{\bf Independent variables} are called explanatory variable, the exogenous 
variable, or the predicting variable. 
{\bf Dependent variable} is also called the explained variable, the endogenous 
variable, or the predicted variable.
\subsection{Assumptions in linear regression and interpret regression coeff.}
\begin{enumerate}
    \item Assumptions of linear regression:
        \begin{enumerate}
            \item Linear relationship must exist.
            \item The indepedent variable is uncorrelated with residuals.
            \item Expected Residual term is value. $E(\epsilon)=0$
            \item variance of the residual term is const. $E(\epsilon_i^2)=
                \sigma_\epsilon^2$. Otherwise, it will be "heteroskedastic"
            \item The residual term is independently distributed. otherwise -- "auto correlation"" 
                $E(\epsilon_i\epsilon_j)=1$    
            \item The residual term is normally distributed.
        \end{enumerate}
    \item Simple Linear Regression Model
        \begin{enumerate}
            \item Model: $Y_i=b_0+b_1X_i+\epsilon_i$, where $i=1...n$, and $Y_i$ is 
     the actual observed data.
            \item The fitted line, the line of best fit
                : $\hat{Y}=\hat{b_0}+\hat{b1}X_i$. Where $\hat{b_0}$
                is the estimated parameter of the model.
            \item How to choose the best fitted line? {\bf Sum of squared errors}
                 is minimum.
                 $$
                    \hat{b_1} = \frac{cov_{x,y}}{sigma_x^2}
                 $$
                 where $X$ is the indepdent variable. $\hat{b_1}$ is 
                 "regression coeffcient".
                 $$
                    \hat{b_0} = \bar Y - \hat{b_1}\bar X
                 $$
                 where $\bar X, \bar Y$ are the mean.
        \end{enumerate}
    \item Interpreting a regression coefficient: Similar to basic ideas of 
        "slope". Keep in mind: any conclusion regarding this parameter needs 
        the statistical significance of the slope coefficient.
\end{enumerate}
\subsection{Standard error of estimate, the coeff. of determination and a 
confidence interval for a regression coefficient.}
\begin{enumerate}
    \item Standard error of estimate (SEE): Standard deviation between $Y_{estimate}$
        and $Y_{actual}$. - Smaller: better
    \item Coefficient of Determination ($R^2$)
        The percentage of the total variance in the dependent variable that is 
        predictable from the indepedent variable. - One indepdent variable: $R^2=r^2$,
        where $r^2$ is the square of correlation coefficient.
    \item Regression Coefficient confidence interval
        \begin{enumerate}
            \item Hypothesis: $H_0: b_1=0 \Leftrightarrow H_a: b_1\neq 0$ 
            \item Confidence interval:
                $\hat{b_1}-(t_c s_{\hat{b_1}})<b_1<\hat{b_1}+(t_c s_{\hat{b_1}})$ 
            $s_{\hat{b}_1}$ is the standard error of the regression coeffi.  
        \end{enumerate}
\end{enumerate}
\subsection{Hypothesis: Determine if $\hat{b}_1=b_1$}
\begin{enumerate}
    \item t-test statistic: $t_{b_1}=\frac{\hat{b}_1 - b_1}{s_{\hat{b}_1}}$
    \item Reject: if $t>+t_{critical}$ or $t<-t_{critical}$
\end{enumerate}
\subsection{Calculate the predicted value for the depedent variable}
If an estimated regression model is known, $\hat{Y}=\hat{b}_0+\hat{b}_1 X_p$
\subsection{Calculate and interpret a confidence interval for the predicted 
value of the depedent variable}
\begin{enumerate}
    \item Eq: $\hat{Y}\pm(t_c s_f)$, where $s_f$ is the {\color{red}std error of the forecast.}
    \item $s_f^2=SEE^2 \left[1+\frac{1}{n}+\frac{(X-\bar X)^2}{(n-1)s_x^2} \right]$
        \begin{enumerate}
            \item $SEE^2=$ variance of the residuals
            \item $s_x^2=$ variance of the indepdent variable
            \item $X =$ value of the independent variable where the forecast was
                made.
        \end{enumerate}
\end{enumerate}
\subsection{ANOVA in regression. Interpret results, and calculate F-statistic}
\begin{enumerate}
    \item Analysis of variance (ANOVA) is used to analyze the total variability 
        of the depedent variable. 
        \begin{enumerate}
            \item Total sum of squares(SST): $SST=\sum_{i=1}^n(Y_i-\bar Y)^2$
                \\SST is the total variation in the depedent variable.
                $Variance=SST/(n-1)$
            \item Regression sum of squares(RSS): $RSS=\sum_{i=1}^n(\hat{Y}_i-\bar Y)^2$
                \\RSS is the explained variation.
            \item Sum of squared errors(SSE): $SSE=\sum_{i=1}^n(Y_i-\hat{Y}_i)^2$
                \\SSE is the unexplained variation.
            \item {\color{red}$SST = RSS+SSE$ I cannot get this equation yet}
                You need to know how to use these squares.
            \item Degree of freedom: i) Regression(Explained): $k=1$, since we only
                estimate one parameters.
                ii) Error(Unexplained) $df=n-k-1=n-2$
                iii) Total variation $df=n-1$
        \end{enumerate}
    \item Calculating $R^2$ and {\bf SEE}
        \begin{enumerate}
            \item $R^2=explained variation/total varn=RSS/SST$
            \item $\bf{SEE}=\sqrt\frac{SSE}{n-2}$ {\bf SEE} is the std deviation of the
                regression error terms.
        \end{enumerate}
    \item The F-Statistic: used to explain whether {\it at least one} indepdent parameter
        can significanly explain the dependent parameter.  
        \begin{enumerate}
            \item F-statistic eq: $F=\frac{MSR}{MSE}=\frac{RSS/k}{SSE/n-k-1}$
                 where $MSR=$ mean regression sum of squares.
                       $MSE=$ mean squared errors. 
                       Note: \color{red}{One tailed test!}
        \end{enumerate}
    \item F-statistic with one independent variable.
        \begin{enumerate}
            \item Hypothesis: $H_0: b1=0 \Leftrightarrow H_a: b1\ne 0$
            \item degree of freedom: $df_{rss}=k=1,df_{sse}=n-k-1$
            \item Decision rule: reject $H_0$ if $F>F_c$
        \end{enumerate}
\end{enumerate}
\subsection{Limitations of regression analysis}
    \begin{enumerate}
        \item Parameter instability: the estimation eq may not be useful for other times.
        \item Limited usefulness: other participants may also use the same eq.
        \item Assumptions does not hold: i) Heteroskedastic, i.e., non-const
            variance of the error terms. ii) autocorrelation, i.e., error terms
            are not independent.
    \end{enumerate}

\section{Reading 10: Multiple Regression and Issues in Regression Analysis}
Some basic ides
    \begin{enumerate}
        \item Model: 
            $Y_i=b_0+b_1 X_{1i}+b_2 X_{2i}+...+b_k X_{ki}+\epsilon_i$
        \item Multiple regression methodology estimates the intercept and 
            slope coefficients so that $\sum_i^n \epsilon_i^2$ is minimized.
    \end{enumerate}
\subsection{Interpret estimated regression coefficients and their p-values.}
They are just simple linear functions with multiple parameters. Ignore.
\subsection{Formulate a null/alternative hypothesis, do correspoding calculations}
\begin{enumerate}
    \item Hypothesis Testing of Regression coefficient. (Multi-parameters). \\
        Use t-statistics to determine if one parameter significantly contribute
        to the model.
        $$
            t=\frac{\hat{b}_j-b_j}{s_{\hat{b}_j}}, df=n-k-1
        $$
        where $k$ is the number of regression coefficients, and $1$ corresponds to
        the intercept term, and $s_{\hat{b}_j}$ is the coefficient standard error
        of $b_j$
    \item Determining statistical significance.
        \\ ``testing statistical significance" $\Rightarrow H_0: b_j=0, H_a: b_j\ne 0$
    \item Interpreting p-values.
        \begin{enumerate}
            \item Def: p-value is {\bf the smallest level of significance for 
                which the null hypothesis can be rejected.}
                \ If the p-value is less than significance level, the null 
        \end{enumerate}
    \item Other Tests of the Regression Coefficients: $H_0: a=$some value
\end{enumerate}
\subsection{Calculate and Interpret a confidence interval for the population 
value of a regression coefficient or a predicted value for the depedent variable
if an estimated regression model.}
\begin{enumerate}
    \item Confidence intervals for a regress. coeff.: $\hat{b}_j\pm(t_c\times s_{\hat{b}_j})$
    \item predicting the depedent variable: 
        $\hat{Y}_i=\hat{b}_0+\hat{b}_1 \hat{X}_{1i}+...+\hat{b}_k \hat{X}_{ki}$
        \\Even if you may conclude that some $b_i$ are not statistally significantly, you cannot
        treat them as $0$ and keep other parameters unchanged. You should use the
        original model, or you can throw $\hat{b}_k$ away and make a new regression model.
\end{enumerate}
\subsection{Assumptions of a multiple regresssion model}
\begin{enumerate}
    \item Linear relationships exist.
    \item The independent variables are not random, and there is no exact linear relation
        between indepdent variables.
    \item $E[\epsilon|X_1,...,X_k]=0$
    \item Variance of $\epsilon=0$, i.e. $E[\epsilon_i]=0$
    \item $E(\epsilon_i\epsilon_j)=0$
    \item $\epsilon$ is normally distributed.
\end{enumerate}
\subsection{Calculate and interpret F-statistic}
F-test: whether at least {\bf one} of the indepedent variables explains a significant
portion of the variation of the depedent variable. F test is a one-tail test.
\begin{enumerate}
    \item $H_0: b1=b2=b3=0 vs H_a:$ at least one $b_j\ne 0$
    \item $F=\frac{MSR}{MSE}=\frac{RSS/k}{SSE/n-k-1}$
    \item Degree of freedom: $df_{numerator}=k, df_{denominator}=n-k-1$
    \item Rules: reject $H_0$ if $F(test-statistic)>F_c(critical value)$
\end{enumerate}
\subsection{Distinguish between $R^2$ and adjusted $R^2$}
\begin{enumerate}
    \item coefficient of determination $R^2$: used to test if a group of indepedent
        variable can explain the depedent variable: 
        \\$R^2=\frac{total variation - unexplained variation}{total variation}
        =\frac{SST-SSE}{SST}=\frac{RSS}{SST}$
        \\{\bf Multiple R} = $\sqrt{R^2}$
    \item Adjusted $R^2$
        \begin{enumerate}
            \item Note: $R^2$: {\color{red}{Overestimating}}: will increase as variables are added to the model.
        Even the marginal contribution of new variables are not statistically significant.
            \item Introduce $R_a^2$: $R_a^2=1-\left[\left(\frac{n-1}{n-k-1}\right)\right](1-R^2)$
        \end{enumerate}
\end{enumerate}
\subsection{Evaluate the quality of a regression model by analyzing the ouput of the equation/ANOVA table}
\begin{enumerate}
    \item ANOVA Tables, some important quantities
        \begin{enumerate}
            \item $R^2=\frac{RSS}{SST}$
            \item $F=\frac{MSR}{MSE}$ with $k$ and $n-k-1$ df
            \item Standard error of estimate:$SEE=\sqrt{MSE}$
        \end{enumerate}
\end{enumerate}
\subsection{Formulate a multiple regression with dummpy variables to represent qualitative factors}
\begin{enumerate}
    \item Def: Some value is quite qualitative. Using dummy values like 0 or 1 
        to describe their impacts.
    \item Note: Pay attention to \# of dummy variables. If $n$ classes, we must use
        $n-1$ dummy variables.
    \item Interpreting the coefficients in a dummy variable regression. We can use F-statistics to
        test a group of parameters, or use t-test to test the individual slope coefficients.
    \item Example of Regression application with dummy variables. See Notes directly.
\end{enumerate}
\subsection{Why multiple regression isn't as easy as it looks?}
Pay attention to the assumptions that have been used. Violations like::
\begin{enumerate}
    \item Heteroskedasticity
    \item Serial correlation (auto-correlation)
    \item Multicollinarity
\end{enumerate}
Any violations on the assumptions will impact the estimation of SEE, and finaly 
change the t-statistic and F-statistic, and change the conclustion of the hypotesis
test.
\subsection{Types of Heteroskedasticity, how heteroskedasiticity and serial correlation affect inference}
\begin{enumerate}
    \item What is Heteroskedasticity?
        \\ {\bf Corresponding assumptions: Variance of the residuals is constant across observations. -- Homoskedasticity} 
        Heteroskedasity means the variance of the residuals is not equal.
        \begin{enumerate}
            \item Unconditional heter: Not related to the level of the indepedent
                variables. Will not systematiclly increase with changes in the value
                of the indepedent variables. {\color{red}Usually will not casue major problems.}
            \item Conditional heter: Related to the level of the independent variables. Eg:
                Conditional heter exists if the variance of the residuals increase with 
                the value of the independent variables increases. {\color{red}Will cause big problems.}
        \end{enumerate}
    \item Effect of Heteroskedasiticity on Regression Analysis
        \begin{enumerate}
            \item Unreliable standard errors.
            \item The coefficient estimates aren't affected.
            \item Will change the t-statistic, and will change the conclusion.
            \item Unreliable F-test
        \end{enumerate}
    \item Detect Heteroskedasicity
        \begin{enumerate}
            \item Scatter plot
            \item Breusch-pagan test: $BP test = n\times R^2_{resid}$ with $df=k$. 
                where $n=$the number of observations, $R^2_{resid}$=$R^2$ from a second
                regression of the squared residuals from the first regression. $k=$the number
                of independent variables. If $R^2$ or BP-test are too large, something is wrong.
        \end{enumerate}
    \item Correcting Heteroskedasticity
        \be
        \item Calculate robust sndard errors(White corrected std errors.). Use them for t-test.
            \item Generalized least squares. 
        \ee
    \item What is serial correlations?
        \be
            \item Def: auto-correlation, in which the residual terms are correlated.
                Common problem with time series data.
                \be
                    \item Positive serial correlation: a postive error in one time period will
                        increase the posibility to observe a positive one next time.
                    \item Negative serial correlation: Just opposite.
                \ee
            \item Effect: positive serial correlation will get small coefficient std errors. 
                Thus, too large t-statistics. therefore, too many Type I errors: reject the null
                hypothesis $H_0$ while it's actually true. 
            \item Detection:
                \be
                    \item Residual plots
                    \item Durbin-Watson statistics: 
                        $$
                        \color{red}DW=\frac{\sum_{t=2}^T (\hat{\varepsilon}_t-\hat{\epsilon}_{t-1})^2}{\sum_{t=1}^T \hat{\epsilon}^2}
                        $$
                        For large samples, $DW\approx 2(1-r)$, where $r$ is the 
                        correlation coefficient between residuals from one period and thoese from the previous period.
                        \\Results: 
                        \be
                            \item $DW=2\Rightarrow$ Homoskedasitic and not serially correlated.
                            \item $DW<2\Rightarrow$ Postively serially correlated.
                            \item $DW>2\Rightarrow$ Negatively serially correlated.

                        \ee
                        Formulated hypothesis with DW-table, upper and lower critical values
                        \be
                            \item Hypothesis: $H_0:$ the regression has {\bf no}
                                positive serial correlation.
                            \item $DW<d_l$: positive serially correlated. Reject null.
                            \item $d_l<DW<d_u$: inconclusive results.
                            \item $DW>d_u$: {\bf There is no evidence that are positive correlated.} 
                        \ee
                \ee
            \item Correcting serial correlation:
            \be
                \item Adjust the coefficient std errors. {\bf recommended.} Using Hansen method.
                    \be
                        \item Serial correlation only: Hansen method. 
                        \item Heteroskedasticity only: White-corrected stand errors.
                        \item Both: Hans methods.
                    \ee
                \item Imporoe the specification of the model.
            \ee
    \ee
\end{enumerate}
\subsection{Multicolinearity and its cause and effects in regression analysis}
Multicollinearity: Indepedent variables or linear combinations of independent variables are highly
correlated.
\begin{enumerate}
    \item Effect of Multicollinearity on Regression Analysis: Will increase the std errors of 
        the slope coefficients. {\color{red} Type II Error:
        A variable is significant, while we conclude it's not.} 
    \item Detecting: Common situation: $t-statistic$ is not significant while $F-test$ is significant.
        This tells us the indepedent variables are highly correlated.
        \\ A simple rule works if there are 2 indepedent variables: when the absolute value of the sample correlation betewen any two 
        indepedent variables in the regression is greater than 0.7.
    \item Correcting: omit one or more of the correlated indepedent variables. 
        THe problame is that it's hard to find the variables that result in the multicolinearity.
\end{enumerate}
\subsection{Model misspecification}
\begin{enumerate}
    \item Defination of {\bf Regression model of specification}: decide which independent variables
        to be included in the model.
    \item Types of misspecification
        \be
            \item The functional form can be misspecified:
                important variables are ommited;variables should be transformed; data is improperly pooled.
            \item Explanatory variables are correlated with error term in time series model:
                A lagged dependent variable is used as an independent variable; a function of the dependent
                variable is used as an independent variable(forecasting the past);
                independent variables are measured with error.
            \item Other time-series misspecification.
        \ee
\end{enumerate}
\subsection{Models with qualitive dependent variables}
Include qualitative dependent variables, like default, bankcrupcy. Cannot use an 
ordinary regression model. Should use other models like {\bf probit and logit models}
or {\bf discriminant models}.
\be
\item Probit: normal distribution, give probability.
\item Logistic: logistic distribution. 
\item Discriminant: result in an overall score or ranking.
\ee
% End Reading 10


%% Reading 11
\section{Reading 11: Time-Series Analysis}
\subsection{Calculate/evaluate the predicted trend value for a time series given 
the estimated trend coefficients}
\begin{enumerate}
    \item Linear Trend Model and Log-linear Trend
        \be
            \item Definition: $y_t=b_0+b_1(t)+\epsilon_t$ Note: $t$ is just time.
            \item Coefficients is determined by OLS. Ordinary least squared regression.
                \\$\hat{y}=\hat{b}_0+\hat{b}_1$
            \item Log-linear Trend Models
            \item Model: $y_t=\exp{b_0+b_1(t)}\Rightarrow \ln{y_t}=b_0+b_1(t)$
        \ee
\end{enumerate}
\subsection{Factors that determine whether a linear or a log-linear model trend 
should be used}
\be
    \item Factors that determine which model is best: plot data.
    \item Limitaions of trend models: 
        \be
            \item residuals are uncorrelated with each other.
        Otherwise, it will cause auto correlation and we should not use the trend model.
            \item For log-linear model, it is not suitable for cases with serial correlations (autocorrelation).
            \item Detect auto correlation: Durbin Watson statistic. $DW=2.0\Rightarrow$ No auto correlation.
        \ee
\ee
\subsection{Autoregressive model, requirements for covariance statinoary}
\be
    \item Autoregressive model:
        \be
            \item Model: $x_t=b_0+b_1 x_{t-1}+\varepsilon_t$
            \item Statistical inferences bse on ordinary least squares estimates doesn't
                apply unless the time series is {\bf covariance stationary}.
            \item Conditions for covariance stationary
                \be
                    \item Constant and finite expected value.
                    \item Constant and finite variance.
                    \item Constant and finite covariance between values at any given lag.
                \ee
        \ee
\ee
\subsection{An autogressive model of order $p$}
\be
    \item Model(order $p$): $x_t=b_0+b_1x_{t-1}+b_2x_{t_2}+...+b_px_{t-p}+\varepsilon_t$ 
    \item Forecasting with an autoregressive model: 
        \be
            \item One-period-ahead forecast for $AR(1)$: $\hat{x}_{t+1}=\hat{b}_0+\hat{b}_1 x_t$
            \item Two-period-ahead forecast for $AR(1)$: $\hat{x}_{t+2}=\hat{b}_0+\hat{b}_1 \hat{x}_{t+1}$
        \ee
\ee
\subsection{How the residuals can be used to test the autogressive model}
\be
    \item The residual should have no {\it serial correlation} if an AR model is correct.
    \item Steps
        \be
            \item Estimate: Start with AR(1)
            \item Calculate: the autocorrelations of he model residuals
            \item Test: whether the autoccorelations are signficantly different from 0.
                \\The standard error is $\frac{1}{\sqrt{T}}$ for $T$ observations. The t-test for each
                observation is $t=\frac{\rho_{\epsilon_t,epsilon_{t-k}}}{1/sqrt{T}}$, with $T-2$ df. 
        \ee
\ee
\subsection{Mean reversion and a mean-reverting level}
\be
    \item Mean reversion: The time series tends to move toward its mean.
    \item Mean-reverting level: $\hat{x}_{t+1}=x_{t}$, where $\hat{x}_t$ is the predicted value. 
    \item All covariance stationary time series has finite mean-reverting level. 
\ee
\subsection{Contrast in-sample and out-of-sample forecasts and the forecasting accuracy of 
different time-series models based on the root mean squared error criterion.}
\be
    \item in-sample, out-of-sample: determined by if the predicted data is in the range of 
        the observations.
    \item RMSE, root mean squared error: used to compare the accurancy. If the accurancy of 
        out-of-sample is better, you should use it for future applications
\ee
\subsection{Explain the instability of coefficients of time-series models}
\be
    \item Instability or nonstationarity. Due to the dynamic econimic conditions, 
        model coefficients will change a lot from period to period. 
    \item Shorter time series are more stable, but longer time series are more reliable.
\ee
\subsection{Random walk processes and their comparisons between covariance stationary processes}
\be
    \item Random walk: $x_t=x_{t-1}+\varepsilon_t$
    \be
        \item $E(\varepsilon_t)=0$: The expected value of each error is zero.
        \item $E(\varepsilon_t^2)=0$: The variance of the error terms is constant.
        \item $E(\varepsilon_i,\varepsilon_j)=0$: There is no serial correlation in the error terms.
    \ee
    \item Random walk with a Drift: $x_t=b_0+b_1 x_{t-1}+\varepsilon_t$, where $b_1$=0
    \item A random walk or a random walk with a drift have no finte mean-reverting level.
        Since $b_1=1,\frac{b_0}{1-b_1}=\frac{b_0}{0}$. Therefore, they are not covariance stationary.
    \item $b_1=1$, they exhibit a unit root. Thus, {\bf the least square regression that been used in 
        AR(1) will not work unless we transfrom the data}.
\ee
\subsection{Things about unit roots: when they will occur, how to test them, how to transform data to apply AR}
\be
    \item Unit root testing for nonstationarity:
        \be
            \item run an AR model and check autocorrelations
            \item perform Dickey Fuller test.
                \be
                    \item Transform: $x_t=b_0+b_1x_1+\varepsilon\Rightarrow x_t-x_{t-1}=b_0+(b_1-1)x_{t-1}+\varepsilon$
                    \item Direct test if $b_1-1=0$ using a modified t-test.    
                \ee
        \ee
    \item First differencing
        \be
            \item For a random walk, transform the data $y_t=x_t-x_{t-1}\Rightarrow y_t=\varepsilon_t$
                then start to use an AR model $y=b_0+b_1 y_{t-1}+\varepsilon$, 
                where $b_0=b_1=0$
            \item $y$ is covariance stationary.
        \ee
\ee
\subsection{How to test and correct for seasonality in a time-series model, and 
calculate and interpret a forecasted value using an AR model with a sesonal lag.}
\be
    \item Detect: special autocorrelation exists for some seasonal lags.
    \item Correction: Add an additional seasonal lag term.
\ee
\subsection{Explain autogressive conditional heteroskedasticity (ARCH) and describe
how ARCH models can be applied to predict the variance of a time series}
\be
    \item ARCH: the variance of the residuals in one period is dependent on the variance of
        the residuals in a previous period.
    \item Using ARCH models:
        \\Example $ARCH(1)$: $\hat{\varepsilon}_t^2=a_0+a_1\hat{\varepsilon}_{t-1}+\mu_t$
        if $a_1$ is significantly different from zero. $\hat{\varepsilon}_t^2$ is the squared residuals.
        \\Note: Things like generalized least squares should me used to 
                correct heteroskedasticity. otherwise, the std errors of the 
                coefficients will be wrong, leading to invalid conclusions.
    \item Predicting the variance of a time series: using ARCH model to predict the variance of 

        future periods: $\hat{\sigma}^2_{t+1}=\hat{a}_0+\hat{a}_1\hat{\varepsilon}_t^2$
\ee
\subsection{Explain How time-series variables should be analyzed for nonstationarity and/or cointegration before
use ain a linear regression}
\be
    \item Cointegration: 
        \be
            \item Two time series are economically linked or follow the 
        same trend and that relationship is not expected to change. -- Error terms 
        from regressing one on the other is covariance stationary and the t-test are reliable.
            \item How to test conintegration: regress $y_t$ on $x_t$ 
                 $y_t=b_0+b_1x_t+\varepsilon$, $y_t, x_t$ are two different time series.
                 Then do a unit root test using the Dickey Fuller test with critical t-values
                 calculated by Engle and Granger. 
                 \\If "A unit root" is rejected $\Rightarrow$ covariance stationary, cointegrated.
        \ee
\ee


\section{Reading 12: Probabilistic Approaches: Scenario Analysis, Decision Trees, and Simulations}

\end{document}
