\documentclass{article}
\usepackage[top=1.2in, bottom=1in, left=1in, right = 1in]{geometry}
%\usepackage[font=footnotesize,width = 0.8\textwidth]{caption}
\usepackage{titling}
\usepackage{fancyhdr}
\usepackage{float}
\usepackage{graphicx}
\usepackage{subcaption}
\usepackage{indentfirst}
\usepackage{lastpage}
\usepackage{amsmath}
\usepackage{unicode-math}
\usepackage{bm}
\usepackage{wrapfig}
\usepackage[inline]{enumitem}   % This allows me to enumerate things inline.
\usepackage{hyperref}

\hypersetup{
    pdftitle={CFA II Notes},
    pdfauthor={Runmin Zhang},
    %pdfsubject={Your subject here},
    %pdfkeywords={keyword1, keyword2},
    bookmarksnumbered=true,     
    bookmarksopen=true,         
    bookmarksopenlevel=1,       
    colorlinks=true,            
    pdfstartview=Fit,           
    pdfpagemode=UseOutlines,    % this is the option you were lookin for
    pdfpagelayout=TwoPageRight
}


\newcommand{\be}{\begin{enumerate}}
\newcommand{\ee}{\end{enumerate}}
\newcommand{\ra}{$\rightarrow$}
\newcommand{\Ra}{$\Rightarrow$}
\newcommand{\eq}[1]{\begin{align*}\begin{split}#1\end{split}\end{align*}}
\setlength\parindent{0pt}
\pagestyle{fancy}
\fancyhf{}
\fancyhead[L]{CFA II Notes}
\fancyhead[R]{Runmin Zhang}
\fancyfoot[R]{\thepage}
\begin{document}
\tableofcontents
\newpage
\section{Reading 9: Correlation and Regressions}
\subsection{Sample covar and sample correlation coefficients}
Sample covariance: $cov_{x,y}=\sum_i \frac{(X_i-\bar X)(Y_i-\bar Y)}{n-1}$
\\Sample correlation coeff: 
$r_{x,y}=\frac{cov_{x,y}}{s_x s_y}$, where $s_x$ is the sample dev of X.
\subsection{Limtations to correlations analysis}
Outliers: The results will be affected by extreme data points.(outliers)\\
Spurious correlation: There might be some non-zero corrlation coeff, but 
acutally they have no correlation at all.\\
Nonlinear relationships: Correlation only describe the linear relastions.
\subsection{Hypothesis: determine if the population 
correlation coefficient is zero}
Two-tailed hypothesis test:
$$
H_0: \rho=0, H_a: \rho \neq 0
$$
Assume that the two populations are {\bf normally} distrubited, then we 
can use t-test:
$$
t=\frac{r\sqrt{n-2}}{1-r^2}
$$:
Reject $H_0$ if  $t>+t_{critical}$ or $t<-t_{critical}$. Here,$r$ is the
sample correlation. Remember, you need to check t-table to find the t-value.
\subsection{Determine dependent/indepedent variables in a linear regression}
{\bf Simple linear regression}: Explain the variation in a dependent variable 
in terms of the variabltion in a single indepedent variable.
{\bf Independent variables} are called explanatory variable, the exogenous 
variable, or the predicting variable. 
{\bf Dependent variable} is also called the explained variable, the endogenous 
variable, or the predicted variable.
\subsection{Assumptions in linear regression and interpret regression coeff.}
\begin{enumerate}
    \item Assumptions of linear regression:
        \begin{enumerate}
            \item Linear relationship must exist.
            \item The indepedent variable is uncorrelated with residuals.
            \item Expected Residual term is value. $E(\epsilon)=0$
            \item variance of the residual term is const. $E(\epsilon_i^2)=
                \sigma_\epsilon^2$. Otherwise, it will be "heteroskedastic"
            \item The residual term is independently distributed. otherwise -- "auto correlation"" 
                $E(\epsilon_i\epsilon_j)=1$    
            \item The residual term is normally distributed.
        \end{enumerate}
    \item Simple Linear Regression Model
        \begin{enumerate}
            \item Model: $Y_i=b_0+b_1X_i+\epsilon_i$, where $i=1...n$, and $Y_i$ is 
     the actual observed data.
            \item The fitted line, the line of best fit
                : $\hat{Y}=\hat{b_0}+\hat{b1}X_i$. Where $\hat{b_0}$
                is the estimated parameter of the model.
            \item How to choose the best fitted line? {\bf Sum of squared errors}
                 is minimum.
                 $$
                    \hat{b_1} = \frac{cov_{x,y}}{sigma_x^2}
                 $$
                 where $X$ is the indepdent variable. $\hat{b_1}$ is 
                 "regression coeffcient".
                 $$
                    \hat{b_0} = \bar Y - \hat{b_1}\bar X
                 $$
                 where $\bar X, \bar Y$ are the mean.
        \end{enumerate}
    \item Interpreting a regression coefficient: Similar to basic ideas of 
        "slope". Keep in mind: any conclusion regarding this parameter needs 
        the statistical significance of the slope coefficient.
\end{enumerate}
\subsection{Standard error of estimate, the coeff. of determination and a 
confidence interval for a regression coefficient.}
\begin{enumerate}
    \item Standard error of estimate (SEE): Standard deviation between $Y_{estimate}$
        and $Y_{actual}$. - Smaller: better
    \item Coefficient of Determination ($R^2$)
        The percentage of the total variance in the dependent variable that is 
        predictable from the indepedent variable. - One indepdent variable: $R^2=r^2$,
        where $r^2$ is the square of correlation coefficient.
    \item Regression Coefficient confidence interval
        \begin{enumerate}
            \item Hypothesis: $H_0: b_1=0 \Leftrightarrow H_a: b_1\neq 0$ 
            \item Confidence interval:
                $\hat{b_1}-(t_c s_{\hat{b_1}})<b_1<\hat{b_1}+(t_c s_{\hat{b_1}})$ 
            $s_{\hat{b}_1}$ is the standard error of the regression coeffi.  
        \end{enumerate}
\end{enumerate}
\subsection{Hypothesis: Determine if $\hat{b}_1=b_1$}
\begin{enumerate}
    \item t-test statistic: $t_{b_1}=\frac{\hat{b}_1 - b_1}{s_{\hat{b}_1}}$
    \item Reject: if $t>+t_{critical}$ or $t<-t_{critical}$
\end{enumerate}
\subsection{Calculate the predicted value for the depedent variable}
If an estimated regression model is known, $\hat{Y}=\hat{b}_0+\hat{b}_1 X_p$
\subsection{Calculate and interpret a confidence interval for the predicted 
value of the depedent variable}
\begin{enumerate}
    \item Eq: $\hat{Y}\pm(t_c s_f)$, where $s_f$ is the {\color{red}std error of the forecast.}
    \item $s_f^2=SEE^2 \left[1+\frac{1}{n}+\frac{(X-\bar X)^2}{(n-1)s_x^2} \right]$
        \begin{enumerate}
            \item $SEE^2=$ variance of the residuals
            \item $s_x^2=$ variance of the indepdent variable
            \item $X =$ value of the independent variable where the forecast was
                made.
        \end{enumerate}
\end{enumerate}
\subsection{ANOVA in regression. Interpret results, and calculate F-statistic}
\begin{enumerate}
    \item Analysis of variance (ANOVA) is used to analyze the total variability 
        of the depedent variable. 
        \begin{enumerate}
            \item Total sum of squares(SST): $SST=\sum_{i=1}^n(Y_i-\bar Y)^2$
                \\SST is the total variation in the depedent variable.
                $Variance=SST/(n-1)$
            \item Regression sum of squares(RSS): $RSS=\sum_{i=1}^n(\hat{Y}_i-\bar Y)^2$
                \\RSS is the explained variation.
            \item Sum of squared errors(SSE): $SSE=\sum_{i=1}^n(Y_i-\hat{Y}_i)^2$
                \\SSE is the unexplained variation.
            \item {\color{red}$SST = RSS+SSE$ I cannot get this equation yet}
                You need to know how to use these squares.
            \item Degree of freedom: i) Regression(Explained): $k=1$, since we only
                estimate one parameters.
                ii) Error(Unexplained) $df=n-k-1=n-2$
                iii) Total variation $df=n-1$
        \end{enumerate}
    \item Calculating $R^2$ and {\bf SEE}
        \begin{enumerate}
            \item $R^2=explained variation/total varn=RSS/SST$
            \item $\bf{SEE}=\sqrt\frac{SSE}{n-2}$ {\bf SEE} is the std deviation of the
                regression error terms.
        \end{enumerate}
    \item The F-Statistic: used to explain whether {\it at least one} indepdent parameter
        can significanly explain the dependent parameter.  
        \begin{enumerate}
            \item F-statistic eq: $F=\frac{MSR}{MSE}=\frac{RSS/k}{SSE/n-k-1}$
                 where $MSR=$ mean regression sum of squares.
                       $MSE=$ mean squared errors. 
                       Note: \color{red}{One tailed test!}
        \end{enumerate}
    \item F-statistic with one independent variable.
        \begin{enumerate}
            \item Hypothesis: $H_0: b1=0 \Leftrightarrow H_a: b1\ne 0$
            \item degree of freedom: $df_{rss}=k=1,df_{sse}=n-k-1$
            \item Decision rule: reject $H_0$ if $F>F_c$
        \end{enumerate}
\end{enumerate}
\subsection{Limitations of regression analysis}
    \begin{enumerate}
        \item Parameter instability: the estimation eq may not be useful for other times.
        \item Limited usefulness: other participants may also use the same eq.
        \item Assumptions does not hold: i) Heteroskedastic, i.e., non-const
            variance of the error terms. ii) autocorrelation, i.e., error terms
            are not independent.
    \end{enumerate}

\section{Reading 10: Multiple Regression and Issues in Regression Analysis}
Some basic ides
    \begin{enumerate}
        \item Model: 
            $Y_i=b_0+b_1 X_{1i}+b_2 X_{2i}+...+b_k X_{ki}+\epsilon_i$
        \item Multiple regression methodology estimates the intercept and 
            slope coefficients so that $\sum_i^n \epsilon_i^2$ is minimized.
    \end{enumerate}
\subsection{Interpret estimated regression coefficients and their p-values.}
They are just simple linear functions with multiple parameters. Ignore.
\subsection{Formulate a null/alternative hypothesis, do correspoding calculations}
\begin{enumerate}
    \item Hypothesis Testing of Regression coefficient. (Multi-parameters). \\
        Use t-statistics to determine if one parameter significantly contribute
        to the model.
        $$
            t=\frac{\hat{b}_j-b_j}{s_{\hat{b}_j}}, df=n-k-1
        $$
        where $k$ is the number of regression coefficients, and $1$ corresponds to
        the intercept term, and $s_{\hat{b}_j}$ is the coefficient standard error
        of $b_j$
    \item Determining statistical significance.
        \\ ``testing statistical significance" $\Rightarrow H_0: b_j=0, H_a: b_j\ne 0$
    \item Interpreting p-values.
        \begin{enumerate}
            \item Def: p-value is {\bf the smallest level of significance for 
                which the null hypothesis can be rejected.}
                \ If the p-value is less than significance level, the null 
        \end{enumerate}
    \item Other Tests of the Regression Coefficients: $H_0: a=$some value
\end{enumerate}
\subsection{Calculate and Interpret a confidence interval for the population 
value of a regression coefficient or a predicted value for the depedent variable
if an estimated regression model.}
\begin{enumerate}
    \item Confidence intervals for a regress. coeff.: $\hat{b}_j\pm(t_c\times s_{\hat{b}_j})$
    \item predicting the depedent variable: 
        $\hat{Y}_i=\hat{b}_0+\hat{b}_1 \hat{X}_{1i}+...+\hat{b}_k \hat{X}_{ki}$
        \\Even if you may conclude that some $b_i$ are not statistally significantly, you cannot
        treat them as $0$ and keep other parameters unchanged. You should use the
        original model, or you can throw $\hat{b}_k$ away and make a new regression model.
\end{enumerate}
\subsection{Assumptions of a multiple regresssion model}
\begin{enumerate}
    \item Linear relationships exist.
    \item The independent variables are not random, and there is no exact linear relation
        between indepdent variables.
    \item $E[\epsilon|X_1,...,X_k]=0$
    \item Variance of $\epsilon=0$, i.e. $E[\epsilon_i]=0$
    \item $E(\epsilon_i\epsilon_j)=0$
    \item $\epsilon$ is normally distributed.
\end{enumerate}
\subsection{Calculate and interpret F-statistic}
F-test: whether at least {\bf one} of the indepedent variables explains a significant
portion of the variation of the depedent variable. F test is a one-tail test.
\begin{enumerate}
    \item $H_0: b1=b2=b3=0 vs H_a:$ at least one $b_j\ne 0$
    \item $F=\frac{MSR}{MSE}=\frac{RSS/k}{SSE/n-k-1}$
    \item Degree of freedom: $df_{numerator}=k, df_{denominator}=n-k-1$
    \item Rules: reject $H_0$ if $F(test-statistic)>F_c(critical value)$
\end{enumerate}
\subsection{Distinguish between $R^2$ and adjusted $R^2$}
\begin{enumerate}
    \item coefficient of determination $R^2$: used to test if a group of indepedent
        variable can explain the depedent variable: 
        \\$R^2=\frac{total variation - unexplained variation}{total variation}
        =\frac{SST-SSE}{SST}=\frac{RSS}{SST}$
        \\{\bf Multiple R} = $\sqrt{R^2}$
    \item Adjusted $R^2$
        \begin{enumerate}
            \item Note: $R^2$: {\color{red}{Overestimating}}: will increase as variables are added to the model.
        Even the marginal contribution of new variables are not statistically significant.
            \item Introduce $R_a^2$: $R_a^2=1-\left[\left(\frac{n-1}{n-k-1}\right)\right](1-R^2)$
        \end{enumerate}
\end{enumerate}
\subsection{Evaluate the quality of a regression model by analyzing the ouput of the equation/ANOVA table}
\begin{enumerate}
    \item ANOVA Tables, some important quantities
        \begin{enumerate}
            \item $R^2=\frac{RSS}{SST}$
            \item $F=\frac{MSR}{MSE}$ with $k$ and $n-k-1$ df
            \item Standard error of estimate:$SEE=\sqrt{MSE}$
        \end{enumerate}
\end{enumerate}
\subsection{Formulate a multiple regression with dummpy variables to represent qualitative factors}
\begin{enumerate}
    \item Def: Some value is quite qualitative. Using dummy values like 0 or 1 
        to describe their impacts.
    \item Note: Pay attention to \# of dummy variables. If $n$ classes, we must use
        $n-1$ dummy variables.
    \item Interpreting the coefficients in a dummy variable regression. We can use F-statistics to
        test a group of parameters, or use t-test to test the individual slope coefficients.
    \item Example of Regression application with dummy variables. See Notes directly.
\end{enumerate}
\subsection{Why multiple regression isn't as easy as it looks?}
Pay attention to the assumptions that have been used. Violations like::
\begin{enumerate}
    \item Heteroskedasticity
    \item Serial correlation (auto-correlation)
    \item Multicollinarity
\end{enumerate}
Any violations on the assumptions will impact the estimation of SEE, and finaly 
change the t-statistic and F-statistic, and change the conclustion of the hypotesis
test.
\subsection{Types of Heteroskedasticity, how heteroskedasiticity and serial correlation affect inference}
\begin{enumerate}
    \item What is Heteroskedasticity?
        \\ {\bf Corresponding assumptions: Variance of the residuals is constant across observations. -- Homoskedasticity} 
        Heteroskedasity means the variance of the residuals is not equal.
        \begin{enumerate}
            \item Unconditional heter: Not related to the level of the indepedent
                variables. Will not systematiclly increase with changes in the value
                of the indepedent variables. {\color{red}Usually will not casue major problems.}
            \item Conditional heter: Related to the level of the independent variables. Eg:
                Conditional heter exists if the variance of the residuals increase with 
                the value of the independent variables increases. {\color{red}Will cause big problems.}
        \end{enumerate}
    \item Effect of Heteroskedasiticity on Regression Analysis
        \begin{enumerate}
            \item Unreliable standard errors.
            \item The coefficient estimates aren't affected.
            \item Will change the t-statistic, and will change the conclusion.
            \item Unreliable F-test
        \end{enumerate}
    \item Detect Heteroskedasicity
        \begin{enumerate}
            \item Scatter plot
            \item Breusch-pagan test: $BP test = n\times R^2_{resid}$ with $df=k$. 
                where $n=$the number of observations, $R^2_{resid}$=$R^2$ from a second
                regression of the squared residuals from the first regression. $k=$the number
                of independent variables. If $R^2$ or BP-test are too large, something is wrong.
        \end{enumerate}
    \item Correcting Heteroskedasticity
        \be
        \item Calculate robust sndard errors(White corrected std errors.). Use them for t-test.
            \item Generalized least squares. 
        \ee
    \item What is serial correlations?
        \be
            \item Def: auto-correlation, in which the residual terms are correlated.
                Common problem with time series data.
                \be
                    \item Positive serial correlation: a postive error in one time period will
                        increase the posibility to observe a positive one next time.
                    \item Negative serial correlation: Just opposite.
                \ee
            \item Effect: positive serial correlation will get small coefficient std errors. 
                Thus, too large t-statistics. therefore, too many Type I errors: reject the null
                hypothesis $H_0$ while it's actually true. 
            \item Detection:
                \be
                    \item Residual plots
                    \item Durbin-Watson statistics: 
                        $$
                        \color{red}DW=\frac{\sum_{t=2}^T (\hat{\varepsilon}_t-\hat{\epsilon}_{t-1})^2}{\sum_{t=1}^T \hat{\epsilon}^2}
                        $$
                        For large samples, $DW\approx 2(1-r)$, where $r$ is the 
                        correlation coefficient between residuals from one period and thoese from the previous period.
                        \\Results: 
                        \be
                            \item $DW=2\Rightarrow$ Homoskedasitic and not serially correlated.
                            \item $DW<2\Rightarrow$ Postively serially correlated.
                            \item $DW>2\Rightarrow$ Negatively serially correlated.

                        \ee
                        Formulated hypothesis with DW-table, upper and lower critical values
                        \be
                            \item Hypothesis: $H_0:$ the regression has {\bf no}
                                positive serial correlation.
                            \item $DW<d_l$: positive serially correlated. Reject null.
                            \item $d_l<DW<d_u$: inconclusive results.
                            \item $DW>d_u$: {\bf There is no evidence that are positive correlated.} 
                        \ee
                \ee
            \item Correcting serial correlation:
            \be
                \item Adjust the coefficient std errors. {\bf recommended.} Using Hansen method.
                    \be
                        \item Serial correlation only: Hansen method. 
                        \item Heteroskedasticity only: White-corrected stand errors.
                        \item Both: Hans methods.
                    \ee
                \item Imporoe the specification of the model.
            \ee
    \ee
\end{enumerate}
\subsection{Multicolinearity and its cause and effects in regression analysis}
Multicollinearity: Indepedent variables or linear combinations of independent variables are highly
correlated.
\begin{enumerate}
    \item Effect of Multicollinearity on Regression Analysis: Will increase the std errors of 
        the slope coefficients. {\color{red} Type II Error:
        A variable is significant, while we conclude it's not.} 
    \item Detecting: Common situation: $t-statistic$ is not significant while $F-test$ is significant.
        This tells us the indepedent variables are highly correlated.
        \\ A simple rule works if there are 2 indepedent variables: when the absolute value of the sample correlation betewen any two 
        indepedent variables in the regression is greater than 0.7.
    \item Correcting: omit one or more of the correlated indepedent variables. 
        THe problame is that it's hard to find the variables that result in the multicolinearity.
\end{enumerate}
\subsection{Model misspecification}
\begin{enumerate}
    \item Defination of {\bf Regression model of specification}: decide which independent variables
        to be included in the model.
    \item Types of misspecification
        \be
            \item The functional form can be misspecified:
                important variables are ommited;variables should be transformed; data is improperly pooled.
            \item Explanatory variables are correlated with error term in time series model:
                A lagged dependent variable is used as an independent variable; a function of the dependent
                variable is used as an independent variable(forecasting the past);
                independent variables are measured with error.
            \item Other time-series misspecification.
        \ee
\end{enumerate}
\subsection{Models with qualitive dependent variables}
Include qualitative dependent variables, like default, bankcrupcy. Cannot use an 
ordinary regression model. Should use other models like {\bf probit and logit models}
or {\bf discriminant models}.
\be
\item Probit: normal distribution, give probability.
\item Logistic: logistic distribution. 
\item Discriminant: result in an overall score or ranking.
\ee
% End Reading 10


%% Reading 11
\section{Reading 11: Time-Series Analysis}
\subsection{Calculate/evaluate the predicted trend value for a time series given 
the estimated trend coefficients}
\begin{enumerate}
    \item Linear Trend Model and Log-linear Trend
        \be
            \item Definition: $y_t=b_0+b_1(t)+\epsilon_t$ Note: $t$ is just time.
            \item Coefficients is determined by OLS. Ordinary least squared regression.
                \\$\hat{y}=\hat{b}_0+\hat{b}_1$
            \item Log-linear Trend Models
            \item Model: $y_t=\exp{b_0+b_1(t)}\Rightarrow \ln{y_t}=b_0+b_1(t)$
        \ee
\end{enumerate}
\subsection{Factors that determine whether a linear or a log-linear model trend 
should be used}
\be
    \item Factors that determine which model is best: plot data.
    \item Limitaions of trend models: 
        \be
            \item residuals are uncorrelated with each other.
        Otherwise, it will cause auto correlation and we should not use the trend model.
            \item For log-linear model, it is not suitable for cases with serial correlations (autocorrelation).
            \item Detect auto correlation: Durbin Watson statistic. $DW=2.0\Rightarrow$ No auto correlation.
        \ee
\ee
\subsection{Autoregressive model, requirements for covariance statinoary}
\be
    \item Autoregressive model:
        \be
            \item Model: $x_t=b_0+b_1 x_{t-1}+\varepsilon_t$
            \item Statistical inferences bse on ordinary least squares estimates doesn't
                apply unless the time series is {\bf covariance stationary}.
            \item Conditions for covariance stationary
                \be
                    \item Constant and finite expected value.
                    \item Constant and finite variance.
                    \item Constant and finite covariance between values at any given lag.
                \ee
        \ee
\ee
\subsection{An autogressive model of order $p$}
\be
    \item Model(order $p$): $x_t=b_0+b_1x_{t-1}+b_2x_{t_2}+...+b_px_{t-p}+\varepsilon_t$ 
    \item Forecasting with an autoregressive model: 
        \be
            \item One-period-ahead forecast for $AR(1)$: $\hat{x}_{t+1}=\hat{b}_0+\hat{b}_1 x_t$
            \item Two-period-ahead forecast for $AR(1)$: $\hat{x}_{t+2}=\hat{b}_0+\hat{b}_1 \hat{x}_{t+1}$
        \ee
\ee
\subsection{How the residuals can be used to test the autogressive model}
\be
    \item The residual should have no {\it serial correlation} if an AR model is correct.
    \item Steps
        \be
            \item Estimate: Start with AR(1)
            \item Calculate: the autocorrelations of he model residuals
            \item Test: whether the autoccorelations are signficantly different from 0.
                \\The standard error is $\frac{1}{\sqrt{T}}$ for $T$ observations. The t-test for each
                observation is $t=\frac{\rho_{\epsilon_t,epsilon_{t-k}}}{1/sqrt{T}}$, with $T-2$ df. 
        \ee
\ee
\subsection{Mean reversion and a mean-reverting level}
\be
    \item Mean reversion: The time series tends to move toward its mean.
    \item Mean-reverting level: $\hat{x}_{t+1}=x_{t}$, where $\hat{x}_t$ is the predicted value. 
    \item All covariance stationary time series has finite mean-reverting level. 
\ee
\subsection{Contrast in-sample and out-of-sample forecasts and the forecasting accuracy of 
different time-series models based on the root mean squared error criterion.}
\be
    \item in-sample, out-of-sample: determined by if the predicted data is in the range of 
        the observations.
    \item RMSE, root mean squared error: used to compare the accurancy. If the accurancy of 
        out-of-sample is better, you should use it for future applications
\ee
\subsection{Explain the instability of coefficients of time-series models}
\be
    \item Instability or nonstationarity. Due to the dynamic econimic conditions, 
        model coefficients will change a lot from period to period. 
    \item Shorter time series are more stable, but longer time series are more reliable.
\ee
\subsection{Random walk processes and their comparisons between covariance stationary processes}
\be
    \item Random walk: $x_t=x_{t-1}+\varepsilon_t$
    \be
        \item $E(\varepsilon_t)=0$: The expected value of each error is zero.
        \item $E(\varepsilon_t^2)=0$: The variance of the error terms is constant.
        \item $E(\varepsilon_i,\varepsilon_j)=0$: There is no serial correlation in the error terms.
    \ee
    \item Random walk with a Drift: $x_t=b_0+b_1 x_{t-1}+\varepsilon_t$, where $b_1$=0
    \item A random walk or a random walk with a drift have no finte mean-reverting level.
        Since $b_1=1,\frac{b_0}{1-b_1}=\frac{b_0}{0}$. Therefore, they are not covariance stationary.
    \item $b_1=1$, they exhibit a unit root. Thus, {\bf the least square regression that been used in 
        AR(1) will not work unless we transfrom the data}.
\ee
\subsection{Things about unit roots: when they will occur, how to test them, how to transform data to apply AR}
\be
    \item Unit root testing for nonstationarity:
        \be
            \item run an AR model and check autocorrelations
            \item perform Dickey Fuller test.
                \be
                    \item Transform: $x_t=b_0+b_1x_1+\varepsilon\Rightarrow x_t-x_{t-1}=b_0+(b_1-1)x_{t-1}+\varepsilon$
                    \item Direct test if $b_1-1=0$ using a modified t-test.    
                \ee
        \ee
    \item First differencing
        \be
            \item For a random walk, transform the data $y_t=x_t-x_{t-1}\Rightarrow y_t=\varepsilon_t$
                then start to use an AR model $y=b_0+b_1 y_{t-1}+\varepsilon$, 
                where $b_0=b_1=0$
            \item $y$ is covariance stationary.
        \ee
\ee
\subsection{How to test and correct for seasonality in a time-series model, and 
calculate and interpret a forecasted value using an AR model with a sesonal lag.}
\be
    \item Detect: special autocorrelation exists for some seasonal lags.
    \item Correction: Add an additional seasonal lag term.
\ee
\subsection{Explain autogressive conditional heteroskedasticity (ARCH) and describe
how ARCH models can be applied to predict the variance of a time series}
\be
    \item ARCH: the variance of the residuals in one period is dependent on the variance of
        the residuals in a previous period.
    \item Using ARCH models:
        \\Example $ARCH(1)$: $\hat{\varepsilon}_t^2=a_0+a_1\hat{\varepsilon}_{t-1}+\mu_t$
        if $a_1$ is significantly different from zero. $\hat{\varepsilon}_t^2$ is the squared residuals.
        \\Note: Things like generalized least squares should me used to 
                correct heteroskedasticity. otherwise, the std errors of the 
                coefficients will be wrong, leading to invalid conclusions.
    \item Predicting the variance of a time series: using ARCH model to predict the variance of 

        future periods: $\hat{\sigma}^2_{t+1}=\hat{a}_0+\hat{a}_1\hat{\varepsilon}_t^2$
\ee
\subsection{Explain How time-series variables should be analyzed for nonstationarity and/or cointegration before
use ain a linear regression}
\be
    \item Cointegration: 
        \be
            \item Two time series are economically linked or follow the 
        same trend and that relationship is not expected to change. -- Error terms 
        from regressing one on the other is covariance stationary and the t-test are reliable.
            \item How to test conintegration: regress $y_t$ on $x_t$ 
                 $y_t=b_0+b_1x_t+\varepsilon$, $y_t, x_t$ are two different time series.
                 Then do a unit root test using the Dickey Fuller test with critical t-values
                 calculated by Engle and Granger. 
                 \\If "A unit root" is rejected $\Rightarrow$ covariance stationary, cointegrated.
        \ee
\ee


\section{Reading 12: Probabilistic Approaches: Scenario Analysis, Decision Trees, and Simulations}
\subsection{Describe steps in a simulatino, Explain three ways to define the probability 
        distributions for a simulation's variable, and describe how to treat correlation accross
    variables in a simulation.}
        \be
            \item Steps in simulations:
                \be
                    \item Determine the probabilistic variables
                    \item Define probability distributions for these variables
                        \be
                            \item Option 1: Historical data
                            \item Option 2: Cross-sectional data: estimate the variable from similar 
                                companies.
                            \item Option 3: Pick a distribution and estimate the parameters.
                        \ee
                    \item Checkk for correlations among variables: Use historical data to deterine
                        whether any ststematically related. Strong relations$\Rightarrow$ 1) Allow only
                        one of the variables can be removed. Or 2) Build the rules of correlations
                        into the simulation.
                    \item Run the simulation.
                \ee

        \ee
\subsection{Describe advantages of using simulations in decision making}
\be
    \item Advantages: 1) Better input quality   2) Provides a distribution of expected value
        rather than a point estimate.
\ee
\subsection{Describe some common constraints introduced into simulations}
\be
    \item Constraints: specific limits imposed by users of simulations.
    \item Types of constraints
        \be
            \item Book value constraints:
                \be
                    \item Regulatory capital requirements: minimum capital requirements
                    \item Negative equity
                \ee
            \item Earnings and cash flow constraints: might be imposed to meet analyst expectations
            \item Market value constraints
        \ee
\ee
\subsection{Describe issues in using simulations in risk assessment}
\be
    \item Limitations of using simulations
        \be
            \item Input quality: garbage in, garbage out
            \item Inappropriate statistical distributions
            \item Non-statinoary distributions: parameters will change
            \item Dynamic correlations: correlations between input variables will change.
        \ee
    \item Risk-adjusted value: cash flows from simulations are not risk-adjusted. 
        SHOULD NOT be discounted at risk-free rate.
\ee
\subsection{Compare scenario analysis, decision trees, and simulations}
\be
    \item Scenario analysis: computes the value of an investment under some specific cases.
        Total probability is less than 1.
    \item Decision trees: good when risk is discrete and sequential. Sum of probability is 1
\ee

\section{Reading 13: Currency Exchange Rates: Determination and Forecasting}
\subsection{Calculate and interpret the bid-ask spread}
\be
    \item Exchange rates
        \be
            \item Important things: exchange rate, spot exchange rate, forward exchange rate.
            \item Bid/offer(ask) rates:
                //Bid: The price that bank will buy. Offer: The price that bank will sell.
            \item Foregin Exchange Spread. Unit: "1 pip"$=1/10000=0.0001$. Spread depend on:
                \be
                    \item Spread in the interbank market. (Currencies, time, market volatility)
                    \item Size of transaction.
                    \item Relationship between the dealer and client.
                \ee
        \ee
\ee
\subsection{Identify a triangular arbitrage opportunity and calculate its profit}
\be
    \item Example: USD/AUD. USD is the price currency, and AUD is the base currency.
        \be
            \item Buy the base currency at the ask $\Rightarrow$ Sell the price currency at the ask
            \item Sell the base currency at the bid $\Rightarrow$ Buy the price currency at the bid
        \ee
    \item For investors,Rule: {\it up-the-bid-and-multiply, down-the-ask-and-divide}
        \be
            \item Convert USD into AUD: going down the quote -- from USD on top 
                to AUD on bottom. Use the ask price for the quote.
            \item Convert AUD into USD: similar. But from bottom to top.
        \ee
    \item Cross Rate: The exchange rate between two currencies with the help by 
        a common third currency.
    \item Cross Rate with bid-ask spreads.
        \be
            \item Rule 1:
            $$
            \left(\frac{A}{C}\right)_{bid}=\left(\frac{A}{B}\right)_{bid}
            \times \left(\frac{B}{C}\right)_{bid};
            \left(\frac{A}{C}\right)_{offer}=\left(\frac{A}{B}\right)_{offer}
            \times \left(\frac{B}{C}\right)_{offer}
            $$
            \item Rule 2:
            $$
            \left(\frac{B}{C}\right)_{bid}=\frac{1}{\left(\frac{C}{B}\right)_{offer}};
            \left(\frac{B}{C}\right)_{offer}=\frac{1}{\left(\frac{C}{B}\right)_{bid}}
            $$
        \ee
    \item Triangular Arbitrage: If the dealer's quote is different from the cross
        rate, arbitrage opportunities may exist. Check it with Notes.
\ee
\subsection{Distinguish between spot and forward rates and calculate the forward
premium/discont for a given currency}
\be
    \item Forward premium relative to a second currency: 
        Forward price > Spot price. Forward premium = $F-S_0$
    \item Calculate the market-to-market value of a forward contract
            $$
                V_T=(FP_T-FP)(contract size)
            $$
            where:
            \be
                \item $V_T$ = value of the forward contract at time $T$, denominated
                    in price currency
                \item $T$ = maturity of the forward contract
                \item $FP$ = forward price locked in at inception to buy base currency
                \item $FP_T$ = forward price to {\bf sell} the same currency at time $T$
            \ee
    \item Value prior to expiration.
    $$
    V_t=\frac{(FP_t-FP)contract size}{1+R(\frac{days}{360})}
    $$
    where 
    \be
        \item $V_t$ is the value of the forward price
        \item $FP_t$: forward price at time $t$
        \item $days$ number of days remaining
        \item $R$ interest rate
    \ee
\ee
\subsection{Explaining international parity relations (covered and uncovered 
interest rate parity, purchasing power parity, and the international Fisher effect)}
\be
    \item Covered interest rate parity: ``Covered" means bound by arbitrage. 
        Investor should earn the same return using either currency.
        $$
            F=\frac{1+R_A(\frac{days}{360})}{1+R_B(\frac{days}{360})}S_0
        $$
    \item Uncovered interest rate parity: Forward currency contract is unavailable, 
        which makes the interest rate not bound by arbitrage.
        For a quote A/B, the base currency is expected to apprecitate
        $$
            E(\%\Delta_S)_(A/B)=R_A-R_B
        $$
        Uncovered interest rate parity can only {\bf forcast} the future spot exchange rate.
    \item Comparing covered and uncovered interest parity:
        \be
            \item Covered interest parity $\Leftrightarrow$ No-arbitrage forward rate
            \item Uncovered interest parity $\Rightarrow$ {\bf Expected} future spot rate
        \ee
    \item International Fisher Relation
        \be
            \item $R_{nominal}=R_{real}+E(inflation)$
            \item Under real interest rate parity, the real interest rate are 
                assumed to converge across different markets.
                $$
                    R_{nominal A}-R{nominal B}=E(inflation_A)-E(inflaction_B)
                $$
        \ee
    \item Purchasing Power Parity: Assumed by one price law.
        \be
            \item Absolute purchasing power parity: The average price of a basket
                of consumption goods.
                $$
                    S(A/B)=CPI(A)/CPI(B)
                $$
                May not hold due to different weights of consumptions.
            \item Relative Purchasing Power Parity: Changes in exchange rates 
                should exactly offset the price effects of any inflation differencital b
                between the two contries.
                $$
                \%\Delta S(A/B)=Inflation_A-Inflation_B=change in spot price(A/B)
                $$
                Not always held in short run.
            \item Ex-Ante Version of Purchasing Power Parity: Similar to relative PPP,
                but Ex-Ante uses expected inflation instead of actuall inflation.
        \ee
\ee
\subsection{Describe the relations among the international parity conditions}
See Notes Page 263, Vol. 2.
\subsection{Evaluate the use of the current spot rate, the forward rate, purchasing
parity and uncovered interest parity to forecast future spot exchange rates}
\be
    \item Real Exchange Rate = $S_t\left[\frac{CPI_B}{CPI_A} \right]$, $S_t$ is 
        the spot rate at time $t$ given as A/B
\ee
\subsection{Explain how flows in the balance of payment accounts affect currency
exchange rates}
\be
    \item Balance of Payments: accounting method to track transactions between a
        country and its international trading partners.
        \be
            \item Incluing government, consumer, and business transactions.
            \item current account + financial account + offcial reserve account = 0
            \item
                \be
                    \item Current account: Exchanges of goods/services, exchanges of
                        investment income and unilateral transfers like gifts.
                        \be
                            \item Surplus: we sell more to other countries, buy less from them
                            \item Deficit: we buy more from the rest, sell less to them
                        \ee
                    \item Financial account/Capital account: Flows of funds for debt
                        and equity investment into/out of a country. Surplus: Money is
                        flowing into the country.
                    \item Offcial resreve: thoes made from the reserves held by the government.
                        Normally doesn't change from year to year.
                \ee
        \ee
    \item Influence of BOP on Exchange Rates
        \be
            \item Current Account
                \be
                    \item Flow mechanism
                        \be
                            \item Deficit: increase the supply of that currency in the market. 
                                Because exporters to our countries need to convert
                                their revenue to their own currency. $\Rightarrow$ Down on the exchange
                                value.
                            \item Depreciation of the currency may rebalance the 
                                current account. Depending on {\bf The initial deficit, the influence
                                of exchange rates on import/export prices, price elasiticity of traded goods.}
                                See Notes P265 for details.
                        \ee
                    \item Portfolio Composion mechanism. Countries with current account surpluses
                        usually have capital account deficits, which typically take the form of investments
                        in countries with current account deficits. As a result of these flows of capital, investor
                        countries may find their portfolios' composition being dominated by few investee currencies. 
                        When investor countries decide to rebalance their investment portfolios, it can have a significant negative
                        impact on the value of those investee country currencies.
                    \item Debt sustainability mechanism: Current account deficit may be balanced
                        by borrowing money from other countries. If the debt too high, lenders may question
                        the security, leading to the deprecitaion of the borrower's currency.
                \ee
            \item Capital Account Influences: Money flow in$\Rightarrow$Demand for 
                my country's currency increases$\Rightarrow$Appreciation. 
                \be
                    \item Good: can help to overcome a shortage of internal savings
                    \item Bad: Too much money can be problematic for emerging markets.
                        \be
                            \item Excessive appreciation of the domestic currency
                            \item Financial asset, real estate bubbles
                            \item Increase in external debt
                            \item Excessive consumption in the domestic market funded by credit
                        \ee
                \ee

        \ee
    \item real exchange rate (A/B) = equilibrium real exchange rate (A/B) + (real interest rate$_B$ - real interest rate$_A$)
        -(risk premium$_B$ - risk premium$_A$) \\
        This eqution is not precise. We cannot use it to calculate the rate.
    \item Taylor Rule
        $$
            R=r_n+\pi+\alpha(\pi-\pi^\ast)+\beta(y-y^\ast)
        $$
        \be
            \item R = Central bank policy rate implied by the Taylor Rule
            \item $r_n$ = Neutral {\bf real} policy interest rate
            \item $\pi$ = Current inflation rate
            \item $\pi^\ast$ = Central bank's target inflation rate
            \item $y$ = log of current level of output
            \item $y^\ast$ = log of central bank's target (sustainable) output
            \item $\alpha, \beta$ = policy response coefficients. (suggested value: 0.5 for both)
        \ee
        $$
        Real interest rate = r = R-\pi=r_n+\alpha(\pi-\pi^\ast)+\beta(y-y^\ast)
        $$
        Substitute the real interest rate equation, we have\\
        Real exchange rate (A/B) = equilibrium real exchange rate(A/B) + differenc
        in neutral real policy interest rate(B-A)+$\alpha$[difference in inflation gap (B-A)]+
        $\beta$[difference in output gap(B-A)]-(risk premium$_B$-risk premium$_A$)
        //Where: Inflation gap = current inflation - target inflation, 
        Output gap = current output - target output
\ee
\subsection{Explain approaches to assessing the long-run fair value of an exchange rate}
\be
    \item The ex-ante version of relative PPP holds $\Rightarrow$ The real exchange rates constant. 
        However, relative PPP does not necessarily hold over the short term. Over long
        term, PPP holds, and the real rate will be near its equilibrium level.
    \item IMF asseses long-term equilibrium real exchanges rate based on
        \be
            \item Macroeconomic balance approach: if the Ex rates need to be adjusted to 
                equalize the expected current account imbalance and the sustainable current
                account imbalance.
            \item External sustainability approach. How rates need to be adjust to force
                a country's external debt relative to GDP towards its sustainable level.
            \item Reduced-form econometric model approach. 
        \ee
\ee
\subsection{Describe the carry trade and its relation to uncovered interest rate
parity and calculate the profit from a carry trade.}
\be
    \item FX carry trade: Invest in a higher yielding funding with the funds borrowed in
        a lower yielding currency. This is due to the uncovered interest rate parity may not
        hold.
    \item Risk of the Carry Trade
        \be
            \item The exchange rate may change abruptly.
            \item The return distribution is not normal. Negative skewness and 
                excess kurtosis (fat tails). $\Rightarrow$ High probabilithy of large loss
        \ee
    \item Risk Management in Carry Trades
        \be
            \item Volatility filter: if volatitlity > certain threshold, close the carry trade.
            \item Valuation filter: valuation band for each currency based on PPP. If the value
                of a currency falls below the band, we will increase its ratio.
        \ee
\ee
\subsection{Describe the Mundell-Fleming model, the monetary approach and the asset
market approach to exchange rate determination.}
\subsection{Forecast the direction of the expected change in an exchange rate based on
balance of payment, Mundell-Fleming, monetary, and asset market approaches to exchange rate determination.}
\subsection{Explain the potential effects of monetary and fiscal policy on exchange rates.}
\be
    \item Mundell-Fleming Model: evaluate the impact of monetary and fiscal policies on interest rates, 
        and therefore on exchange rates.
    \item Flexible Exchange Rate Regimes: rate are determined by markets.
        \be
            \item High Capital Mobility: Expansionalry M and F are likely to have
                opposite effects. Expansionary M will reduce the interest rate, 
                reduce the inflow of  capital investment, reduce the demand for 
                domestic money, depprecation.
            \item Low Capital Mobility: Uncertain
            \item Summary:\\
                \begin{tabular}{c c c}
                \hline
                Monetary/Fiscal & High Capital Mobility & Low Capital Mobility \\
                \hline
                Expan/Expan &   Uncertain   & Depreciation \\
                Expan/Restr &   Depreciation & Uncertain    \\
                Restr/Expan &   Appre       & Uncertain     \\
                Restr/Restr &   Uncertain   & Appreciation  \\
                \hline
                \end{tabular}   
            \item Fixed Ex rate regimes
                \be
                    \item If monetary expansionary (depreciation), governments need to buy money in the 
                        FX market, therefore will reverse the effect from monetary expansionary.
                    \item Fiscal expansionary \ra Appreciation(More money needed) \ra Government need to 
                        sell money to keep Ex rate stable. \ra Fiscal effect on aggregate demand will be reinforced.
                \ee
        \ee
    \item Monetary Approach to Exchange Rate determination\\
        Inflation play no role in exchange rate in Mundell-Fleming model.
        \\Assumptions: 1. Output is fixed.
        \be
            \item Method 1: Pure Monetary model. Assume: PPP holds, output is constant.
            \item Dornbusch overshooting model. Price are inflexible in short term.
                Expan Monetary \ra price increase, interest rate down \ra depreciatoin of currency.
                Therefore, in short term, price sticky, interest rate down too much. \ra depreciation is greater
                than PPP implies.
        \ee
    \item Portfolio Balance Approach to Exchange rate determination.
        \be
            \item It focues on long-term implications of fiscal policy on currency values.
            \item Fisical deficit\ra sell bonds\ra When investors thinks the country is safe, they will continue to buy bonds. 
                If the investors refuse to fund the deficits \ra depreciation
        \ee
    \item In short term, with free capital flows, expan fiscal \ra appreciation\\
        Long term\ra government has to reverse expan fiscal. Otherwise, investor
        will refuse to fund it, then the country have to monetize its debt (print money).\ra depreation

\ee
\subsection{Objectives of central bank intervention and capital controls and describe
the effectiveness of intervention and capital controls.}
See Notes P274. Old version.
\subsection{Describe warning signs of currency crisis.}
\be
    \item Terms of trade deteriorate
    \item Foreign reserve down quickly
    \item Real exchange rate is extremly higher than mean-reverting value.
    \item INflation increases.
    \item Equity markets have a boom-bust cycle.
    \item Money supply relative to bank reserves increases.
    \item Nominal private credit grows.
\ee
\subsection{Technical analysis}
See Notes P275 Old version.


\section{Economic Growth and the Investment Decision}
\subsection{Compare factors favoring and limiting economic growth in developed and developing economies}
Two important factors. 1: GDP per capita. 2: Growth of GDP
\be
    \item Preconditions for Growth
        \be
            \item Saving and investment. Positively correlated with economic development.
            \item Financial markets and intermediaries. Help resources reallocation. 
                However, it may increase leverage, risks.
            \item Political stability, rule of law and property rights.
            \item Investment in human capital. Worker's skills.
            \item Tax and regulatory systems. Lower tax burdens are good.
                Lower regulation levels are good.
            \item Free trade and unrestriced capital flows.
        \ee
\ee
\subsection{Describe the relation between the long-run rate of stock market appreciation and
the sustainable groth rate of economy.}
The growth in the price is related to earnings and GDP: 
$\Delta_P=\Delta_GDP + \Delta(E/GDP)+\Delta(P/E)$.
Over the long-term, $\Delta(E/GDP)=0,\Delta(P/E)=0$. Only GDP growth matters.
\subsection{Explain why potential GDP and its growth rate matter for equity and fixed income investors.}
Higer GDP growth\ra Higher interest rates \ra Higer real asset returns.\\
Higher GDP growth makes people think that future income is increasing, therefore increase consumptions
and reduce savings. To encourage consumers save, higher interest rate is needed.

In short term, actual GDP in excess of potential GDP will result in rising prices \ra inflationary pressure.
\subsection{Distinguish between capital deepening investment and technological progress and explain
how each affects economic growth and labor productivity}
\be
    \item Factor input and Economic growth
        \be
            \item Model: 2-factor aggregate production: Y~F(L,K) at a level of tech T.
                Output Y is a function of labor(L) and capital.
            \item Cobb-Douglas Production: $Y=TK^\alpha L^{(1-\alpha)}$
                \\Dividing both sides by $L$, the output per worker is 
                $$
                    Y/L=T(K/L)^\alpha
                $$
        \ee
\ee
\subsection{Forecase potential GDP based on growth accounting relations}
\be
    \item Growth Accounting Relations
        $$
        \Delta Y/Y=\Delta A/A+\alpha\times(\Delta K/K)+(1-\alpha)(\Delta L/L)
        $$
        i.e. growth rate in potential GDP = long-term growth of tech 
        + $\alpha$ long-term growth rate of capital
         + (1-$\alpha$)*(long-term growth rate of labor)
        \\ The growth of technology is not observable. Can be estimated from previous
        data: ex-growth rate - ex-growth rate from L and K
\ee
\subsection{Explain how natural resources affect economic growth and evaluate the arument
that limited availability of natural resources constrains economic growth}
\be
    \item Access to natural resouces does not require ownership of resources.
    \item Another theory: ownership of natural resources may actually inhibit growth. 
        \ra Dutch disease: global demand for natural resouces  drives up the coutries
        currency, making all other exports more expensive and uncompetitive.
\ee
\subsection{Explain how demographics, immigration, and labor force participation
affect the rate and sustainability of economic growth}
\be
    \item Labor Supply Factors
        \be
            \item Demographics: A countries age distribution. Countries with younger
                age will have a higher potential growth.
            \item Labor force participation.
            \item Immigration: a potential source in developed countries \Ra increase work force
            \item Average hours worked
        \ee
\ee
\subsection{Explain how investment in physical capital, human capital, and technological
development affects economic growth}
\be
    \item Human capital: knowledge and skills that individuals possess. Can be enhanced
        via education.
    \item Physical capital: infrastructure, computers, telecommunications(ICT)
        AND non-ICT capital(machineary, transportation and non-residential construction).
        More investment in physical capital \Ra Good GDP growth.
        \\ MOre investment may enhance the tech improvements.
    \item Technological development. Investment in tech will increase the productivity.
    \item Public infrastructure: like roads, bridges, and municipal facilities. 
        This will enhance total productivity. Because the private investment will not
        invest these public things for their little returns.
\ee
\subsection{Compare classical growth theory, neoclassical growth theory, and endogenous
growth theory}
\be
    \item Classical growth theory: In the long-term, population growth increases whenever
        there are increases in per capita income above subsistence level due to increase
        in capital or tech progress. \Ra Growth in real GDP per capita
        is not permanant. \Ra This is not supported by observed facts.
    \item Neoclassical Growth theory:
        \be
            \item Estimate steady state growth rate. Equilibrium economy is when the output-to-capital
                ratio is constant. When the output-capital ratio is constant, the labor-to-capital
                ratio and output per capita also grow at the equilibrium rate. Check textbooks here.
            \item Based on Cobb-Douglas function, 
                \be
                    \item Sustainable growth of output per capita: $g^\ast=\frac{\theta}{1-\alpha}$,
                        where $\theta$ is the growth rate in techonology, and $1-\alpha$ 
                        is the labor's share of GDP.
                    \item Sustainable growth rate: $G^\ast=\frac{\theta}{1-\alpha}
                        +\Delta_L$, which is the growth rate of output per  capita plus
                        the growth of labor.
                    \item Comments
                        \be
                            \item Caital deepening will not affet the growth rate in the long run.
                        \ee
                \ee
        \ee
    \item Endogenous Growth Theory
        \be
            \item Technological growth is a result of investment in physical and human capital. Returns to capital are constant.
            \item Private investments in R\&D also benefits all economy.
        \ee
\ee
\subsection{Explain and evaluate convergence hypoteses}
\be
    \item Absolute convergence: Less developed countries will achieve equal living standards
        overtime.
    \item Conditional convergence: Convergence in living standards will only occur for countries with the same savings rates, population
        growth rates and production functions. 
    \item Club convergence: Countries may be part of a club. Poorer countries that are part of the club will catch up
        their richer peers. Institutional changes can help a country to join the club. 
        Those are not in the club will never catch up.
\ee
\subsection{Describe the economic rationale for governments to provide 
incentives to private investments in technology and knowledge.}
R\&D are risky. Governments support may provide incentives to private R\&D, and therefore
boosts the growth of the overall economy.
\subsection{Describe the expected impact of removing trade barriers on capital investment 
and profits, employment and wages, and growth in the economies involved.}
\be
    \item Increased investment from foreign savins
    \item Allows focus on industries where the country has advantage.
    \item Increased markets
    \item Increased sharing of tech
    \item Increased competition removes bad firms and rellocating assets.
\ee

\section{Reading 15:Economics of Regulation}
\subsection{Describe classifications of regulations and regulators}
\be
    \item Regulations: i. Statues; ii. Administrative regulatinos; iii. judicial law
    \item Regulators: government agencies/indepedent regulators/outside bodies. Independent regulators
        including self-regulating organizations that regulates and represents their members.
        Outside bodies will not regulate, but their products are referenced by regulators.
\ee
\subsection{Describe uses of self-regulation in financial markets}
\be
    \item US: FINRA is an SRO recognized by SEC.
    \item In civil-law countries, independent SROs are rare, and government agencies
        fulfill the role of SROs.
    \item In common=law countries, independent SROs are historically good.
\ee
\subsection{Describe the economic rationale for regulatory intervention}
\be
    \item Economic Rationale for Regulation. Regulations are needed when
        \be 
            \item Information frictions. When information is not equally availabel or distributed.
            \item Externalities. 
        \ee
\ee
\subsection{Describe regulatory interdependencies and their effects.}
\be
    \item Regulatory Interdependencies.
        \be
            \item Regulatory capture theory: Regulators will at some point in time
                be influenced or controlled byt the industiry that is being regulated.
                Because the regulators will be influenced by the industry, and the
                experience will sometimes lead to impartial conclusions.
            \item Regulatory competition: Regulatory difference betweeen jurisdictions will lead to it.
                Regulators compete to provide the most business-friendly environment.
            \item Regulatory arbitrage: businesses find a coutry that best for itself.
        \ee
\ee
\subsection{Describe the tools of regulatory intervention in markets}
\be
    \item Price mechanisms
    \item Restricting/requiring certain activities. Ban certain activities or 
        require to perform some activities.
    \item Provision of public goods or financing private projects. 
\ee
\subsection{Explain purposes in regulating commerce and financial markets}
\be
    \item  Regulating commerce: Government regulations, including company laws, tax laws,
        contract laws, competition laws, banking laws, bankruptcy laws and dispute resolution systems.
        \\This laws may help or hinder commerce.
    \item REgulating financial markets: regulation of securities markets and regulation of 
                financial institutions. -- protecting investors, creating confidence in
                the markets, and enhancing capital formation.
        \be
            \item Regulation of security markets
            \item Regulation of Financial Institutions: Prudential supervision,
                the monitoring and regulation of financial institutions to reduce
                system-wide risks and to protect investors. Cost-benefit analysis should include
                hidden costs.
        \ee
\ee
\subsection{Describe anticompetitive behaviors targeted by antitrust laws globally and evaluate
the antitrust risk associated with a given business strategy.}
Antitrust regulation works to promote domestic competition. Like blocking a merger that leads to
excessive concentration of market share. International companies may be subject to antitrust
laws in different countries.
\subsection{Benefits and costs of regulation}
Costs include the implementation cost and the cost of the regulation to the private
sector.
{\it Regulatory burden}: the cost of compliance for the regulated entity. 
Regulatory burden - benefits that private sector receives = Net regulotry burden.
\subsection{Evaluate how a specific regulation affects and industry companly, or security}
\be
    \item Can help or hinder the industry or the company.
    \item Not necessarily always costly for those that being regulated.
    \item May introduce inefficiencies in the market. For example, government bailout of financial
        institutions may convey a message that companies will be helped. And the credit
        spreads will not fully reflect their risk.
    \item Certain industries have more exposure to certain types of regulations.
\ee


\section{Reading 16: Intercorporate Investments}
\subsection{Describe the classification, measurement, and disclosure under IFRS for
1) Investments in financial assets, 2) Investments in associates, 3) joint ventures, 4)
bussiness combinations, and 5) special purpose and variable interest entities}
\subsection{Distinguish between IFRS and US GAAP in the classification, measurement, and disclosure
of things above.}
\be
    \item Classification: a. Investments in financial assests. b. Investments in associates (investing
        firm has a significant influence but not control). c. Bussiness combinations.
        \be
            \item Financial assets. Ownership $<$ 20\%. Accouting treatment:
                \be
                    \item IFRS: held-to-matruity, available-for-sale, fair value through profit/loss.
                    \item GAAP: similar to IFRS.
                    \item IFRS 9 start from 1/1/2018. Early adoption is allowed.
                \ee
            \item Investments in associates. Ownership 20\%$\sim$50\%. Most important thing is ``significant
                influence.'' Equity method is used.
            \item Bussiness combinations. Ownership $>$ 50\%. ``Controling" is important. The acquision method is used.
            \item Joint ventures. Equity method for it.
        \ee
    \item Reporting of Intercorporate investments
        \be
            \item Financial assets. Acquisition is recorded at cost, and dividend or interest income
                is in Income Statement.
                \be
                    \item Held-to-maturity. Debt securities that cannot be sold prior to maturity
                        excepte in unusual circumstances. Long-term: reported on the balance sheet at the
                        amortized cost. Interest income(coupon cash flow adjusted for amortization
                        ) in the income statement but subsequent changes in fair value are ignored.
                    \item Fair value through profit or loss.
                        \\a. Held-for-trading: Debt/equity for the purpose of profiting in the near term. $<$ 3 month
                        Changes in fair value(realized or not) and dividend/interest are in Income State.
                        \\b. Designated at fair value: Report debt/equity that may be treated as held-to-maturity
                        or available-for-sale at fair value. Gains/Loss are in Income State.
                    \item Available-for-sale: are neither held-to-m nor held-for-trading. Reported on the balance at the
                        fair value. However, only the realized gains/losses and dividend or interest income are in the
                        income statement. Unrealized gains/losses are in comprehensive income. When sold, these unrealized
                        things will move to income statement.
                        \\NOTE: In IFRS, unrealized gains or losses on available-for-sale {\bf sale} 
                        that from foreign exchange movements are in the income
                        statement. In USGAAP, the entired unrealized gain/loss are recognized in equity.
                        \\ Available-for-sale {\bf Equity}, the treatments are similar in IFRS/USGAAP.
                \ee
        \item Reclassification of Investments in Fiancial Assets.
            \be
                \item IFRS: a.does not allow reclassiication into/out of the designated at fair value. b.Out of the held-for-trading
                    is severely restricted. \\c.Debt securities in available-for-sale can be reclassified as held-to-maturity. The value will be 
                    remeasured to reflect its fair value at the time it is reclassified. 
                    \\d. Held-to-maturity: can be reclassified as available-for-sale. Carrying value is remeasured
                    t the fair value, and any difference is recoreded in comprehensive income.
                \item USGAAP: 
                    \be
                        \item Permit reclassification into/out of held-for-trading or designated at fair value.
                            Unrealized gains on the income stmt is reclassified. 
                        \item Reclassificiation out of available-for-sale to held-for-trading, the cumulative g/l
                            in comprehensive income will be recognized in income.
                        \item Out-of-available for sale to held-to-maturity: G/L in comprehensive income stmt will
                            be amortized over the remaining life of the security.
                            \item Out of held-to-maturity to available-for-sale: Unrealized G/L goes into comprehensive income
                            stmt.
                        \item Summary\\
                            \begin{tabular}{ccc}
                            \hline
                            From & To & Unrealized G/L \\
                            \hline
                             Fair value w/ G/L*   & Any   &Income Statement\\
                             Held-to-maturity   & Fair value*  &  Income statement \\
                             Held-to-mat   & Available-for-sale   & Other comprehensive income\\
                             Available-for-sale & Held-to-maturity & Amortize out of other comprehensive income\\
                            Available-for-sale  & Fair value w/ G/L* & Transfer out of other comprehensive income\\
                            \hline
                            *: Restricted under IFRS
                            \end{tabular}
                    \ee
            \ee
        \item Impairment of Fianancial Assets: Held-to-Maturity(HTM) and available-for-sale (AFS) evaluabed for impairment
            at each reporting period.
            \be
                \item US GAAP. If its decline in value is determined to be other than temporary. HTM/AFS, the write-down
                to fair value is treated as a realized loss.
                \item US GAAP Reversals: NOT allowed.
                \item IFRS: Impairments are in the income stmt. Impairment of a debt or equity seurity
                $\Leftarrow$ at least one loss evet HAS occured, and its effect
                on the security's future cash flows can be estimated reliably.
                \\Debt: loss events -- default on payments
                \\Equities: loss evets -- fair value has experiened a decline, and
                it's unlikly to recover.
                \\HTM security: if it's impaired, its carrying value will be the 
                PV of future cash flows, using the interest rate when the security
                was PURCHASED.
                \item IFRS Reversals: Permited on HTM, available-for-sale. Not permitted
                    for equity.
            \ee
        \item Analysis of Investments in Financial Assets: Seperate a firm's operating
            results from its investment results. 
            \\For comparison: market values for financial assets.
            \\ Remove nonoperating assets when calculating the return on operating
            assets ratio.
            \\ Investment classification will be misleading.
        \item IFRS 9 (New standards)
            \be
                \item Instead of HTM, availabel-for-sale, and held-for-trading,
                    Using new terms: amortized cost, fair value through proft or loss
                    (FVPL), and fair value through other comprehensive income (FVOCI)
                \item Amortized cost (For debt securities only): If the securities are
                    \be
                        \item Business model test: being held to collect contractual cash
                            flows
                        \item Cash flow characteristic test: the contractual cash
                            flows are eithe principal or interest on principal only.
                    \ee
                \item Fair Value Through Profit or Loss (for debt and securities)
                    \be
                        \item Debt: can be FVTPL if held-for-trading, or Amortized cost
                            results in an accounting mismatch.
                        \item Equity: Must be FVTPL if it is held-for-trading. Other equity
                            can be FVTPL or fair value trhough OCI, but once classified, the choice
                            cannot be reversed.
                    \ee
                \item Fair Value Through OCI (Equity Only)
                    =Available-for-sal
            \ee
        \item Reclassification under IFRS 9
            \be
                \item Reclassification of Equity is not allowed.
                \item  Reclassification
                    of debt from amortized cost to FVPL or vice versa is OK only 
                    if the business model has changed. Unrecognized G/L on debt securities
                    carried at amortized cost and reclassified as FVPL are in the income stmt.
                \item Debt that out of FVPL: measured at amortized cost transfered at fair value
                    on the transfer date, and the fair value becom the carrying amount.
            \ee
        \item Investments in Associates
            \be
                \item Using equity method.
                \item Initial investment is recorded at cost and reported on balance
                    sheet as a noncurrent asset.
                \item Subsequent: proportional share of earnings increases 
                    the investment account on the balance
                    sheet, and is recognized in the income stmt. 
                    Dividends recevied are treated as a return of capital, and 
                    reduce the investment account, will no be in income stmt.
                \item If the investee has a loss, investor will have a proportionate loss
                    in balance, and income stmt.If the investment account on balance sht reduce to
                    0, we stop using equity method until the earnigs recovered.
            \ee
        \item Fair Value Option
            \be
                \item USGAAP allows equtiy method investment to be recored at fair value.
                \\ IFRS: the fair value only good to venture capital firms, mutal funds and similar firms.
                \\Decision to use FVO is irrevocable. If use FVO, any changes are in income stmt.
                \item Excess of Purchase Price Over Book Value Acquired. 
                    \be
                        \item At the acquisition date: the excess of the purchase
                            price over the proportionate share of book value is allocated
                            to the investee's identifiable assets and liabilities based on their fair
                            values, and in investor's balance sheet. The remainder is good will.
                        \item Investor need to recognize expense based on the excess amount assigned 
                            to the investee's asset and liabilities.
                        \item Investor may need to include additional depreciation
                            proportionate of the Excess of purchase price. See Notes-2 P11
                            examples and textbooks.
                    \ee
                \item Impairments of Investments in Associates:
                    \be
                        \item Equity method investments need tests for impairment.a
                        \item GAAP: Fair value of the investment < the carrying value(investment account on 
                            the balance sheet), and decline is permanent. Write-down to fair value,
                            loss in income stmt.
                        \item IFRS/GAAP: asset cannot write up
                    \ee
                \item Transactions with the Investee
                    Profit from these transactions deferred until the profit is
                    confirmed through use/sale to a third party.
                    \be
                        \item Upstream(investee to investor): investee recognized
                            all profit in its income stmt. Eliminate its proportionate
                            share of the profit.
                        \item Downstream(investor to investee). Investor recognize
                            profit in its income stmt. Eliminate the proportionate share
                            of unconfirmed profit.
                    \ee
                \item Analytical Issues for Investments in Associates.
                    \be
                        \item Equity method may have higher earnigs.
                        \item Investor only report investee's proportionate share of equity.
                            Investee's debt are ignored, and leverage is lower.
                        \item Proportionate shaore of Investee's earnings may be reinvested, not
                            available to investor via dividend.
                    \ee
            \ee
        \item Bussiness Combinations
            \be
                \item Classification:
                    Acquisition method is required for business combinations.
                    \be
                        \item IFRS: None
                        \item GAAP:
                                \\-Merger. Acquiring firm survived.
                                \\-Acquisition. Acqing and Acqed continue to exist in 
                                    a parent-subsidiary form. Not 100\% of sub is owned by parent.
                                \\- Consolidation. A new entity absorbs both of companies.
                    \ee
                \item Accounting methods
                    \be
                        \item Purchase method
                        \item Pooling-of-interests method (eliminated)=
                            uniting-of-interests method in IFRS
                            \\- Just combine assets and liabilities.
                            \\- Two companies combined using historical book vales
                            \\- Operating results are restated, as two have been always combined.
                            \\- Ownership interests continue, and former accounting bases maintained.
                        \item Acquisition method
                            \\- A, L, Revenue, Expense of sub are combined with the parent. Intercompany
                            transactions are excluded. Stock holder's equity is ignored. Because this
                            is not controlled by the company.
                            \\- noncontronling interest account may be needed for proportionate asset that
                            are not owned by the parent. Check examples.
                        \item Good will in acquisition    
                            \\- Goodwill. Fair value for identifiable assets/liabilities.
                            Any remainder will be unidentifiable A/L \Ra Goodwill
                            \\--GAAP: full goodwill. Fair value of the sub(calculated by acquired ratio) - Fair value 
                            of net identifiable net assets of the subsidiary; IFRS can use full goodwill
                            or partial goodwill, partical goodwill = purchase price - (\%owned*FV of net identifiable
                            asset)
                            \\Noncontrolling interest: For Full good will - nci based on the 
                            acquired company's fair value. For Partial goodwill -
                            nci based on the fair value of the acquired company's identifiable net
                            assets.
                        \item Annual test impariment of Goodwill
                            \\ IFRS: carrying amount of cast generting unit $>$ the recoverable amount
                            \\ GAAP: 2 steps. Carrying value of the reporting unit $>$ the fair value; 
                            the loss = carrying value of the good will $-$ the implied fair value of the goodwill.
                    \ee
            \ee
        \item Bargain Purchase: If acquisition price $<$ fair value of net asset aquired, gain should be in income stmt for 
            GAAP and IFSR.
        \item Joint Ventures: 
            \be
                \item One entitiy shared by multiple investors. Equity method in GAAP and IFRS.
                \item Rare case: Proportionate consolidation method is OK for GAAP/IFRS. 
                    In Proportionate consolidation, investor only reports the proportionate
                    share of assets, l, reve, expense. No minority owner's interest.
            \ee
        \item Special Purpose and Variable Interest Entities
            \be
                \item SPE: Special purpose entity. Isolate certain A and L of the sponsor.
                    SPE is often off-balanced-sheet. Thus enhance the ratios.
                \item VIE is a special SPE in FASB. Consolidated by the primary beneficiary.
                    VIE Characteristics:
                    \be
                        \item At-risk equity, insufficient to finance the entity's activities without additional support.
                        \item Equity investors lack: decision making rights, obligation to absorb loss, or the right
                            to recevie expected returns.
                    \ee
                \item IFRS: Sponsoring entity must consolidate if it controls SPE.
            \ee
    \ee
    \item Analyze how different methods used to account for intercorportate
        investments affect financial statements and ratios.
        \\ Four important effects, Equity/Proportionate consolidation/Acquisition
        \be
            \item All 3 report the same Net Income.
            \item Equity: Equity method = Proportionate = Acquisition method - minority interest
            \item Assets and Liabilities: Acquision $>$ Proportionate consolidation $>$Equity 
            \item Revenues and expenses: Acquision $>$ Proportinate consolidation $>$ Equity
        \ee
\ee


\section{Reading 17: Employee Compensation: Post-Employment and Share-Based}
\subsection{Descreibe the types of post-employment benefit plans and implications for financial reports}
Types: Defined-contribution plan and Defined-benefit plan. For defined-contribution plan, accounting is easy. Just
the employer's contribution.\\
Defined-benefit:
\be
    \item Usually via a separate legal entity, like a trust.
    \item funded status: difference between the pension obligation and the plan assets.
    \item Other post-employment benefits: basically healthcare benefits.
\ee
\subsection{Explain and calculate measures of a defined benefit pension obligation, i.e., 
PV of the defined benefit obligation and projcted benefit obligation, and net pension liability.}
\be
    \item Things to know
        \\-Projected Benefit Obligation, or Present Value of Defined Benefit Obligation under IFRS.
        :The PV of all future obligation, based on expected future salary increases.
        PVB will change from one period to the next.
        \be
            \item Current service cost: PV of benefits earned by during the current period.
            \item Interest cost: Increases in the obligation due to the passage of time.
                Interest will accrue. The cost = the pension obilgation at the beginning of
                the period times the discount rate.
            \item Past service costs: retroactive benefits awarded to employees when plan's changed.
                IFRS: expensed immediatedly. GAAP: ammortized over the average service life of 
                employees.
            \item Changes in actuarial assumptions: Gains/losses due to changes like mortality, 
                employee turnover, retirement age, and the discount rate.
            \item Benefits paid.
        \ee
    \item Balance Sheet Effects:
        \\Funded status = fair value of plan assets - PBO
        \\ Balance sheet asset(liability) = funded status. This is good for IFRS/GAAP.
\ee
\subsection{Describe the components of a company's defined benefit pension costs.}
\be
    \item Total periodic pension cost = employer contributions - (ending funded status - beginning funded status)
        \\TPPC = current service cost + interest cost - actual return on plan assets +/- actuarial losses,gains due
        to changes in assumptions affecting PBO + prior service cost.
    \item Periodic Pension Cost Reported in P\&L, i.e. Income stmt
        \be
            \item {\bf Current service cost}. Immediately recognized in income stmt.
                CSC is the increase in PBO since the employee worke one more period.
            \item {\bf Interest cost}. Immediately recognized. Increase in PBO due to the passage of time.
            \item {\bf Expected return on plan assets}. The return on the plan assets has no effect on the PBO.
                Use expected return to compute the reported pension expense. Difference
                in expected return and actual return are in "actual gains and losses"
                IFRS: expected rate of return is assumed = the discount rate to compute
                PBO.
            \item {\bf Actuarial gains and losses} Recognized in Other Comprehensive Income.
                has 2 components. G/L due to changes
                in actuarial assumptions; difference in actual/expected return. 
                IFRS: not amortized. GAAP: amortized with corridor.
            \item {\bf Corridor Approach.} If G/L$>$10\% of max(begining PBO, plan assets),
                the excess amount should be amortized over the remaining service 
                life of employees. Time can be shorter if wanted, but consistent.
            \item {\bf Past(prior) service costs.} If pension plan is changed, the change
                reported in OCI. And amortized of the remainig life of affected employees. 
                In IFRS: changes are reported in income stmt instantly.
            \item Summary\\
                \begin{tabular}{ccc}
                        \hline
                        Component & GAAP & IFRS \\
                        \hline
                        Current service costs & Income stmt & Income stmt \\
                        Past service costs  & OCI, amortized over life & Income\\
                        Interest costs & Income & Income \\
                        Expected return & Income & Income \\
                        Actuarial G/L & Amortized part in Income, Others in OCI & OCI\\
                        \hline
                \end{tabular}
            \item {\bf Presentation.} GAAP: income stmt aggreated and presented in one line.
                IFRS: components may be presented separately. 
            \item {\bf Capitalizing Pension Costs.} 
       \ee
\ee
\subsection{Explain and calculate the effect of a defined benefit plan's assumptions on the 
defined benefit obligation and periodic pension cost.}
3 Assumptions need disclosures.
\be
    \item {\bf Discount rate:} Interest rate to compute the PV of BO and the current service cost.
    \item {\bf Rate of compensation growth:} average annual increase rate of
        employee's compensation.
    \item {\bf Expected return on plan assets:} long-term rate of return on the plan's investments.
        Only good in GAAP, b/c in IFRS, it's equal to the discount rate.
\ee

To improve reports, a company
\be
    \item Increase the discount rate to reduce PBO, pension costs, and interest cost.
    \item Decrease the compensation growth rate to reduce futher benefit payments, PBO, and
        current service cost and interest cost.
    \item Increase the expected return on plan assets.
\ee
Ultimate healthcare trend rate: constant rate of a health care inflation.
\subsection{Explain and calculate how adjusting for items of pension and other
post-employment benefits affect financial stmts and ratios.}
One need to pay attention to different assumptions when compare companies:
\be
    \item Gross vs. net pension assets/liabilities. ROA will be lower with gross
        pension A/L; leverage ratios will be higher.
    \item Differnces in assumptions used: like discount rates.
    \item IFRS, GAAP. 
    \item Difference due to classification in the income stmt. GAAP: the entire 
        periodic pension cost in P\&L, including interest are operating expense.
        IFRS: can be in various line items.
\ee
\subsection{Interpret pension plan note disclosures including cash flow related 
information.}
If the difference between cash flow and total periodic pension cost is material,
the difference can be reclassified from Operating activities to Financing activities.
\subsection{Explain issues associated with accounting for share-based compensation}
\be
    \item Forms: stock options and outright share grants.
    \item Recording: issues for stocks -- the value needs estimatings. 
    \item SHould be spread over the period for which they reward the employee.
\ee
\subsection{Explain how accounting for stock grants and stock options affects
financial stmts, and the importance of companies assumptions in valuing these grants
and options.}
IFRS and GAAP are similar.
\be
    \item {\bf Stock options.} Expense is based on the fair value of options. Spread over 
        the service time(grant date to the actual date that employees can act). Net income
        and retaining earnings will decrease, but total-equity will not change.
    \item {\bf Determining Fair Value.} If on-market, use market price.
        Otherwise, Using different models to find it.
    \item {\bf Stock grants.} Compensation expense is based on the fair value of
        the grant date. Allocated over the service period.
    \item {\bf Stock appreciation rights.} It's different from stock options. 
        It gives the employee the right to receive compensation based on the increase in the price
        of the firm's stock over some threshold. No shares are issued. No dillustion, but 
        the company needs to pay cash.
    \item {\bf Phantom stock.} Similar to stock appreciation rights. But is based on the 
        performance of hypothetical stock
\ee


\section{Reading 18: Multinational Operations}
\subsection{Distinguish among presentation currency, functional currency and local currency.}
\be
    \item Definition:
        \\-Local currency: currency of the country being referred to.
        \\-Functional currency: determined by the management. The main currency the company uses.
        \\-Presentation (reporting) currency: The currency the parent company prepares stmts
\ee
\subsection{Describe foreign currency transaction exposure, including accounting for and 
disclosures about foreign currency transaction gains and losses}
\be
    \item Foreign currency denominated transactions are measured in the presentation
        currency at the spot rate on the transaction date. Currency risk arises when the
        transaction date and payment date differ, leads to differnt spot rate.
    \item If balance sheet date occurs before the transaction is settled, recognize G/L in
        blance sheet, and unrealized G/L in income stmt. When the transaction settled,
        additional G/L may need to be recognized.
    \item Analyst Issues: G/L due to currency in income stmt may be in operating or non-operating
        income. B/c accounting std do not provide any guides. Pay attention to this.
\ee
\subsection{Analyze how changes in exchange rates affect the translated sales of the subsidiary
and parent company}
\subsection{Compare the current rate method and the temporal method, evaluate how each
affects the parent company's balance sheet and income stmt, and determine which method is 
appropriate in variuos scenarios.}
\be
    \item Methods to translante financial stmts of sub to parent reporting currency
        \\- Remeasurement: converting the local to functional currency with temporal method.
        \\- Translantion: convert functional currency to paraent's reporting currency
        using the currenty rate method.
        \\- Define appropriate translation method. See P64 in notes, new version.
        \\- More definitions. a) Current rate: the exchange rate on the balance sheet rate.
        b) Average rate: the average exchange rate over the reporting period. c). historical rate:
        the actual when the transactions occurs.
    \item Apply the current rate method process:
        \be
            \item All income stms are translate at the average rate.
            \item All balance sheet accounts are translated at the current rate 
                {\it except for common stock, at historical rate}.
            \item Dividends are at the rate that applied when they were declared.
            \item Translation G/L is reported in shareholder's equity as a part
                of the cumulative translation adjustment.
        \ee
    \item Applying the Temporal Method
        \be
            \item Monetary A/L(fixed in the amount of currency) remeasured using the current exchange rate.
            \item All other A/L are nonmonetary assets,  like inventory, fixed assets,
                intangible assets.for example, Unearned(deferred) revenue. They are 
                remeasured in hisotyrical rate. ({\it Exception: non-money A/L on the balance sheet
                at fair value are remeasured at the current rate}).
            \item Common stock, dividends paid are at historical rate.
            \item Expense related to nonmonetariy assets are remeasured based on the 
                historical rates at the time of purchase.
            \item Revenues and all other expenses are at the average rate.
            \item Remeasurement G/L is in income stmt. 
        \ee
    \item Inventory and COGS under the Temporal Method.
       \\ Numerous historical exchange rate needs to be remembered. Inventory
                are complicated. Inventory and COGS are remeasured at different rates in 
                FIFO/LIFO.
    \item Parents Company Exposure to Changing Exchange Rates
        \be
            \item In Current rate method: Exposure in the net asset position of the subsidiary.
            \item In Temporal method: net monetary A/L are exposed.
        \ee
    \item Calculating the Translation/Remeasurement G/L
        \\G/L is reported in CTA, and GTP  is used to make A=L+E.
\ee
\subsection{Calculate the translation effects and evaluate the translation of a 
subsidiary's balance sheet and income }
\be
    \item -CR method: start with income stmt. As the net income here will be used 
        for the retained earnings in balance stmt.
        \\-Temporal method: start with blance stmt.
    \item Different results from CR and Temporal methods. Why?
        \be
            \item Income before translation G/L is different, due to the different 
                rate used for items. Example: COGS and depreciation, avearge rate
                in CR method and historical rate in Temporal.
            \item Translation G/L are different. Since net assets' are exposed to
                the depreciation of Local Currenty in CR, but net MONETARY assets
                are in temporal method.
            \item Net income is different. This is due to different exchange rate. Besides,
                in CR method, translation G/L are in CTA. In Temporal, remeasurement G/L are
                in income stmt.
            \item Total assets are different b/c inventory and net fixed assets are different.
        \ee
    \item
\ee
\subsection{Analyze how the current rate method and the temporal method affect financial 
statements and ratios.}
\be
    \item Pure Balance Sheet and pure Income Stmt ratios. PURE balance sheet and pure
        income statement ratios 
    \item Mixed Balance Sheet/Income Statement ratios. CR result in small changes.
        \\ Key points to remember
        \\ --Pure balance sheet/pure income ratios will be the same.
        \\ --If foreign currency is depreciating, translated mixed ratios (with
        income stmt in up and end-of-period balance sheet item down) will be larger.
    \item Compare ratios from Temporal method and current rate method. Compare the rates
        is the key.
        \be
            \item Determine whether the foreigen currency is appreciating/depre.
            \item which rates is on numberator or denominator.
        \ee
\ee
\subsection{Analyze how alternative translation methods for subsidiaries operating
in hyperinflationary economies affect financial stmts\& ratios}
\be
    \item Hyperinflation def. In FASB, cumulative infation $>$100\% over 3-yr
        period. IASB: doesn't have definition. But 100-3-yr is a good indication.
        Nonmonetary A/L are not affected by hyperinflation.
    \item GAAP: in hyperinflation, the parent's presentation currency is the functional
        currency. -- Temporal method 
    \item IFRS: foreigen currency stmts are restated for inflation, and then tranlated
        with CR method.
        \\--Nonmonetary A/L restated with price index.
        \\--Monetary A/L doesn't change.
        \\--Shareholder's equity are restated with price index.
        \\--Retained earnings, plug figure
        \\--Income stmt: times the change in the price index from the transaction
        date.
        \\--Net purchasing power G/L recognized in income stmt
        \\--Check the examples on P82, new notes, book2
    \item Analyzing Foreign Currency Disclosure
        \be
        \item
        --Multiple foreign subs may exist. And disclosure information are limited.
        Can be found in footnotes (financial stmt) and management discussion/analysis
        of the annual report.
        \\---Possiblie Solution:Add the change in CTA into netincome.
        \\---Also, can add the unrealized G/L to net income.
        \item 
        -- Clean-surplas accounting: add G/L that are reported in shareholder's
        equity to net income stmt.
        \\-- Dirty-surplus accounting: report G/L in shareholder's equity.
        \ee
\ee
\subsection{Describe how multinational operations affect a company's effective tax rate}
\be
    \item Tax Implications of Multinational Operations
        \\--Effective tax rate: tax expense divided by pretax profit
        \\--Statutory tax rate: Provided by the tax code of the home country. 
        \\Companies needs to reconcile these two rates.
        \\Influence on the effective rate:
        \\--Changes in the mix of profits from different countries
        \\--Changes in tax rates
\ee
\subsection{Explain how changes in the components of sales affect the sustainability 
of sales growth}
\be
    \item Sales growth due to currency appreciation are not sustainable.
    \item Organinc growth: growth in sales excluding the effects of acquisitions/
        divestitures and currecy effects.
\ee
\subsection{Analyze how currency fluctuations potentially affect financial results,
given a company's countries of operation}
Major Sources of Foreign Exchange Risk
\be
    \item Can affect value of A/L
    \item Related disclosures in MD\&A. 
        \\--Helpful for Earnings change from currency change
        \\--Can do sensitivity analysis or inquire further information of hedgeing
        tools the company used.
\ee

\section{Evaluating Quality of Financial Reports}
\subsection{Demonstrate the use of a conceptual framework for assesing the quality
of a company's financial reports}
\be
    \item Financial Report Quality=earnings quality+reporting quality
        \\--reporting quality: decision useful information
        \\--earnings quality:high-level earning + sustainable earning
        \\--Cannot have low-quality reporting and high-quality earning
    \item Questions to ask:
        \\-- Standard compliant AND decision useful?
        \\-- Are the earnings of high quality?
\ee
\subsection{Explain potential problems that affect the quality of financial reports}
Two problems: Measurement and timing issues and/or Classification issues
\be
    \item Problems
        \\$∙$Measurement and Timing Issues: 
        aggressive/conservative recognition practices influence p,e,a; 
        omission/postponement of expense will increase profits,equity,assets.
        \\$∙$Classfication Issue: How an individual financial stmt element within
        a particular financial stmt. Influence particular item.
    \item Biased Accounting: Examples below.
        \be
            \item  Misstate profitability P102 NB2
                \\-Aggressive revenue recog
                \\-Lessor use of finance lease classification
                \\-Classifing non-operating reve/income as operating, and operating
                expense as non-operating
                \\-Channeling gains in net income and expense in OCI
            \item Warnings signs of misstated profitability:
                high revenue growth than peers; receivable growth$>$revenue growth;
                higher rate of returns;high proportion of revenue is received in 4th quarter;
                unexplained bosst to opearting margin; operating cash flow lower than operating income;
                inconsistency in operating vs non-operating classificaiton;aggressive
            \item Misstate A/L
                \\-Choose band inputs to change estimated value of stmt elements
                \\-Reclassificatoin from current to non-current
                \\-Over/understating allownances or reserves
                \\-Understating identifiable assets
            \item Warning signs of A/L
                \\-Inconsistent inputs for estimating A/L
                \\-Typical current A is in non-current.
                \\-Allowances and reserves differ from peers, and flucturate
                \\-high goodwill
                \\-Use of special purpose entities
                \\-Large fluctuations in deferred tax A/L
                \\-Large off-balance-sheet liability
            \item Overstate opearting cash flows
                \\-Managin activities to affect CFlow from operating
                \\-Misclassfying investing CFlow from operations.
        \ee
    \item Business Combinations -- Acquisition method accounting
        \be
            \item Give opportunities to change cf stmt:Purchase cash-generating entities to increase CFlow. Payment using
                stock can bypass the cash flow stmt.
            \item Give motivations to impact stmt.
        \ee
    \item GAAP accounting but not Enconomic reality
\ee
\subsection{Describe how to evaluate the quality of a company's financial reports}
\subsection{Evaluate the quality of a company's financial reports}
\be
    \item Steps:
        \\- Understand the company, industry, and accouting principles
        \\- Understand management, evaluate insider trades and related party transactions
        \\- Identify material areas of accounting that are vulnerable to subjectivity
        \\- Make cross-sectional and time series comparisons of stmts and ratios
        \\- Check for warning signs
        \\- multinational firms, check for shifting of profits/revenues to specific part
        of businees that the firm wants to highlight.
        \\- Use quantitative tools to evaluate the likelihood of misreporting
    \item Quantitative tools
        \be
            \item The beneish model
                \\ M-score$>$-1.78 indicates a higher-than-acceptable probability
                of earnings manipulation.
                \\ Limitatings: relies on accounting data.
            \item Altman model: Z-score to assess the probability that a firm
                will file for bankrupcy.
                \\ Limitations: a single-period static model.
        \ee
\ee
\subsection{Describe indicators of earnings quality}
\be
    \item High-quality earnings: sustainable, adequate
    \item low-quality earnings may due to
        \\-below the firm's cost of capital
        \\-not sustainable
        \\-poor reporting quality
\ee
\subsection{Describe the concept of sustainable earnings}
\be
    \item Definiation: earnings that are expected to recur.
    \item Possible gaming parts: 1. Classification items. 2. use non-GAAP metrics.
        \\-One way to gauge earnings: earnings(t+1)=$\alpha$+$\beta_1$ earnings(t)+$\varepsilon$
    \item Accruals:
        \\earnings(t+1)=$\alpha$+$\beta_1$ cash flow(t)+$\beta_2$ accruals + $\varepsilon$
        \\Acruals from normal business: non-discretionalry accruals
        \\Red flag: A company reports positive net income while negative operating
        cash flow.
    \item Other indicators: companies repeatedly meet or barely beat consensus estimates.
        External: enforcement actions.
\ee
\subsection{Explain mean reversion in earnings and how the accruals component of earnings
affects the speed of mean reversion}
Extreme earnings will revert back to mean. When earnings are largely comprised
of accruals, mean reversion will occur more.
\subsection{Evaluate the earnings quality of a company}
\be
    \item Earnings manipulations: 
        1. Revenue recognition issues; 
        2. Expense recognition issues (capitalization)
    \item Revenue recognition issues:
        \be
        \item Issues
        \\1. Channel-stuffing, bill-and-hold
        \\2. higher growth rate of receivables wrt the growth rate of revenue
        \\3. Increasing days' sales outstanding over time
        \item Steps
        \\-1. Understand the basics
        \\-2. Evaluate and question ageing receivables
        \\-3. Cash vs accruals
        \\-4. Compare financials with physical data provided by the company.
        \\-5. Evaluate revenue trends and compare with peers
        \\-6. Check for related party transactions
        \ee
    \item Expense Capitalization
        \\Checking steps
        \\-1. Understand the basics
        \\-2. Trend and comparative analysis. Stable profit margins with a buildup
        of non-current assets is bad.
        \\-3. Check for related party transactions.
\ee
\subsection{Describe indicators of cash flow quality}
\be
    \item High-quality cash flow: reported CF is high; reporting quality is high.
    \item Startup: negative OCF is OK. Mature: negative OCF is bad.
    \item Operating CF is most important. OCF that is sustainable and adequate are good.
    \item Manipulate CF via strategic decisions (timing issues)
\ee
\subsection{Evaluate the cash flow quality of a company}
Steps
\be
    \item Checking for any unusual items or items that not shown in prior yrs.
    \item Checking revenue quality.
    \item Checking for strategic provisioning.
    \item Remember: different standards of GAAP and IFRS may influence cash flow.
\ee
\subsection{Describe indicators of balance sheet quality}
\subsection{Evaluate the balance sheet quality of a company}
Completeness, unbiased measurement, clarity of presentation
\be
    \item Completeness
        \\- If off-balance-sheet liabilities exist, then need to restate the balance sheet.
        \\- Equity method can make certain ratios higher than acquisition method. If firms 
        use equity method rather than acquision method, then pay attention to it.
    \item Unbiased Measurement. Some subjectiveity:
        \\- Value of the pension liability, based on several assumptions
        \\- Value of investment in debt or equity of other companies for which
        a market vvalue is not available
        \\- Goodwill value
        \\- Inventory valuatoin
        \\- Impairment of PP\&E
    \item Clear Presentation
        \\A single-line item or items grouped together? Although standard 
        doesn't specify how much items must be presented. Clear presentation is good.
\ee
\subsection{Describe sources of information about risk}
Financial statements; Auditor's report; Notes to financial stmts; Management discussion and Analysis;
SEC form "NT"; Financial press




\section{Reading 20: Integration of Financial Statement Analysis Techniques}
\subsection{Demonstrate the use of a framework for the analysis of financial
statements, given a particular problem, question, or purpose. (e.g., valuing
equity based on comparables, critiquing a credit rating, obtaining a comprehensive
picture of financial leverage, evaluting the perspectives given in management's
discussion of financial results)}
Steps: in notebook.
\subsection{Identify financial reporting choices and biases that affect the quality
and comparability of companies' financial statements and explain how such biases may
affect financial decisions}
\be
    \item Sources of Earnings and Return on Equity: Use DuPont decomposition
        \be
            \item Use DuPont to find the performance drivers.
                $$
                ROE=\frac{NI}{EBT}\times \frac{EBT}{EBIT}\times \frac{EBIT}{Revenue}
                \times \frac{revenue}{average assets}\times \frac{average}{average equity}
                $$
            \item Consider, if the income is generated internally or externally. Remove them
                from DuPont analysis.
            \item Rm pro-rata share of investee's earnings in influential investments.
            \item In equity method, Rm the carrying value of investments in balance sheet.
        \ee
    \item Asset Base
        \\ Try to present balance sheet items in a common-size format. To get an
        overview of the changes in the compoisition of assets over time.
    \item Capital Structure
        \\ Must be able to support management's strategic objectives as well as honor
        obligations in future.
        \\Some liabilities are more burdensome than others.
    \item Capitall Allocation Decision
        \be
            \item Financial stmts should be disaggreated by segment.
                \\- Business segment: $>$ 10\%of a large company
                \\- Geographic segement
            \item Compare methods
                \\- Compare EBIT margin to capital allocations, to see if
                the company invest on most profitable segment.
                \\- Compare cash flow generated by each segment with capital allocations
                . Cash flow $\approx$ EBIT + depreciation + amortization
        \ee
\ee
\subsection{Analyze and interpret how balance sheet modifications, earnings normalization,
and cash flow statement related modifications affect a company's financial statements, financial ratios,
and overall financial condition}
\be
    \item Earnings quality and cash flow analysis
        \be
            \item Earnings quality: persistence and sustainability. Earnings closer
                to OCF are good. Check ratio of accruals to net operating assets to
                measure earnings quality. Split accruals and cash flow in earnings with balance method
                or cashflow stmt method.
            \item Accruals Ratio
                \\-Balance sheet method: 
                $$
                    Accruals = \Delta Asset - \Delta Liability - \Delta Cash=
                    NOA_{end}-NOA_{beg}
                $$
                where $NOA=Net operating asset = Asset - Liability - Cash$.
                $$
                    accrual ratio^{bs}=\frac{NOA_{end}-NOA_{beg}}{(NOA_{end}+NOA_{beg})/2}
                $$
                \\-Cash flow statement approach
                $$
                    Accruals = NI - CFO - CFI
                $$
                Higher ratio or wide fluctuated ratio are bad, indicating earnings manipulation.
                \\To compare these two measures.Eliminate cash paid for interest and taxes from OCF by adding them back. They 
                are not opearting income. -- CGO, cash generated from opeartions.
                $$
                    CGO = EBIT + non-cash charges - increase in working capital
                $$
                Compare CGO vs operating income to see if any problems.
            \item Market Value Decomposition
                \\ It's good to determine the standalone value of the parent company.
        \ee
\ee
\subsection{Evaluate the quality of a company's financial data and recommend appropriate
adjustments to improve quality and comparability with similar companies, including adjustments
for differences in accounting standards, methods, and assumptions.}
\be
    \item Off-Balance-Sheet Financing
        \\Some important items are not reported on Blance sheet. Example: Opearting Leases.
        In analysis, an operating lease should be treated as a finance lease.
        \\Methods to convert operating lease:
        \\--Euqity is OK. Since assets and liabilities are increased by the same amount.
        \\--Income statement: replace the rental expense for the operating lease with 
        depreciation expense (on the lease asset) and interest expense(on the lease liability).
\ee


\section{Reading 21: Capital Budgeting}
\be
    \item Warm-up: Basics of Capital Budgeting
    \item Categories of Capital Budgeting Projects
        \\- Replacement project to maintain business
        \\- Replacement projects for cost reduction
        \\- Expansion projects
        \\- New product or market
        \\- Mandatory
        \\- Other projects
    \item Principles of Capital Budgeting
        \be
            \item Based on cash flows: 
                \\-Sunk cost: costs that cannot be avoided.
                \\Externalities: effects that the acceptance of a project
                may have on other cash flows
            \item Cash flows are based on opportunitiy costs.
                \\-OC: cash flow that the firm might lose by undertaking
                the project.
            \item The timing of cash flow is important.
            \item Cash flows are anayzed on an after-tax basis.
            \item Financing costs are reflected in the project's required rate 
                of return.
        \ee
    \item Modified Accelerated Cos Recovery System (MACRS)
        \be
            \item Definition: A depreciation method that most US companies used
                for tax purpose. Should use it as well in capital budgeting.
            \item You will have a MACRS table to compute incremental cash flows
            \item Half-year convention: asset is is in service in the middle of the
                first year. Therefore, 3-yr asset will have 4 calendar years.
            \item Depreciable basis: purchase price + any shipping or handling and
                installation costst.
        \ee
\ee
\subsection{Calculate the yearly cash flows of expansion and replacement
capital projects and evaluate how the choice of depreciation method affects those
cash flows}
\be
    \item Classification: 1, Initial investment outlay. 2, Operating cash flow over
        the project's life. 3, terminal-year cash flow
        \\- Initial investment outlay = FCInve + NWCInv = invest in Fixed capital
        + investment in net working capital.  NWCInv = $\Delta$non-cash current assets - $\Delta$non-debt current liabilties
            = $\Delta$NWC = changes in networking capital. Cash is not operating asset. If NWCInv is positive, cashflow will be negative. Cash
        is needed to invest in NWC.
        \\-  After-tax operating cash flows: CF = (S-C-D)(1-T)+D 
        = (Sales-Cash operating expense - Depreciation expense)(1-marginal tax rate)+D
        \\- Terminal year after-tax non-operating cash flows(TNOCF). Sometimes, NWCInv can be
        reverted at the terminal year.
    \item Expansion Project Analysis: increase both the size and earnings of a business.
        Using Initial investment outlay, After-tax ocf, and TNOCF to calculate NPV and IRR. Then 
        decide if we should accept the project.
    \item Other Presentatoin Formats.
        \\-Type: table format with cash flows collected by yr; tbl formate with cf collected by type.
    \item Replacement Project Analysis. Different from Expansion
        \\-Initial outlay, old asset will be sold: Outlay = FCInv+NWCInv-Sal$_0$+T(Sal$_0$-B$_0$)
        \\-Incremental operating CF, CF from new asset - CF from old asst: 
        $\Delta$CF=($\Delta$S-$\Delta$C)(1-T)+T$\Delta$D
        \\-TNOCF = (Sal$_TNew$-Sal$_TOld$)+NWCInv-T[(Sal$_TNew$-B$_TNew$)-(Sal$_TOld$-B$_Told$)]
\ee
\subsection{Expain how inflation affects capital budgeting analysis}
\be
    \item Analyzing nominal or real CF. Nominal CF has inflation, while real CF not. CF should be
        discounted at a correct rate. (Nominal rate or real rate.)
    \item Changes in inflation affect project profitability. Changes in inflation rate
        will change the value of future CF.
    \item Inflation reduces the tax savins from depreciation. Because the depreciation
        savings is less valuable, the tax paid is more.
    \item Inflation decreases the value of payments to bondholders.
    \item Inflation may affect revenues and costs differently.
\ee
\subsection{Evaluate cpaital projects and determine the optimal capital project in situation
of 1)Mutally exclusive projects with unequal lives, using either the least common multiple
of lives approach or the equivalent annual annuity approach, and 2) capital rationing.}
\be
    \item Mutually Exculive Projects with Different Lives
        \be
            \item Least Common multiple of lives approach. 
                \\ For example. a-3yr, b-6yr. We will use Cashflows for 2*a to compare
                with b-6yr.
            \item Equivalent annual annuity approach: Use FV=0 and current PV, to calculate
                the PMT.
        \ee
    \item Capital Rationing
        \\Firms will continue to invest in positive NPV until marginal return = marginal cost.
        \\If Firm doesn't have enough funds, it needs to allocate funds to maximize NPV.
\ee
\subsection{Explain how sensitivity analysis, scenario analysis, and Monte Carlo simulation
can be used to asses the stand-alone risk of a capital project.}
Sensitivity analysis: change an input to see the changes in results.
\\Scenario analysis: A risk analysis technique that considers both the sensitivity of key output
variable to key input variables and the likely probability distributions of these variables. It studies
different possible scenarios, like {\it worst case, best case, base case.}
\subsection{Explain and calculate the discount rate, based on market risk methods, to use
in valuing a capital project}
\be
    \item CAPM: $R_{project}=R_F + \beta_{Project}[E(R_{MKT})-R_F]$
    \\ The calculated R is the appropriate discount rate. Also, it's the required return rate
    specific for one project.
    \item R=Hurdal rate
\ee
\subsection{Describe types of real options and evaluate a capital project using real
options}
\be
    \item Real options: allow managers to make future decisions that change the
        value of capital budgeting decisions made today.
    \item Types of real options:
        \\-Timing options: allow to delay aking an investment with the hope of having
        better information in the future.
        \\-Abandonment options: right to drop a project
        \\-Expansion options: right to make additional investments.
        \\-Fleibility options: 1. price-setting options: allow the company to change the price of
        of a product. 2. production-flexibility options: give some flexibilities in productions.
        \\-Fundamental options: projects that themselves are options. b/c the payoff
        depend on the price of an underlying asset.
    \item Evaluate project with real options. Approaches:
        \\1. Determine the NPV of the project w/t the option.
        \\2. Calculate the project NPV without the option and then add the 
        estimated value of the real option. 
        $$
            overall NPV = project NPV(based on DCF) - option cost + option value
        $$
        \\3. Use decision trees.
        \\4. Use option pricing models.
\ee
\subsection{Describe common capital budgeting pmistaks}
\be
    \item Failing to incorporte ecnomic response into the analysis. Example: 
        low barriers to entry will have more competitors.
    \item Misusing standardized templates.
    \item Pet projects of senior management. Projects backed by influential people
        are usually overrated.
    \item Basing investment decisions on EPS or ROE. Managers whose incentive is related
        related to ROE.
    \item Using the IRR criterion for project decision. NPV is good.
    \item Poor cash flow estimation.
    \item Misestimation of overheaded costs.
    \item Using the incorrect discount rate. Should use the rate for a specific project.
    \item Politics involved wity spending the entire capital budget. Spend entire budget
        and then ask for an increase for the next year.
    \item Failure to generate alternative investment ideas.
    \item Improper handling of sunk and opportunity costs. Shouldn't consider sunk costs
        in the evaluation of a project.
    \item 
\ee
\subsection{Calculate and interpreat accounting income and economic income in the 
context of capital budgeting.}
\be
    \item Economic income and Accounting income
        \\Economic income=cash flow + (ending market value - begin market value)
        =cash flow - economic depreciation. begining market value is PV of the remaining after-tax 
        cash flows.
        \\Accounting income=net income
    \item Difference
        \\ 1. Accounting depreciation is based on the original cost
        \\ 2. Financing costs are subtracted out to arrive at net income.
\ee
\subsection{Distinguish among the economic profic, residual income, and claims valuation models
for capital budgeting and evaluate a capital project using each}
\be
    \item {\bf Economic profit:} EP=NOPAT - \$WACC=Net operating profit after tax
        -  WACC*capital. Capital = dollar amount of investment=equity + debt
        \\Returns on all supplies of income.
        \\Company value = NPV + initial capital investment
    \item MVA: market value added, is the NPV based on economic profit
        \\NPV=MVA=$\sum_{t=1}^{\infty}\frac{EP_t}{(1+WACC)^t}$
    \item {\bf Residual income}: returns on equity
        \\Residual income = net income - equity charge
        \\$RI_t=NI_t-r_eB_{t-1}$. $r_e$: required return on equity.
        $B_{t-1}$: beginning of period book value of equity
        \\NPV =$\sum_{t=1}^{\infty}\frac{RI_t}{1+r_e}^t$
        \\Company value = NPV + initial capital = NPV + inital euqity + debt
    \item {\bf Claims valuation approach}: Valuae debt and equity cash flows separately.
        \be
            \item Cash flows to debt holders: interest and principal payments, discounted at the cost of debt
            \item Cash flows to equity holders: dividends and share repurchases. rate at the cost of equity.
                \\ CF to equity = Operating income - principle payment to debt 
                = NI + depreciation - principal payments
            \item company value = market value of debt + mv of equity
        \ee
\ee


\section{Reading 22: Captial Structure}
\subsection{Capital Structure Theory}
\subsection{Expalin the Modigliani-Miller propositions regarding capital structure,
including the effects of levergae, taxes, financial distress, agency costs, and
asymmetric information on a company's cost of equity, cost of capital, and optimal capital
structure}
\be
    \item MM Proposition I(No Taxes): The capital Structure Irrelevance Proposition
        \\Sum: MM proved that the value of a firm is unaffected by its capital structure
        under some restrictive assumptions.
        \\Assumptions:
        \\- Capital markets are perfectly competitve. No transactions costs, taxes,
        bankruptcy costs.
        \\- Investors have homogeneous expectations.
        \\- Riskless borrowing and lending: borrow at risk-free rate.
        \\- No agency costs: no conflict of interest between managers and shareholders.
        \\- Investment decisions are unaffected by financing decisions.
        \\Results: Value$_{leverge}$=Value$_{unleverage}$
    \item MM Proposition II(No Taxes): Cost of Equity and Leverage Proposition
        \\Sum: the cost of equity increases linearly as a company increases its proportion
        of debt financing.
        \\Result: 
        -Debtholders have a priority claim on assets and income, thus cost
        of debt $<$ cost of equity.
        -If the use of debt is increasing, the risk increase. Cost of equity is increasing.
        -No change in WACC
        -$r_e=r_0+\frac{D}{E}(r_0-r_d)$; $r_e$ is the cost of CAPITAL. $r_0$ is unleveraged equity.
    \item MM Proposition I(With Taxes): Value is Maximized at 100\% Debt
        \\{\bf Tax shield provided by debt:} Firms like using debt because interest
        is tax-deductible.
        \\$V_L=V_U+(t\times d)$; Value of Leverage firm = Value of unxx firm + 
        tax rate times value of debt
        \\Value will be maximized if use 100\% debt.
    \item MM Proposition II(Wtih Taxes): WACC is Minimized at 100\% Debt
        \\$r_E=r_0+\frac{D}{E}(r_0-r_D)(1-T_c)$
    \item Costs and Their Potential Effect on the Capital Structure
        \be
            \item Costs of financial distress
                \\- Costs of finanical distress and bankrupty: direct(direct fees),
                indirect(lost some investment opportunities and trust from customers, suppliers, etc)
                \\- Probability of financial distress: Higer amounts of leverage result in 
                higher probabilty of distress.
                \\ Higher expected cost of fin. distress will discourage companies from debt.
            \item Agency costs of euqity: conflicts between interest between managers and shareholders.
                \\Net agency cost of euqity has 3 components
                \\- Monitoring costs: costs to supervise management
                \\- Bonding costs: assumed by manageent to assure shareholders that
                the managers are working in the shareholder's best interest.
                \\- Residual losses
            \item Costs of asymmetric information:
                \\ Resultig from the fact that managers have more infor than owners or creditors.
                \\ Shareholders creditors looking for the signals that tell what management have
                \\ - Taking on the commitment to make fixed insterest payments via debt financing
                is good. 
                \\ - Issuing equity is usually bad.
            \item Pecking order theory: signals management sends to investors via financing choices.
                \\ Managers will make financing choices that are least likely to send signal to
                investors.
                \\Choice: Internally generated equity(retained earnings) $>$ Debt $>$ External equity
                \\ Based on POT, the capital structure is the by product of the individual financing
                decision.
        \ee
    \item Static Trade-off Theory, include fin. distress:
        balance costs of fin. distress and tax shielding from debt
        \\$V_L=V_U + (t\times d)$-PV(costs of financial distress)
    \item Implications for Managerial Decison Making
        \\-MM's Propositions iwth no taxes: 1. capital structure is irrelevant with firm value. 
        2. WACC will not change. Increase use of debt will increase the cost of euqity.
        \\-MM"s Propositions with taxes: 100\% debt is good due to tax shielding.
        \\-Static trade-off theory
\ee
\subsection{Describe target capital structure and explian why a compny's actual captial
structure may flucturate around its target.}
\be
    \item Target capital structure: optiamal capital structure
    \item May flucturate
        \\- Management may choose to exploit opportunities in a specific financing source
        \\- Market value flucturation will occur, like changes in stock and bond markets.
\ee
\subsection{Describe the role of debt ratings in captial structure policy}
     Debt ratings: cost of capital is tied to debt ratings. Thus companies will keep the minimum
     ratings of debts.
\subsection{Explain factors an analysis should consider in evaluating the effect
of capital structure policy on valuation.}
Consider:
\be
    \item Changes in the company's capital structure over time
    \item Capital structure of competitiors with similar business risk
    \item Company-specific factors(like quality of corporate governance). Better 
        corporate governence will reduce agency costs.
\ee
\subsection{Describe internationa differences in the use of financial leverage, 
factors that expalin these differences and implications of these differences for investment analysis}
Capital structure of international firms will impact a firm's capital policy.
\be
    \item Total debt: Janpan, Italy, France will have more total debt in USA, UK.
    \item Debt Maturity: Co. in North Americal will use longer maturity debt than Co in Jp.
    \item Emerging market differences: Co. in developed companies will use more and longer debt
        than those in emerging markets.
\ee
Other factors are
\be
    \item Institutional and Legal factors
        \\- Strength of legal system: strong legal system results in less debt and long maturity
        in capital structure.
        \\- Information asymmetry: strong asymmetry \ra More debt. 
        \\- Taxes
    \item Financial Markets and Banking system factors
        \\- Liquidity of capital markets: larger and more liquid markets \ra longer maturity debt
        \\- Reliance on banking system: If more relian on banking systems, companies
        will be more leveraged.
        \\- Institutional investor presence: If instituional investors play more role,the
        capital structure may be changed.
    \item Macroeconomic factors
        \\- Inflation: higher inflation\ra use less debt and short maturity.
        \\- GDP growth: higher GDP growth\ra long maturity debt.
\ee
\begin{table}[h]
\centering
\begin{tabular}{lll}
\hline
Country specific facotr                 & Use of Total Debt & Maturity of Debt \\ \hline
Institutional and Legal factors        &                   &                  \\
-Strong legal systems                    & Low               & Long             \\
-Less information asymmetry              & low               & long             \\
-Favorable tax rates on dividend         & low               &                  \\
-Common law opposed to civil law         & low               & long             \\
Financial Market factors                &                   &                  \\
-More liquid stock and bond markets      &                   & long             \\
-Greater reliance on banking system      & high              &                  \\
-Greater institutional investor presence & low               & long             \\
Macroeconomic Factors                   &                   &                  \\
-High inflation                          & low               & short            \\
-High GDP growth                         & low               & long             \\ \hline
\end{tabular}
\end{table}



\section{Dividends and Share Repurchases: Analysiss}
\subsection{Compare theories of dividend policy and explain implications of each for share
value given a describption of a corporate dividend action}
\be
    \item {\bf Dividend irrelavance}:In a perfect world with no taxes, broker fees, and infinity
        divisiable shares, dividend policy has no effect on the price of a firm's
        stock or its cost of capital. Will not affect the required return on equity capital.
    \item {\bf Bird-in-hand argument for dividend policy}: Required return on equity capital
        $r_s$ decreases as the dividend payout increases.
    \item Tax aversion:Historically, higher tax on dividends than capital gain. Therefore, investors
        prefer to not receive dividends. While companies will make dividend payments necessary
        to avoid tax.
    \item Conclusion: 
\ee
\subsection{Describe types of information (signals) that dividend initiations, increases, 
decreases, and omissions may convey}
\be
    \item Dividend initiation: ambiuous. Positive: company is sharing its wealth; Negative: 
        a company has a lack of profitable reinvestment opportunities.
    \item Unexpected dividend increase: Good.  Companies future is good.
    \item Unexpected dividend decreases or omissions: negative signals.
\ee 
\subsection{Expain how clientele effects and agency issues may affect a company's payout
policy.}
\be
    \item Clientele effect: dividend preferences of different groups are different.
        \\- Tax consideration. High tax-bracket investors(individuals) prefer low dividend
        payouts. Low-tax-bracket investors prefer high payouts.
        \\- Requirements of institutional investors: some of institu. investors
        will only invest companies that have dividend yield above some threshold.
        \\- Invdividual investor preferences.
    \item Agency issues
        \\- Between shareholders and managers: agency cost is due to difference interest
        between managers and stockholder. Manager may overinvest. Therefore, one way to reduce
        the agency cost is to increase the payout of free cash flow as dividends.
        \\- Between shareholders and bondholders: If high-risk debt outstanding, shareholders may pay
        themselves a large dividend.
\ee
\subsection{Explain factors that affect dividend policy.}
6 factors:
\be
    \item Investment opportunities. If the firm has many opportunities and need
        to react quickly, dividend payout would be low.
    \item Expected volatility of future earnings. If earnings are volatile, firms
        are more cautious in {\it changing} dividend payout.
    \item Financial flexibility.
    \item Tax considerations.
        \\- Dividend preference may different if capital gains and dividends are taxed
        at different rate.
        \\- Taxes on dividends are paid when dividend is received. Capital gain taxes
        are paid when shares are sold.
        \\- Tax-exempt institutions will be indifferent between dividends/capital gains.
    \item Flotation costs: 3\% to 7\% fee will be applied for new stocks, while retained
        earnings have no such fee. High flotation costs\ra lower dividend payout.
    \item Contractual and legal restrictions.
        \\- Impariment of capital rule: in some countries, dividend cannot be larger
        than retained earnings.
        \\- Debt covenants: many covnants require a firm to meet/exceed a certain target
        for liquidity rations and coverage ratios before they can pay a dividend.
\ee
\subsection{Calculate and interpret the effective tax rate on a given currency unit of
corporate earnings under double taxation, dividend imputation, and split-rate tax systems.}
\be
    \item Double-taxation system
        \\ Earnings are taxed at corporate regardless of whether dividend. Dividend are taxed again
        for shareholders.
        \\ Effective rate = Corporate tax rate + (1-Corporate tax rate)(individual tax rate)(payout ratio)
    \item Imputation tax system
        \\ All taxes are effectively paid at the shareholder rate.
    \item Split-rate
        \\ Earnings that are distributed as dividends are taxed at a lower rate. Individual: dividends
        are taxed as income.
\ee
\subsection{Compare stable dividend, constant dividend payout ratio, and residual dividend
payout policies, and calculate the diviend under each policy.}
\be
    \item Stable Dividend Policy:
    \item Target Payout Ratio Adjustment Model
        \\ Expected dividend = previous dividend + (exp.d increase in EPS)(target payout ratio)(adjustment factor).
        Where adjustment factor = 1/number of years of which the adjustment will take place.
    \item Constant Dividend Payout Ratio Policy
        \\ Seldom used.
    \item Residual Dividend Model:
        \\ (Dividends) is based on (Earnings) less (funds retained to for the equity portion.) 
        \\ Advantages: 1. Simple to use; 2. Pursue investment opportunities without
        being constrained by dividend considerations.
        \\ Disadvantages: Dividends will be unstable.
    \item Long-term residual dividend model.
\ee
\subsection{Explain the choice between paying cash dividends and repurchasing shares}
Possible reasons to repurchase shares
\be
    \item Potential tax advantage
    \item Share price suport/signaling: company to show that they are confident with
        their stock.
    \item Added flexibility: repurchase can be a supplement to the dividend. Repurchase
        doesn't need long-term commitment.
    \item Offsetting dilution from employee stock options. 
    \item Increase financial leverage. If funded by new debt, share repurchase will 
        increase leverage. Besides, share repurchase may increase EPS, if the cost
        of fund $<$ EPS. 
\ee
\subsection{Describe broad trends in corporate dividend policies}
\be
    \item A lower proportion of US companies pay dividends compared to European ones.
    \item In developed markets, the proportion of companies paying cash dividends is down.
    \item The percentage making stock repurchases has been upwards.
\ee
\subsection{Calculate and interpret dividend coverage ratios based on 1) net income
and 2) free cash flow.}
\subsection{Identify characteristics of companies that may not be able to sustain
their cash dividend.}
{\bf Dividend safety:} the metric used to evaluate the probability of dividends
        continuing at the current rate for a company.
        \\-Useful ratios: Dividend payout ratio(dividens/net income) and 
        dividend coverage ratio(net income/dividends)
        \\Free cash flow to equity(FCEE): cash flow available for distribution
        to stockholders after working capital and fixed capital needs are accounted for.
        \\FCFE coverage ratio = FCFE/(dividends + share repurchase)


\section{Reading 24: Corporate Performance, Governance, and Business Ethics}
\subsection{Compare interest of key stakeholders gropus and explain the purpose of
a stakeholder impact analysis}
Stakeholders:
\be
    \item Definition: Groups with an interest or claim in a company.
    \item Key internal stakeholders
        \\-Stockholders
        \\-Employees
        \\-Managers
        \\-Members of the broad of directors
    \item Key expernal stakeholders
        \\-Customers
        \\-Suppliers
        \\-Creditors
        \\-Uninons
        \\-Governments
        \\-Local communities
        \\-General public
\ee

Reconciling interests and the stakeholder analysis
\be
    \item Not all stakeholders are so interested in the profitability. e.g. Customers
        don't want to overpay. 
    \item Stakeholder Impact Analysis (SIA): Force the company to identify
        which stake groups are critical to the company.
    \item Stockholders are special stakeholders. Return on invested capital(ROIC)
        and growth in profits\ra Measure if the company satisfy shareholders
\ee

\subsection{Discuss problems that can arise in principal-agent relationships and
mechanism that may mitigate such problems.}
The Principal-Agent Relationship
\be
    \item Definition: PRA arises when one group delegats decision to another group.
    Problems: Information asymmetry. Agent doesn't tell everything to the principal.
    \item Example Problems
        \\ CEOs manipulate the board of directors to extract excessive compensation packages.
    \item Controlling PAR problems
        \\ Guide the behavior of agents
        \\ Reduce the asymmetry of information
        \\ Remove agents who misbehave and violate ethical principles.
    \item Ethics and Strategy, examples of unethical behavior
        \\Self-dealing
        \\Information manipulation
        \\Anticompetitive behavior
        \\Substandard working conditions
        \\Environmental degradation
        \\Corruption
\ee
\subsection{Discuss roots of unethical behavior and how managers might ensure that ethical issues
are considered in business decision making}
Roots of Unethical Behavior
\be
    \item Personal ethics are flawed
    \item Failure to realize
    \item Culture to focus on profit/growth
    \item Flawed business culture where top management sets unrealistic goals
    \item Unethical leadership
\ee
\subsection{Compare the Friedman doctrine, Utilitarianism, Kantian Ethics,
and Rights and Justicee Theories as approches to ethical decision making}
\be
    \item Friedman Doctrine: The only social responsibility is increase profits 
        ``within the rules of the game".
    \item Utilitarianism: seeks to produce the highest good for the largest number
        of people.
    \item Kantian Ethics: People are more than an economic input and desreve dignity
        and respect.
    \item Rights theories: all individuals have fundamental rights and privileges, and the
        utilitarianism's greatest good doesn't trmp these fundamental rights.
    \item Justice theories: justice is met if all participants would agree the rules
        are fair if the results is decided under 'veil of ignorance'.
\ee


\section{Reading 25: Corporate Governance}
\subsection{Describe objectives and core attributes of an effective corporate 
governance system and evaluate whether a company's corporate goverence has those
attributes}
\be
    \item Objectives
        \\-Eliminate or reduce confilicts of interest
        \\-Use the company's assets in a good way
    \item Properties of effective corporate governance system
\ee
\subsection{Compare major business forms and describe the confilicts of interests
associated with each}
\be
    \item Sole proprietorships: Business owned and operated by a single individual
        No distinction between the business and its owner, unlimited liability.
        \\Conflict of interest: suppliers, creditors.
    \item Partnerships
        \\Def: two or more owners/managers, but are otherwise similar to a sole proprietorship.
        Unlimited liability.
        \\Typical: Law firms, real estate firms, advertising agncies
        \\CoI: crditors and suppliers; partners conflicts are addressed in contracts.
    \item Corporations
        \\Def: Distinct legal entities that have rights similar to those of an individual
        person.
        \\CoI: owners vs management
\ee
\subsection{Explain confilicts that arise in agency relationships, including
manager-shareholder conflicts and director-shareholder conflicts}
Principal-agent problem includes:
\be
    \item Managers and shareholders, examples
        \\Using funds to expand the size of the firm
        \\Get high salaries and perquisites
        \\Investing in risky ventures
        \\Not taking enough risk: Some risk-averse managers will be conservatie
        to keep their job, rather than do a better job for shareholders 
    \item Directors and shareholders
        \\Lack of independene
        \\Board members have personal relationships with management
        \\Board members have consulting or other business agreement with the firm
        \\Interlinked boards
        \\Directors are overcompensated.
\ee
\subsection{Describe the responsibilities of the board of directors and explain quaifications
and core competencies that an investment analyst should look for in the board of directors}
\subsection{Explain effective corporate governance practice as it relates to
the board of directors and evaluate strengths and weaknesses of a company's corporate governance
practice}
The Board of Directors -- Standards related to the effectiveness
\\3/4 of board members should be indepedent.
\\Whether the board has an independent chairman: separate is good.
\\Qualifications of directors: should bring skills and experimence: for
board members to have the requisite industry, strategic planning, and risk management
knowledge, not serve on more than two or three boards.
\\How the board is elected: Annual elections are good
\\Board self-assessment practices
\\Frequency of separate sessions for independent directors: independent members
should meet at least annually, prefer quarterly.
\\Audit committee and audit oversight: audit comittee only has independent directors,
has expertise in financial/accounting, has full access to the cooperation management, and meets
with auditors at least annually.
\\Nominating committee: only have indepdent directors.
\\Compensation committee and the compensation awarded to management: should focus on 
long term goals. Shouldn't use the salary at other companies as a reference point. 
Good: have salaries as a small percentage of compensation, with bounus, options
and restricted stock for good performance.
\\Use of independent or expert legal counsel. Common practice: advise the board
of directors, but this is bad. Best: use independent, outside counsel whenever
legal counsel is required.
\\Statement of governance policies
\\Disclosure and transparency. Best: more disclosure is better. Should provide info
about organization structure, corporate strategy, insider transactions, compensation policies
and changes to governance structure.
\\Insider or related-party transations. Best: Any related-party transactions should be
proved by the board of directors.
\\Resonsiveness to shareholder proxy votes.  Should consider the shareholder's opinions.

\subsection{Describe elements of a company's statement of corporate governance
policies that investment analysts should assess.}
\be
    \item Code of ethics
    \item Directors oversight, monitoring, and review responsibilities
    \item Management's responsibility to the board
    \item Reports of directors' oversight and review of management
    \item Board self assesments
    \item Management performance assessments
    \item Director training
\ee
\subsection{Describe environmental, social, and governance risk exposures}
Environmental, Social, and Governance Factors
ESG factors
\be
    \item Environmental risk
    \item Social risk
    \item Governance risk
\ee
Organized as follows
\be
    \item Legislative and Regulatory risk: new laws may be bad
    \item Legal risk: potential results for lawsuits may be bad
    \item Reputational risk
    \item Opearting risk: due to impact of ESG factors
    \item Financial Risk:
\ee
\subsection{Explain the valueation implications of corporate governance}
Strong/effective corporate governance system is good.
\\Weak or ineffective corporate governance system is bad
\be
    \item Financial disclosure risk
    \item Asset risk
    \item Liability risk
    \item Strategic policy risk
\ee


\section{Mergers and Acquisitions}
\subsection{Classify merger and acquisition activities on forms of integration
and relatedness of business activities}
Forms of Integration
\be
    \item Statutory merger: acquiring compan acquires all of the target's A and L.
    \item Subsidiary merger: the target company will be a sub of the purchaser.
    \item Consolidation: both companies will combine into a new company.
\ee
Types of Mergers
\be
    \item Horizontal merger: businesses in teh same or similar industries
    \item Vertical merger: the acquiring company seeks to move up/down the product
        supply chain.
    \item Conglomerate merger: two companies operate in completedly separate
        industries.
\ee
\subsection{Explain common motivations behind MA activity.}
\be
    \item Synergies:combined company will be worth more than the two companies.
        Horizontal combination will reduce cost.
    \item Achieving more rapid growth. 
    \item Increased market power.
    \item Gaining access to unique capabilities.
    \item Diversification: diversify the firm's cash flow. Merger are not likely to increse
        value purely for diversification reasons. B/c merger process involves a lot
        of costs.
    \item Bootstrapping EPS. 
    \item Personal benefits for managers. Large size of company\ra high salary for managers.
    \item Tax benefits
    \item Unlock hidden value.
    \item Achieving international business goals, some specific driving factors
        \\-Taking advantage of market inefficiencies
        \\-Working arond disadvantageous government poliies
        \\-Use technoloy in new markets
        \\-Product differentiation
        \\-Provide support to existing multinational clients
\ee
\subsection{Explain bootstrapping of earnings per share and calculate a company's
postmerger EPs}
when a high P/E company purchases a low P/E firm in a stock transaction. When we calculate
EPS for the new company, total earnings is equal to the sum, however, the total shares
outstanding is less than the sum of the combined firms. EPS will increase.
\subsection{Explain, based on industry life cycles, the relation between merger motivations
and types of mergers.}
\begin{tabular}{llll}
\hline
Industry life cycle stage & Industry Characteristics                                                                                                       & Merger Motivations                                                                                                         & \multicolumn{1}{l|}{Common Types of Mergers}                                          \\ \hline
\hline
Pioneer/development       & \begin{tabular}[c]{@{}l@{}}1. Unsure of product acceptance\\ 2. Large capital requirements and low profit margins\end{tabular} & \begin{tabular}[c]{@{}l@{}}1. Gain access to capital from more mature businesses\\ 2. Share management talent\end{tabular} & \begin{tabular}[c]{@{}l@{}}1. Conglomerate\\ 2. Horizontal\end{tabular}               \\
\hline
Rapid Growth              & \begin{tabular}[c]{@{}l@{}}1. High profit margins\\ 2. Accelarating sales and earnings\\ 3. Competition still low\end{tabular} & \begin{tabular}[c]{@{}l@{}}1. Gain access to capital\\ 2. Expand capacity to grow\end{tabular}                             & \begin{tabular}[c]{@{}l@{}}1. Conglomerate\\ 2. Horizontal\end{tabular}               \\
\hline
Mature growth             & \begin{tabular}[c]{@{}l@{}}1. Lots of new competition\\ 2. Still opportunities for above average growth\end{tabular}           & \begin{tabular}[c]{@{}l@{}}1. Increase operational efficiencies\\ 2. Economies of scale/synergies\end{tabular}             & \begin{tabular}[c]{@{}l@{}}1. Horizontal\\ 2. Vertical\end{tabular}                   \\
\hline
Stabilization             & \begin{tabular}[c]{@{}l@{}}1. Competition has reduced growth potential\\ 2. Capacity constraints\end{tabular}                  & \begin{tabular}[c]{@{}l@{}}1. Economies of scale/reduce costs\\ 2. Improve management\end{tabular}                         & Horizontal                                                                            \\
\hline
Decline                   & \begin{tabular}[c]{@{}l@{}}1. Consumer tastes have shifted\\ 2. Overcapacity/shrinking profit margins\end{tabular}             & \begin{tabular}[c]{@{}l@{}}1. Survival\\ 2. Operational efficiencies\\ 3. Acquire new growth opportunities\end{tabular}    & \begin{tabular}[c]{@{}l@{}}1. Horizontal\\ 2. Vertical\\ 3. Conglomerate\end{tabular} \\
\hline
\end{tabular}
\begin{figure}[ht]
    \center
    \includegraphics[width=0.8\textwidth]{figs/merger}
\end{figure}
\subsection{Contrast merger transaction characteristics by form of acquisition,
method of payment, and attitude of target management}
Forms of Acquisition
\be
    \item Stock purchase: usually entire company.
        \\-Payment: directly to target company shareholders in exchange
        for their shares.
        \\-Approval: Majority shareholder approval required.
        \\-Corporate taxes: None.
        \\-Shareholder taxes: shareholders pay capital gains tax.
        \\-Liabilities Acquirer assumes liabilities of target.
    \item Asset purchase: usually part of a company
        \\-Payment: Made directly to target company.
        \\-Approval: No shareholder approval needed unless asset sale is substantial.
        \\-Corporate taxes: Target company pay capital gain tax.
        \\-Shareholder taxes: None.
        \\-Liabilities: Acquirer usually avoids assumptions of target's liabilities.
\ee
Method of Payment
\be
    \item Securities offering: pay with shares.
    \item Cash offers: pay with cash.
    \item Factors to choose securities/cashes
        \\-Distribution between risk and reward for the acquirer and target
        shareholders.
        \\-Relative valuations of companies involved. Stock offering is kind of
        a signal that acquirer's share may be overvalued.
        \\-Changes in capital structure.
\ee
Attitutde of Target management
\be
    \item Friendly merger offers process
        \\-both companies will negotiate, and draft a merger agreement, and 
        finally release all the information to the public.
    \item Hostile merger offers
        \\-Bear hug: Acquirer submits a merger proposal directly to the target's
        board of directors.
        \\-If bear hug fails, appeal directly to shareholders
        \\---Tender offer: buy the shares directly from shareholders
        \\---Proxy battle: have shareholders approve a new "acquirer approved" board
        of directors."
\ee
\subsection{Distinguish among pre-offer and post-offer takeover defense mechanims}
Pre-offer and Post-offer. Pre-offer defense is better, b/c faces less scrutiny in court.
\\Pre-Offer Defense Mechanisms
\be
    \item Poison pill: give shareholder's right to buy shares at extremely
        attractive price
        \\- flip-in pill: target company's shareholders can buy target's share at a discount
        \\- flip-over pill: target shareholders can buy acquirer's shares at a discount
    \item Poison put: give bondholder's the option to get repayment immediately
    \item Restrictive takeover laws: some states in US will protect company 
        against a hostile takeover. Ohio/Pennsylvania
    \item Staggered board: Board of directors are split into 3 groups, each is 
        elected for a 3-year term. Prevent bidder controlling the board.
    \item Supermajority voting provision for mergers: require more support for 
        merger, like 66.7\%, 75\% or 80\%.
    \item Fair price amendment: restricts a merger offer unless a fair price is offered.
    \item Golden parachutes: give the managers a lot of money if they leave the company 
        after merger.
\ee
Post-Offer Defense Mechanisms
\be
    \item Just say no: target can tell shareholders merger is bad.
    \item Litigation: file a lawsuit against the acquirer that takes long time
        and a lot of money.
    \item Greenmail: payoff to the acquirer to terminate the merger.
    \item Share repurchases: target company submit a tender offer for its own shares.
        Acquirer therefore has to increase the bid price.
    \item Leveraged recapitalization: target assumes a lot of debt to repurchase
        shares. The capital structure will be unattractive.
    \item Crown jewel defense: target sell a essential subsidiary or major asset
        to a neutral theird party.
    \item Pac-Man defense: target can acuquire the aquirer.
    \item White knight defense: a white knight is a friendly third party. The friendly
        third party will start a bidding war against the acquirer.
    \item White squire defense: seek a third party to buy a minority stake
        of the company. However, the stake is big enough to block the merger.
\ee
\subsection{Calculate and interpret the Herfindahl-Hirschman Index, and evaluate the likelihood
of an antitrust challenge for a given business combination.}
Calculation:
$$
HHI=\sum_{i=1}^n(MS_i\times 100)^2
$$
where $MS_i$ = market share of firm i.
\\Steps to challenge antitrust
\be
    \item Whether the post-merger HHI$>$1000?
    \item If post-merger HHI between 1000 and 1800\ra Moderately concentrated category
        \ra check the changes of HHI. If $>$100, bad. possible antritrust.
    \item If post-merger HHI$>$1800\ra Highly concentrated\ra check the change
        of HHI. If $>$50, bad. Antitrust almost certain.
\ee

-----------------------------------------------
\\Valuing a target company
\be
    \item Discounted cash flow analysis
    \item Comparable company analysis
    \item Comparable transaction analysis
\ee
\subsection{Calculate free cash flows for a target company, and estimate the
company's intrinsic value based on discount}
Dscounted Cash Flow(DCF) analysis: 1) Calculate future free cash flow.  
2) Calculated discounted cash.
\be
    \item Determine which free cash flow model to use for the analysis.
    \item Develop pro forma financial estimates.
    \item Calculate free cash flows using the pro forma data, starting with net income:
        \\
            \begin{tabular}{ll}
              & Net Income                                        \\
            + & Net interest after tax                            \\
            \hline
            = & Unlevered net income                              \\
        $\pm$ & Change in deferred taxes                          \\
            \hline
            = & Net operating profit less adjusted taxes (NOPLAT) \\
            + & Net non cash charges                              \\
            - & Change in net working capital                     \\
            - & Capital expenditures (capex)                      \\
            \hline
            = & Free cash flow                                   
            \end{tabular}
    \item Discounted free cash flows back to the present at the appropriate 
        discount rate. Usually use WACC, or WACC$_{adjusted}$ which includes the risk
        from the merger.
    \item Determine the terminal value and discount it back to the present.
        \\Method 1: assumes the company grows at a constant rate. 
        $$
            terminal value = \frac{FCF_T (1+g)}{WACC_{adjusted}}-g
        $$
        Method 2: $Terminal value_T=FCF_T\times (Projected price/FCF)$
    \item Add the discounted FCF values for the first state and ther terminal 
        value to determine the value of the target firm.
        Check examples on Page 294.
\ee
\subsection{Estimat the value of a target company using comparable company
and comparable transaction analyses}
Comparable company analysis
\be
    \item Identify the set of comparable firms: same industry with similar size and capital structure.
    \item Calculate various relative value measures basaed on the current market price
        of companies in the sample with some useful quantities: 
        \\-Enterprise value(EV)=market value of debt and equity - the value of cash and investments.
        \\-Price to earnings P/E
        \\-Price to book P/B
        \\-Price to sales P/S
    \item Calculate descriptive statistics for the relatie value metrics and apply
        those measures to the target firm.
    \item Estimate a takeover premium:$TP=\frac{DP-SP}{SP}$, where TP: takeover premium,
        DP: deal price per share, SP: target company's stock price. The takeover
        premium is estimated by looking at recent takeovers on similar companies.
    \item Calculate the estimated takeover price for the target as the sum of
        estimated stock value based on comparable and the takeover premium.
\ee
Comparable transaction analysis: using recent takeover transactions of similar
companies to estimate the takeover price.
\be
    \item Identify a set of recent takeover transactions. Should be firms
        in the same industray and have a similar capital structure.
    \item Calculate various relative value measures baesd on \textbf{completed deal prices} for
        companies in the sample. 
    \item Calculate descriptive statistics for the relative value metrics and apply
        those measures to the target firm.
\ee
\subsection{Compare the discoutned cash flow, comparable company, and comparable transaction analyses 
for valuing a target company, including the advantages and disadvantages of each.}
Discounted cash flow analysis: based on a forecash of the target firm's CF
\be
    \item Advantages
        \\-relatively eash to model
        \\-based on forecasts of conditions in the future
        \\-the model is easy to customize
    \item Disadvantages
        \\-Hard to apply when free cash flows are negative
        \\-estimates of CF and earnings have error.
        \\-Discount rate changes over tive
        \\-Estimation error
\ee
Comparable company analysis: use market data from similar firms AND a takeover 
premium
\be
    \item Advantages
        \\-Data for comparable companies is easy to access
        \\-Assumptions that similar assets have similar value is good.
        \\-Estimates of value are from the market.
    \item Disadvantages
        \\-Assumes that the market valuation of the comparable companies is accurate
        \\-Just get a fair stock price. Takeover premium must be determined.
        \\-Difficult to incorporate merger synergies or changing capital structures
        in analysis.
        \\-Historical data for premiume estimate may not be timely.
\ee
Comparable transaction analysis: uses data from completed M\&A deals
\be
    \item Advantages
        \\-No need to estimate takeover premiums
        \\-Estimates of value from actuall M\&A deals
        \\-Reduce the risk that shareholders sue that the managers/BoD for 
        mispricing the deal. 
    \item Disadvantages
        \\-Assumes that M\&A transactions are valued accurately.
        \\-Not enough transactions to use.
        \\-Difficult to incorporate merger synergies or changin capital structure.
\ee
\subsection{Evaluate a takeover bid, and calculate the estimated post-acquisition
value of an acquirer and the gains accrued to the target shareholders versus the
acquirer shareholders.}
\be
    \item Post-Merger Value of an Acquirer: $V_{AT}=V_{A}+V_T+S-C$, where 
        $V_{AT}$ is the post-merger value of the combined company, $V_A$, $V_T$ are
        acquirer and target, $S$ is synergies, and $C$ is the cash paid to target
        shareholders.
    \item Gain Accrued to the Target: $Gain_T=TP$=Takeover premium=$P_T-V_T$
    \item Cash payment VS Stock payment
        \\-Cash payment: price is just the cash
        \\-Stock payment: $P_T=N\times P_{AT}$=number of new shares$\times$price 
        per share of combined firm after the merger announcement.
\ee
Check the examples in notes. P306-Vol2
\subsection{Explain how price and payment method affect the distribution of risks and benefits 
in M\&A transactions}
\be
    \item Effect of price: acquirer hopes buy low, target hopes to sell high.
    \item Effect of Pyament Method
        \\-Cash offer: acquirers assumes the risk and the potential reward.
        \\-Stock offer: some of the risk and potential rewards frmo the
        merger shift to the target.
\ee
\subsection{Describe characteristics of M\&A transactions that creat value}
\be
    \item Short term: target's stock price gains \~30\%, and acquirer's lose 1\%~3\%
        \\-Winner's curse: the firm who wins will overpay the most
        \\-Managerial hubris: managers overestimate the synergies and expected benefits.
    \item Long term: Acquirers tend to underperform their peers
    \item mergers enhance value for the acquirer
        \\-Strong buyer: acquirers that have strong performance in the prior 3 yrs.
        \\-Low premium
        \\-Few bidders
        \\-Favorable market reaction
\ee
\subsection{Distinguish among equity carve-outs, spin-offs, split-offs and liquidation}
    Diverstitures: a company selling, liquidating, spinning off a division  or subsidiary.
\be
    \item Equity carve-outs: create a new company. Shares are issued in public. 
    \item Spin-offs: create a new company. Shares are not issued to public but distributed
        to shareholders of parent company.
    \item Liquidations: break up the firm, and sell assets.
\ee
\subsection{Explain common reasons for restructuring}
\be
    \item Division no longer fits into management's long-term strategy.
    \item Lack of profitability
    \item Individual parts are worth more than the whole.
    \item Infusion of cash. Parents need money.
\ee

\section{Reading 27: Equity Valuation: Applications and Process}
\subsection{Define valuation and intrinsic value and explain sources of
perceived mispricing}
Valuation assets, steps:
\be
    \item Understnad the business
    \item Forecast company performance
    \item Select the appropriate valuation model
    \item Convert the forecasts into a valuation.
    \item Apply the valuation conclusions
\ee
Intrinsic value and actual intrinsic
$$
    IV_{analyst}-price=(IV_{actual}-price)+(IV_{analyst}-IV_{actual})
$$
\subsection{Explain the going concern assumption and contrast a going concern value
to liquid value.}
Going concern assumption: a company will continue to operate.
\\Liquidatio value: company will die.
\subsection{Describe definitions of value and justify which definition of value
is most relevant to public company valuation.}
\be
    \item Intrinsic value: used for valuing public equities
    \item Fair market value: the price that a hypothetical willing, informed, and able seller
        would trade an seet to a willing, informed, and able buyer.
    \item Investment value: the value of a stock to a particular buyer. Some buyer's 
        may have specific values for specific stocks.
\ee
Investment: intrinsic value is best.
Acquisition: investment value is good.
\subsection{Describe applications of equity valuation}
Valuation: the process of estimating the value of an asset.
\\Methods: 1) Modeling 2)Comparing with similar assets
\\Uses:
\be
    \item Stock selection
    \item Reading the market
    \item Projecting the value of corporate actions
    \item Fairness opinions
    \item Planning and counsulting
    \item Communication with analysits and investors
    \item Valuation of private business
    \item Portfolio Management
        \\-Planning: plan investment strategy, select portfolios
        \\-Executing the investment plan
\ee
\subsection{Describe questions that should be addressed in conducting an industry
and competitive analysis}
Five elements of industry structure(Porter's five forces)
\be
    \item Threat of new entrants in the industry
    \item Threat of substitutes
    \item Bargaining power of buyers
    \item Bargaining power of suppliers
    \item Rivalry among existing competitiors
\ee

Three generic strategies to compete and generate profits
\be
    \item Cost leadership: being the lowest-cost producer of the good
    \item Product differentiation
    \item Focus: 
\ee

Several problems may encountered
\be
    \item Accelerating or premature recognition of income
    \item Reclassifying gains and nonoperating income
    \item Expense recognition and losses
    \item Amortization, depreciation, and discount rates
    \item Off-balance sheet issues: SPE, leases
\ee
\subsection{Contrast absolute and relative valuatin models and describe examples 
of each type of model}
\be
    \item Absolute valuation models: determine an asset's intrinsic value
        \\-PV of all cash flows
        \\-Dividend ciscount models
        \\-free cash flow
        \\-residual income
        \\-Asset-based models: estimate a firm's value as the sum of the mkt value
        of the assets it owns.
    \item Relative valuation models: Financial factor should be similar for similar
        companies, like P/E. P/E higher than others: overvalued.
\ee
\subsection{Describe sum-of-the-parts valuation and conglomerate discounts}
\be
    \item Sum-of-the-parts value: company operates mutiple divisions. Analyst value
        individual parts and add them.
    \item Conglomerate discount: investors apply a markdown to the value of a company that
        operates in many unrelated divisions.
        \\Reason
        \\-Internal capital inefficiency
        \\-Endogenous(internal) factors
        \\-Research measurement errors: some hypothesize that this discount do not
        exist.
\ee
\subsection{Explain broad criteria for choosing an appropriate approach for
valuing a given company}
Things to consider
\be
    \item Fits the characteristics of the company
    \item appropriate based on the quality and availability of input data
    \item suitable for the given purpose of the analysis
\ee


\section{Reading 28: Return Concepts}
\subsection{Distinguish among realized holding period return, expected holding period return, required return
, return from convergence of price to intrinsic value, discount rate and internal rate
of return.}
\be
    \item Hoding Period Return: $r=\frac{P_1-P_0+CF_1}{P_0}=\frac{P_1+CF_1}{P_0}-1$
    If the CF is received before the end of the period, CF$_1$=the cash flow received
    and the interest earned.
    \\Holding period return is annualized.
    \item Realized and Expected Holding Period Return
        \\Realized return: historical return based on past prices/Cf
        \\Expected return: forecast
    \item Required return: minimum return an investor requires given the asset's risk. 
        \\Also called Opportunity cost.
        \\ Expected return greater(less) than required return\ra Undervalued(Overvalued)
    \item Price Convergence: Expected return = required return + $\frac{V_0-P_0}{P_0}$
    \item Discount Rate: 
    \item Internal Rate of Return: Rate that makes PV of CF = The current price of securities
        \\In efficient market, IRR=required return
\ee
\subsection{Calculate and interpret an equity risk premium using historical and forward-looking
estimation approaches}
Background:
\be
    \item Equity risk premium: The return in excess of risk-free rate = required return on equity
    index - risk-free rate
    \item Required return for individual stocks = rfr+$\beta_j\times$ (equity risk premium)
\ee
Estimates of The Equity Risk Premium:
\be
    \item Historical Estimates: use mean market return and mean rfr
        \\-Strength: Objectivity, simplicity, unbiase`d
        \\-Weakeness: i) Assume mean/variable constant over time. ii) Biased by survivorship
    \item Forward-Looking Estimates: use current informatin and expectations on
        economic/financial variables.
        \\-Strength: doesn't rely on stationary assumption; less influenced by
        survivorship.
        \\-Three model: Gordon growth, supply-side, estimate from survays
\ee

Three main
\be
    \item Gordon Growth Model=constant growth model
        \\GGM equity risk premium = (1-year forecasted dividend yield on mkt index)
        + (consensus long-term earnings growth rate) - (long-term government bond yield))
        =$D_1/P + \hat{g}-r_{LT,0}$
        \\-Weakness:i) this estimation will change over time. ii)Assume stable growth rate
    \item Supply-Side Estimates(Marcoeconomic Models): based on relations between
        macroeconomic variabls and financial variables.
        \\-Strength: proven models and current information
        \\-Weakness: only good for developed countries
        \\-Model: ERP=$[1+\hat{i}]\times[1+\hat{rEg}\times[1+\hat{PEg}]]-1+\hat{Y}
        -\hat{RF}$
        \\where $\hat{i}$=expectaion inflation, $\hat{rEg}$=expected real growth in EPS,
        $\hat{PEg}$=expected changes in the P/E ratio, $\hat{Y}$=the expected yield on the index,
        $\hat{RF}$
    \item Survey Estimates: easy to obtain, but may has a wide disparity.
\ee
\subsection{Estimate the required return on an equity investment using the capital asset
pricing model, the Fama-French model, the Pator-Stambaugh model, macro-economic multifactor models,
and the build-up method.}
\be
    \item Capital Asset Pricing Model
        \\Required return on stock j = risk-free rate + equity risk premium $\times$ beta of j
    \item Multifactor models
        \\Required return = RF + risk premium$_1$ + risk premium$_2$ + ... + risk premium$_n$
        \\risk premium$_i$=factor sensitivity$\times$ factor risk premium
    \item Fama-French Model: accounts for the higher returns associated with small-cap stocks
        \\required return of stock j=RF + $\beta_{mkt,j}\times(R_{mkt}-RF)+
        \beta_{SMB,j}\times(R_{small}-R_{big})+\beta_{HML,j}\times(R_{HBM}-R_{LBM})$
        \\ $(R_mkt-RF)$ = return on a value-weighted market index - risk free rate; 
        \\$R_{small}-R_{big}$ = a small-cap return premium equal to the average
        return on small-cap portfolis - the average return on large-cap portfolios;
        \\ $R_{HBM}-R_{LBM}$=a value return premium equal to the average return on high-book-to-market
        portolios - the average return on low-book-to-market portfolios
        \\Expected value for $\beta_{mkt,j}=1$, latter two $\beta$ are 0
    \item Pastor-Stambaugh Model: adds a liquidity factor to the F-F model.
        base: 0; less liquid: positive $\beta$
    \item Macroeconomic Multifactor Models: use multiple factors and correspoding beta. Burmeister, Roll, and Ross Model:
        \\-Confidence risk
        \\-Time horizon risk
        \\-Inflation risk
        \\-Business cycle risk
        \\-Market timing risk
        \\Add RF and beta times these risks.
    \item Build-up method: good for a company where betas are not readily obtainable
        \\Required return = RF + equity risk premium + size premium + specific-company premium
    \item Bond-Yield Plus Risk Premium Method: good for a company with publicly traded debt
        \\Risk Premium = Risk premium + the YTM of Long-term debt
\ee
\subsection{Explain beta estimation for public companies for public companies, thinly
traded public companies, and nonpublic companies}
\be
    \item Beta Estimates for Public Companies
        \\-Public company: compute by regressing the returns of stock on the returns
        of the overall mkt.
        \\-Common Index: SP500, NYSE; Common time scale: 5 years of monthly or 2 years of weekly data.
    \item Adjusted Beta for Public Companies
        \\-beta drift: tendency of an estimated beta to revert to 1.0 over time
        \\-Blume method: adjusted beta = (2/3 regression beta) + (1/3$\times$1.0)
    \item Best Estimates for Thinly Traded stocks and Nonpublic Companes
        \\Steps
        \\-Find a benchmark company XYZ, publicly traded and similar
        \\-Estimate the benchmark beta
        \\-Find Unlevered beta: unlevered beta for XYZ = (beta of XYZ) $1/(1+debt of XYP/equity of XYZ)$
        \\-Estimated beat for ABC = unlevered beta of XYZ $\times (1+debt of ABC/equity of ABC)$ 
\ee
\subsection{Describe strengths and weaknesses of methods used to estimate the required
return on an equity investment}
\be
    \item CAPM
        \\S: simple, one factor
        \\W: need to choose appropriate factor; low explanatory power
    \item Multifactor models:
        \\S: higher explanatory power
        \\W: Complex, expensive
    \item Build-up models: 
        \\S: simple
        \\W: use historical values that may or may not be relevant to current market conditions
\ee
\subsection{Explain international considerations in required return estimation}
Need to consider the exchange risk. Two models
\be
    \item Country Spread Model: Use a developed market as benchmark and add a risk premium
        (difference between the yield on bonds)
    \item Country Risk Rating Model: Use a model, and take the risk rating as 
        independent variable.
\ee
\subsection{Explain and calculate the weighted average cost of capital for a company}
WACC=
$$
    \frac{market value of debt}{mkt value of debt and equity}\times r_d \times(1-tax rate)
    +\frac{mkt value of equity}{mkt value of debt and euqity}\times r_e
$$
\subsection{Evaluate the appropriateness of using a particular rate of return as
a discount rate, given a description of the cash flow to be discounted and other
relevant facts}
CF to the entire company: WACC
\\CF to equity: required return on equity


\section{Reading 29: Industry and Company Analysis}
\subsection{Compare top-down, bottom-up, and hydrid approaches for developing
inputs to equity valuation models
}
\be
    \item Bottom-Up: starts with an individual company
    \item Top-down: starts with macroeconomic variables, often the expected growth rate of nominal
        GDP
    \item Hybrid: incorporate both. Most common.
\ee
\subsection{Compare growth relative to GDP growth and mkt growth and mkt share approaches
to forecasting revenue}
\be
    \item Growth Relative to GDP growth: modeled as GDP plus x\%
    \item Mkt growth and mkt share: mkt share $\times$ mkt growth(industry sales)
\ee
\subsection{Evaluate whether economies of scale are present in an industry by analyzing 
operating margins and sales levels}
\be
    \item Economies of scale: average costs of production down and sales up
    \item Check income stmts. Lower COGS as a proportation of sales for larger companies\ra
        economies of scale in COGS
\ee
\subsection{Forecast the following cost: COGS, SGA, financing cost and income taxes}
\be
    \item Cost of Goods Sold(COGS)
        \\-Forecoast COGS = historical COGS/revenure$\times$ estimate of future revenue
        \\- Or Forecast COGS = (1 - gross margin)$\times$(estimate of future revenue)
    \item Selling General and Administrative Costs(SG\&A)
        \\-SGA operating expenses are less sensitive to changes in sales.
        \\-SGA selling and distribution costs are more directly related to sales volumes.
    \item Financing Cost: interest for debt/equity financing
        \\-Net Debt: gross debt - cash, cash equivalents and short-term securities
        \\-Net interest expense: gross interest expense - interest income on cash/short-term debt
        securities
    \item Income Tax Expense
        \\-Statutory rate: percentage tax charged in the country where the firm is
        \\-Effective tax rate: income tax expense as a percentage of pretax income
        \\-Cash tax rate: cash taxes paid as a percentage of pretax income
\ee
\subsection{Describe approaches to balance sheet modeling}
\be
    \item Inventaory management: inventory turnover. Use forecasted COGS/inventory ratio
        to forecast inventory value
    \item Accounts receivable: Projected accounts receivable = days sales outstangin $\times$ forecasted
        sales/365
    \item Property, Plant, and Equipment(PP\&E)
        \\-Method 1: equal to its historical average proportion of sales
\ee
\subsection{Describe the relationship between return on invested capital and competitive
advantage}
\be
    \item Return  on invested capital = Net operating profit adjusted for taxes /
        Invested capital(operating assets - operating liabilities)
    \item Return on capital empolyed = Pretax operating earnings / invested capital
\ee
\subsection{Explain how competitive factors affect prices and costs}
\subsection{Judge the competitive position of a company based on a Porter's five forces analysis}
\be
    \item Companies have less pricing power when the \textbf{Threat of substitute products}
        high and switching costs are low
    \item Companies have less pricing power when the \textbf{intensity of industry rivalry}
        is high, when industry concentration is less, fixed cost/exit barriers are high, industry
        growth is slow/negative, products not differentiated to a significant degree.
    \item Company prospects for earnings growth are lower when the \textbf{bargaining
        power of supplier} is high
    \item Company has less pricing power when the \textbf{bargaining power of customers} is high
    \item Company has more pricing power and batter for earnings growth when \textbf{threat of new entrants}
        is low
\ee
\subsection{Explain how to forecast industry and company sales and costs when they
are subject to price inflation or deflation}
\be
    \item Costs of input will affect. Hedge or vertical integration can avoid the
        risk.
    \item Raw material Product's elasitcity of demand will affect the price
    \item Analysts should understand a company's hedging activities
\ee
\subsection{Evaluate the effects of technological development on demand, selling prices, costs, and margins}
Advances in technology will decrease costs of production; improved substitutes
or wholly new products.
\\Model the introduction of new substitutes: cannibalizatoin factor -- new product sales
that will replace the old sales.
\subsection{Explain considerations in the choice of an explicit forecast horizon}
\be
    \item Forecast horizon: expected holding period for a stock.
    \item Highly cyclical companies: long enough to include middle of a business cycle
        to get mid-cycle level of sales and profits
    \item Important events(merger, acquision): long enough to include the related pofits.
\ee
\subsection{Explain an analyst's choices in developing projections beyond the
short-term forecast horizon}
\be
    \item Earnings in short term: trend growth rate of revenue will continue
    \item Terminal value: estimated with a relative valuation approach or a discounted cash flow approach
    \item Multiples approach: make sure the multiples used are consistent with the estimated growth rate
        and required rate of return.
    \item Discounted cash flow to estimate the terminal value:
        \\-Cash flows: using a mid-cycle value
        \\-Expected growth rate
    \item IMPORTANT: find inflection points, from when the future will not be like
        the past
        \\-Changes in overall economic environment
        \\-Changes in business cycle stage
        \\-Changes in Government regulations
        \\-Changes in Technology
\ee
\subsection{Demonstrate the developent of a sales-based pro forma company model}
Steps
\be
    \item Estimate revenue growth and future expected revenue
    \item Estimate COGS
    \item Estemate SGA
    \item Estimate financing costs
    \item Estimate income tax expenses and cash taxes
    \item Estimate cash taxes, taking into account changes in deferred tax items
    \item Model the balance sheet based on items that flow from the income stmt
    \item Using depreciation and capital expenditures to estimate capital
        expenditures and net PP\&E for the balance sheet
    \item Use the completed pro forma income stmt and balance sheet
        to construct a pro forma cash flow stmt
\ee

\section{Reading 30:Discounted Dividend Valuation}
\subsection{Compare dividends, free cash flow, and residual income as inputs
to discounted cash flow models and identify investment situations for which
each measure is suitable}
Dividend discount models:
\be
    \item Definition: DDM define cash flow as the dividends to be received by the
        shareholders.
    \item Advantages and dis
        \\-A: theoretically justified; dividends are less volatile than other measures
        \\-DA: hard for firms that don't pay dividends; dividend policy may be controled by someone,
        therefore doesn't show the expected cash flow to share holders
    \item Appropriate for
        \\-Companies that have a history of dividend payments
        \\-Dividend policy is clear and related to the earnings of the firm
        \\-when The valuation Perspective is that of a minority shareholder
\ee
Free cash flow to the firm(FCFF):
\be
    \item Definition: Cash flow generated by the firm's operations that in excess
        of the capital investment required to sustain the firm's current productive
        capacity. 
    \item Free cash flow to equity: cash available to stockholders after funding captial and
    and expenses related with debt financing
    \item Adavantage and disadvantage
        \\-A: good for many firms
        \\-Dis: i) Company may have negative CF for many years.
    \item Appropriate for
        \\-Firms do not have dividend payment history or payment history is not related
        to earnings
        \\-firms which CF corresponds to earnings
        \\-when valuation perspective is of a controlling shareholder.
\ee
Residual income
\be
    \item Definition: RI - earnings that exceeds the investor's required return. 
    \item Advantage and dis
        \\-A: good for companies with negative cf and to divident-and non-dividend-paying
        firms
        \\-Dis: require in-depth analysis on accountings. 
    \item Appropriate:
        \\- firms that do not have dividend histories
        \\- negative free cf
        \\- firms with transparent financial reporting
\ee
\subsection{Calculate and interpret the value of a common stock using the dividend
discount model for single and multiple holding periods}
\be
    \item One-Period DDM: $V_0=\frac{D_1+P_1}{1+r}$
        \\D$_1$: dividends to be received at end of year 1
        \\P$_1$: price expeted upon sale at end of year 1
        \\-required return
    \item Two-Period DDM: $V_0=\frac{D_1}{(1+r)^1}+\frac{D_2+P_2}{(1+r)^2}$
    \item Multi-Period DDM: $V_0=\frac{D_1}{(1+r)^1}+\frac{D_2}{(1+r)^2}+...+\frac{D_n+P_n}{(1+r)^n}$
\ee
\subsection{Calculate the value of a common stock using the Gordon growth model and
explain the model's underlying assumptinos}
\be
    \item Assumptions: dividends increase at a constant rate
        \\-Firm expects to pay a dividend,$D_1$, in one year
        \\-Dividends grow in a costant rate $g$
        \\-the growth rate $g$ is less than the required return $r$
    \item Formula:
        $V_0=\frac{D_0(1+g)^1}{1+r}+...+\frac{D_0(1+g)^n}{(1+r)^n}=\frac{D_0(1+g)}{r-g}
        =\frac{D_1}{r-g}$
\ee
\subsection{Calculate and interpret the implied growth rate of dividends using
the Gordon growth model and current stock price}
Gordon $V_0=\frac{D_1}{r-g}$. Know 3 of them, we can find the last one.
\subsection{Calculate and interpret the present value of growth opportunities
(PVGO) and the component of the leading price-to-earnings ratio(P/E) related to PVGO}
$V_0=\frac{E_1}{r}+PVGO$
where
\\E$_1$=earnings at t=1
\\r=required return on equity
If the company has additional opportunities, it will be great.
\subsection{Calculate and interpret the justified leading and trailing P/Es using the Gordon
growth model}
\be
    \item Leading P/E = $\frac{P_0}{E_1}=\frac{D_1/E_1}{r-g}=\frac{1-b}{r-g}$
        \\Trailing P/E = $\frac{P_0}{E_0}=\frac{D_0\times(1+g)/E_0}{r-g}=\frac{(1-b)\times(1+g)}{r-g}$
        \\Here, $P_0$= fundamental value, $D_0$=dividends just paid, $D_1$=dividends
        expected to be received in one year, $b$=retention ratio, $1-b$=dividend payout ratio
\ee
\subsection{Calculate the value of noncallable fixed-rate perpetual preferred
stock}
\be
    \item Value of perpetual preferred shares = $\frac{D_P}{r_p}$, where $D_P$
        is preferred dividend, and $r_p$=cost of preferred equity.
\ee
\subsection{Describe strengths and limitations of the Gordon growth model and
justify its selection to value a company's common shares}
\be
    \item Strength:
        \\-good for stable, mature, dividend-paying firms
        \\-good for valuing market indices
        \\-easily communicated and explaind
        \\-Can be used to determin price-implied growth rates, required rates
        of return and value of growth opportunities
    \item Limitations
        \\-Valuations are sensitive to growth rates and required returns
        \\-not good for non-dividend-paying stocks
        \\-unpredictable growth patterns of some firms will be bad for the company
\ee
\subsection{Explain the assumptions and justify the selection of the two-stage DDM,
the H-model, the three-stage DDM, or spreadsheet modeling to value a company's common
shares}
\be
    \item Two stage DDM:
    \\the company grows at a high rate for a short time, then back to a long-run
        perpetual growth rate. ie. Dividend growth rates will change.
    \item H-Model:
        \\ The growth rate will declines linearly over the high-grwoth stage
        until it reaches the long-run average growth rate.
    \item Three-stage DDM: good for firms that have three distinct stages of earnings growth.
    \item Spreadsheet modeling: in practice, use spreadsheets to model any pattern of dividend
        growth every year.
\ee
\subsection{Explain the growth phase, transitional phase, and matruity phase of a business}
\be
    \item Initial growth phase: rapidly increasing earnings, little or no dividends, and heavy reinvestment
    \item Transition phase: earnings and dividends are increasing but still low
    \item Maturity phase: earnings growth at a stable but slower rate, payout ratios
        are stable, ROE=required rate of return
\ee
\subsection{Describe terminal value and explain alternative approaches to determining
the terminal value in a DDM}
Ways to determin the terminal value:
\be
    \item Gordon growth model: at some point of futre, assume dividends will begin
        to grow at a constant.
    \item Market price multiples: estimate market price multiples, and using it to
        estimate future price.
\ee
\subsection{Calculate and interpret the value of common shares using the two-stage of DDM, 
the H-model, and the three-stage DDM.}
\be
    \item Valuation Using the Two-Stage Model: 
        $$
        V_0=\sum_{t=1}^n\frac{D_0(1+g_s)^t}{(1+r)^t}+ \frac{D_0(1+g_s)^n(1+g_L)}{(1+r)^n(r-g_L)}
        $$
    \item Valuation using the H-Model
        $$
        V_0=\frac{D_0(1+g_L)}{r-g_L}+\frac{D_0\times H\times(g_s-g_L)}{r-g_L}
        $$
        \\Where $H=t/2=$half-life (in years) of high-growth period
    \item Valuation using the Three-stage DDM
        \\Similar to two-stage DDM
\ee
\subsection{Estimate a required return based on any DDM, including the Gordon growth model and the H-model}
LOS doesn't need you to calculate the required return.
\be
    \item From Gordon growth model: $r=\frac{D_1}{P_0}+g$
    \item From H-Model: $r=\frac{D_0}{P_0}((1+g_L)+H(g_s-g_L))+g_L$
\ee
\subsection{Explain the use of spreadsheet modeling to forecsh dividends
and to value common shares}
Need to estimate the terminal value of a security. Then discount it back to current days.
\subsection{Calculate and interpret the sustainable growth rate of a compnay and demonstrate the
use of DuPont analysis to estimate a company's sustainable growth rate.}
SGR: the rate at which earnings and dividends can continue to grow indefinitely
\\SGR=b$\times$ROE
$$
    ROE=\frac{net income}{stockholder's equity}=\frac{net income}{sales}
    \frac{sales}{total assets}\times\frac{total assets}{stockholder's equity}
$$
PRAT Model: Profit margin, Retention rate, Asset turnover, and financial leverage
$$
    SGR=b\times ROE=\frac{Net income-dividends}{net income}\frac{net income}{sales}
    \frac{sales}{total assets}\times\frac{total assets}{stockholder's equity}
$$
\subsection{Evaluate whether a stock is overvalued, fairly valued, or undervalued by the
market based on a DDM estimate of value.}
If market price higher than the DDM price, then it's overvalued. 

\section{Reading 31: Free Cash Flow Valuation}
Free Cash Flow: Free cash flow removes cash to pay operation expense, working capital
investment, and fixed capital investment.\\
FCFF: cash available to stockholder's and bondholders.
\subsection{Compare the FCFF and FCFE approaches to valuation.}
\be
    \item Firm value = FCFF discounted at WACC
    \item Equity value = FCFE discounted at required return on equity
    \item Equity value = Firm value - Market value of debt
\ee
Difference between FCFF and FCFE:
\be
    \item FCFE is easier and straightforard
    \item FCFF is good if the firm has significant debt outstanding.
\ee
\subsection{Explain the ownership perspective implicit in the FCFE approach}
Ownership perspective
\be
    \item Free cash flow approach
        \be
        \item Control perspective: for an acquirer, who can change the firm's dividend policy
        \item For minority shareholders perspective
        \ee
    \item Dividend approach
        \\ Perspective: minority owner. They have no direct control over the firm's dividend
        policy.
\ee
Why prefer Free Cash Flow than dividend-based valuation?
\be
    \item Companies pay no/low cash dividends.
    \item Dividends are decided by the board of directors. Not align with the firm's long-run profitability.
    \item Free Cash Flow are better if the company is an acquisition target. B/c new owners
        might change the distribution of dividends.
    \item FCF is more related to long-run profitability.
\ee
\subsection{Explain the appropriate adjustments to net income, EBIT, EBITDA, CFO to calculate
FCFF and FCFE}
Terms:
\be
    \item EBIT: Earnings before inteerest and taxes
    \item EBITDA: Earnigns before interest, taxes, depreciations, and amortization
    \item CFO: Cash flow from operations
\ee
Calculating FCFF from net income: 
\\FCFF = NI + NCC + Int$\times$(1-tax rate)-FCInv-WCInv
\be
    \item NCC(non cash charges): like amortization of intangible assets, amortization of a bond discount, 
        deferred taxes.
    \item Fixed capital investment: they do not appear on the income statement, but they represent
        cash leaving the firm.
        \\FCInv = Capital expenditures - proceeds from sales of long-term assets
        \be
            \item If no long-term assets were sold:FCInv = ending net PP\&E - beginning PP\&E + depreciation
            \item If long-term assets sold: FCInv = ending net PP\&E - beginning PP\&E + depreciation - Gain on sale.
        \ee
        \be
            \item Working capital investment: Change in working capital, excluding cash, cash equivalents, notes payble, 
                and the current portion of long-term debt
            \item Interest expense: should add it back.
        \ee
\ee
Other Formulas:
\be
    \item Calculating FCFF from EBIT:
    \\FCFF=EBIT$\times$(1-tax rate)+Depreciation - FCInv - WCInv
    \item Calculating FCFF from EBITDA
        \\FCFF=EBITDA$\times$(1-tax rate) + Dep$\times$tax rate - FCInv - WCInv
    \item Calculating FCFF from CFO
        \\FCFF=CFO+Int$\times$(1-tax rate)-FCInv
    \item Calculating FCFE from FCFF:
        \\FCFE=FCFF-Int$\times$(1-tax rate)+net borrowing
    \item Calculating FCFE from net income
        \\FCFE=NI+NCC-FCInv-WCInv+Net borrowing
    \item Calculate FCFE from CFO:
        \\FCFE=CFO-FCInv+Net borrowing
\ee
Free Cash flow with Preferred stock
\be
\item FCFF/FCFE above assume that the company only have debt and common equity.
Preferred stock dividends should be added back to get FCFF. PS can be considered as
debt, but dividends are not tax deductible.
\ee
\subsection{Calculate FCFF and FCFE}
\be
    \item FCFF = Changes in cash balance + Net payments to debt providers
        + Net payments to equity stakeholders
    \item FCFE = Changes in cash balances + Net payments to equity stakehoders
\ee
\subsection{Describe approaches for forecasting FCFF and FCFE}
Two approaches
\be
    \item Historical Free Cash Flow:
        \\-Assumptions: constant growth, constant fundamental factors
    \item Components of free cash flow
        \\-Assumptions: each component grow at different rate; ties sales forecasts
        to future captial expenditures, depreciation expenses, and changes in working
        capital
        \\-IMPORTANT: Capital expenditures\ra i) Outlays that are needed to
        maintain existing capacity: current levels of sales; ii) Outlays that are
        needed to support growth: predicted sales growth.
\ee 
Forecasting FCFE with the second method, Target debt-to-asset ratio for the new
investment is fixed in Fixed Capital and Working Capital
\\FCFE=NI-(1-DFR)$\times$(FCInv-Dep)-(1-DR)$\times$WCInv
\subsection{Compare the FCFE model and dividend discount models}
The FCFE: control perspective\\
Dividend discount: Minority perspective
\subsection{Explain how dividends, share repurchases, share issues, and changes
in leverge may affect future FCFF and FCFE}
dividends, share repurchases, share issues have no effect on FCFF and FCFE
\\Changes in leverge have no effects on FCFF and minor effects on FCFE.
\subsection{Evaluate the use of net income and EBITDA as proxies for cash flow in valuation}
Net income: Poor for FCFE
\\FCFE=NI+NCC-FCInv-WCInv+net borrowing
\\EBITDA: poor for FCFF
\\FCFF=EBITDA$\times$(1-tax rate)+Dep$\times$tax rate-FCInv-WCInv
\subsection{Explain the single-stage, two-stage, and three-stage FCFF nd FCFE models
and select and justify the appropriate model given a company's characteristics}
Single-Stage FCFF Model
\be
    \item Assumptions: i)FCFF grows$@$constant g. ii)growth rate$<$WACC
    \item Eqution: Value of the firm = $\frac{FCFF_1}{WACC-g}$=
        $\frac{FCFF_0\times (1+g)}{WACC-g}$
\ee
Single-stage FCFE Model
\be
    \item Assumptions: Similar to FCFF
    \item Value of the equity = $\frac{FCFE_1}{r-g}$=
        $\frac{FCFE_0\times (1+g)}{r-g}$
    \item Often used in international valuation
\ee
Multistage Models: How Many Variations are there?
\be
    \item key: Value = PV of expected future CF discounted at the appropriate discount rate
    \item Comparisons
        \\-FCFF VS FCFE: FCFF - WACC, FCFE - required return on equity
        \\-Two stage vs three stage: just different time scale
        \\-Forecasting growth in total free cash flow (FCFF or FCFE) VS forecasting
        the growth rates in the components of free cash flow: These two models are good.
\ee
Model Assumptions and Firm Characteristics
\be
    \item Two- or three stage? two stage: short-term supernormal growth; three-stage: like
        growth phase, mature phase, and a transistion phase
\ee
\subsection{Estimate a company's value using the appropriate free cash flow models}
Examples on Vol 3 P137
\subsection{Explain the use of lsensitivity analysis in FCFF and FCFE Valuations}
\be
    \item Sensitivity analysis: how sensitive an analyst's valulation results are to changes
        in each of a model's inputs
    \item Two majors sources
        \\- Estimate the future growth in FCFF and FCFE, growth forecasts depend on a firm's
        future profitability.
        \\- The chosen base years for FCFF or FCFE growth forecasts
\ee
\subsection{Describe approaches for calculating the terminal value in multistage valuation model}
Two way
\be
    \item Use single-stage model, assume that the forecasted FCFF and FCFE will grow at
        the long-term stable growth rate after some time
    \item Use multiples: 
        \\- Terminal value in year n = trailing P/E$\times$ earnings in year n
        \\- Terminal value in year n = leading P/E$\times$ earnings in year n+1
\ee


\section{Reading 32:Market-Based Valutation: Price and Enterprise Value Multiples}
\subsection{Distinguish between the method of comparables and the method based on
forecasted fundamentals as approaches to using price multiples in valuation, and
explain economic rationales for each approach.}
\be
    \item Method of comparables: values a stock based on the average price multiple
        of the stock of similar companies
        \\- Law of One Price
    \item Method of forecasted fundamentals: values a stock based on the ratio of
        its value from a discounted cash flow model to some fundamental variables.
        \\- Value = PV of all future cash flows
\ee 
\subsection{Calculate and interpret a justified price multiple}
\subsection{Describe rationales for and possible drawbacks to using alternative price multiples
and dividend yield in valuation}
\subsection{Calculate and interpret alternative price multiples and dividend yield}
Justified price multiple: The multiple when the stock is fairly valued.
\\P/E Ratio: price to earnings
\be
    \item A and DisA:
        \\A: popular used, P/E differences different significantly related to long-run average
        \\DisA: Earnings can $<$0, Earnings can be volatile, earnings can be distored in reports.
    \item Equation:
        \\Trailing P/E = Mkt Price per Share/EPS over previous 12 monthes. 
        \\Leading P/E = Mkt Price per Share/EPS expected over next 12 months
\ee
P/B Ratio: price to book value
\be
    \item A and DisA:
        \\A: Book value usually$>$0; Bv mare stable than EPS; Good for firms
        that hold liquid assets; good for firms that are expected to go out of business;
        good to explain differences in long-run average stock returns
        \\DisA: exclude intangible economic assets; misleading; accounting conventions may
        different; inflation/techo change will change the book value/mkt values of assets
    \item P/B ratio = Mkt value of equity/book value of equity = Mkt price pshare/BValue pshare
        \\Book value of equity = Common shareholder's equity = Total Assets - total liabilities - Preferred stock
\ee
P/S Ratio: price to sale
\be
    \item A and DisA
        \\A: always positive; hard to manipulate sales; not volatile; good for
        mature/cyclical industries or start-up
        \\DisA: high growth in sales$\neq$high profit; P/S ratio doesn't describe
        cost structures; revenue recognization may be distorted.
    \item P/S ratio= Mkt value of equity/total sales 
\ee
P/CF ratio: price to cash flow
\be
    \item A: CF is hard to manipulate; P/CF more stable than P/E
    \item Dis: If we use CF=Eps + noncash charges, noncash revenue and net changes
        in working capital are ignored, though they will affect actual cash flow;
        FCFE is preferable to operating CF, however, FCFE is more volatile.
\ee
Dividend Yield: common dividend to the market price
\be
    \item A and DisA:
        \\-A: dividend yield contributes to total investment return; dividends are not as risky as the 
        capital appreciation component of total return
        \\-DisA: DY doesn't include capital appreciation;
    \item Equation
        \\ Trailing D/P = 4$\times$ most recent quarterly dividend/mkt price ps
        \\ Leading D/P = forecasted dividends over next four quarters / mkt ps
\ee
\subsection{Calculate and interpret underlying earnings, explain methods of normalizing
earnings per share(EPS), and calculate normalized EPS}
Underlying Earnings: 
\be
    \item Definition: earnings - nonrecurring components(gain/loss from asset
sales, asset write-downs, provisions for future losses, and changes in accounting estimates)
    \item Molodovsky effect: earnings have a transitory portion that is due to
        cyclicality. In business cycles, P/Es will be high due to lower EPS at the bottom of the cycle
        and low P/Es due to high EPS at the top of the cycle.
\ee
Normalized Earnings: use estimate of EPS in the middle of business cycle
\be
    \item Method of historical average EPS: average EPS over some recent period
    \item Method of average return on equity: average return on equity times the 
        current book value per share = average ROE$\times$BVPS. --Preferred.
\ee
\subsection{Explain and justify the use of earnings yield(E/P)}
When earnings$<$0, $P/E$ ratios are meaningless. we can use $E/P$:
\\ High E/P\ra cheap security; Low E/P\ra Expensive security.
\subsection{Describe fundamental factors that influence alternative price multiples and dividend yield}
\subsection{Calculate and interpret the justified P/E,P/B,P/S ratios for a stock, based on forecasted
fundamentals}
Justified P/E Multiple
\be
    \item Equations:
        $$
        P_0=V_0=\frac{D_0\times (1+g)}{r-g}=\frac{D_1}{r-g}
        $$
        
        Trailing P/E=$$\frac{P_0}{E_0}=\frac{D_0\times(1+g)/E_0}{r-g}=\frac{(1-b)(1+g)}{r-g}
        $$
        Leading P/E=$$
        \frac{P_0}{E_1}=\frac{D_1/(r-g)}{E_1}=\frac{1-b}{r-g}
        $$
        Here, $b$ is the retention rate.
\ee
Justified P/B Multiple
\be
    \item Equation and derivation
        \eq{
            P_0&=D_1/(r-g)=E_1\times (1-b)/(r-g)\\
               &=B_0\times ROE\times(1-b)/(r-g)\\
               &=B_0\times (ROE-g)/(r-g)\\
            g&=ROE\times b, E_1=B_0\times ROE\\
            P/B&=P_0/B_0=\frac{ROE-g}{r-g}
        }
    \item P/B increase
        \\ ROE increase
\ee
Justified P/S Multiple
\be
    \item Equation and derivation
        \eq{
            \frac{P_0}{S_0}&=\frac{E_0/S_0\times(1-b)(1+g)}{r-g}\\
                &=(E_0/S_0)\times\frac{(1-b)\times(1+g)}{r-g}=net\ profit\ margin\times\ justified\ 
            trailing P/E
        }
    \item P/S will increase if
        \\Profit margin increases
        \\Earnings growth rate increase
\ee
Justified P/CF Multiple
\be
    \item Equations and derivation
        \\From FCFE: $V_0=FCFE_0\times(1+g)/(r-g)$
        \\P/CF=$V_0/CF_0$
    \item P/CF increase if
        \\r down, g up
\ee
Justified EV/EBITDA Multiple: Enterprise value/EBITDA
\be
    \item Positively related to the growth rate in FCFF and EBITDA
    \item Netagively related to WACC
\ee
Justified Dividend Yield
\be
    \item $D_0/P_0$=(r-g)/(1+g)
    \item Positively related to r, negatively related to growth rate in dividends.
\ee
\subsection{Calculate and interpret a predicted P/E, given a cross-sectional
regression on fundamental, and explain limitations to the cross-sectional regression
methodology}
\be
    \item Calculation: linear regression of hisotrical P/Es on fundamental variables.
    \item Limitations
        \\-Predictive power of the estimated P/E for a different is uncertain
        \\-Relationships between P/E and the fundamental variables examined may
        change over time
        \\-Multicollinearity, makes it difficult to interpret individual regression coefficients
\ee
\subsection{Evaluate a stock by the method of comparables and explain the importance of fundamentals
in using method of comparables}
\subsection{Evaluate whether a stock is overvalued, fairly valued, or undervalued based on comparisons
of multiples}
Use multiples to value stocks
\be
\item P/E
\be
    \item P/E Multiples$<$the benchmark\ra Undervalued
    \item Usual P/E multiples
        \\-P/E of another company's with similar properties
        \\-average or median peer group
        \\-P/E of an index
        \\-average historical P/E for the stock.
\ee
\item P/B
\be
    \item Similar to P/E
    \item Difference:
        \\-Use trailing book values in calculating P/Bs
\ee
\item P/S
\be
    \item Similar to P/E
    \item Differnce
        \\-Use trailing sales
\ee
\item EV/EBITDA
\be
    \item $<$ benchmark\ra undervalue
\ee
\item Dvidend yield
\be
    \item high dividend yield\ra undervalue
\ee
\ee
The Fed and Yardeni models
\be
    \item Fed model: the overall mkt overvalued when the earnings yield(E/P) on
        the SP500 $<$ the yield on 10-year US Treasury bonds.
    \item Yardeni model
        \\-CEY=CBY-k$\times$LTEG+$\varepsilon_i$
        \\CEY=Current earnings yield of the market; CBY=current Moody's A-rated
        corporate bond yield
        \\LTEG=five-year consensus earnings growth rate; k=constant, $\approx$0.20
\ee
\subsection{Calculate and interpret the P/E-to-growth ratio(PEG) and explain its use
in relative valuation}
$$
PEG ratio = \frac{P/E ratio}{g}
$$
PEG ratio ``standardizes'' the P/E ratio for stocks with different growth rates.
Small PEG ratio\ra better.
\\Drawbacks:
\be
    \item P/E and g is not linear.
    \item PEG ratio doesn't account for risk.
    \item PEG ratio doesn't reflect the duration of the high-growth period
        for a multistage valuation model.
\ee
\subsection{Calculate and explain the use of price multiples in determining terminal
value in a multistage discounted cash flow model.}
Two methods:
\\1. based on fundamentals
\be
    \item Terminal value in year n = justified leading P/E $\times$ forecasted earnings
        in year n+1
    \item Terminal value in year n = justified trailing P/E $\times$ forecasted earnings
        in year n
\ee
2. based on comparables
\be
    \item Terminal value in year n = benchmark leading P/E $\times$ forecasted earnings
        in year n+1
    \item Terminal value in year n = benchmark trailing P/E $\times$ forecasted earnings
        in year n
\ee
\subsection{Explain alternative definitions of cash flow used in price and enterprise value(EV)
multiples and describe limitations of each definition}
Definitions of cash flow available for use in calculating the P/CF
\be
    \item Earnings-plus-noncash charges(CF)
    \item Adjusted cash flow (adjusted CFO)
    \item Free cash flow to equity(FCFE)
    \item Earnings before interest, taxes, depreciation and amortization(EBITDA)
\ee
Earnings-plus-noncash-charges(CF)
\be
    \item Definition: CF=net income + depreciation + amortization
    \item Limitation: it ignores items that affect cash flow, like noncash revenue 
        and changes in net working captial.
\ee
Cash Flow from Operations or Adjusted CFO
\be
    \item How to adjust for nonrecurring cash flows: 
        \\-GAAP: interest paid/received, dividends received to be classified
        as operating cash flows
        \\-IFRS: interest paid in Operating/Financing, interest and dividends
        received in Operating/Investing
\ee
FCFE/EBITDA:
\be
    \item FCFE=CFO-FCInv+net borrowing, where FCInv = fixed capital investing,
        net borrowing = (long- and short-term debt issues)-(long- and short-term
        debt repayments)
    \item EBITDA
\ee
P/CF ratio=markt value of equity/cash flow = mkt price per share/cash flow per share
\subsection{Calculate and interpret EV multiples and evaluate the use EV/EBITDA}
\be
    \item Enterprise value(EV) = market value of common stock + mkt value of preferred equity
+ mkt value of debt + minoirty interest - cash and investments
    \item EV/EBITDA ratio = Enterprise value/EBITDA
    \item EBITDA = recurring earnings from continuing operations + interest
        + taxes + depreciation + amortization
        \\=EBIT + deprecitation + amortization
\ee
EV/EBITDA good/bad
\be
    \item Good
        \\-good when comparing firms with different degrees of financial leverages
        \\-EBITDA is useful for valuing capital-intensive businesses with high levels of depreciation and amortization
        \\-EBITDA is usually positive
    \item Bad
        \\-EBITDA might overstate CFO
        \\-FCFF has the amount of capital expenditures, it's more related
        with valuation theory.
\ee
Alternative to EV: Total invested capital(TIC)=mkt value of invested capital=mkt value of 
companies equity and debt.
\subsection{Explain sources of differences in cross-border valuation comparisons}
\be
    \item Differences in accounting methods, cultures, risk, and growth opportunities.
    \item Other differences: accounting treatement for goodwill, deferred income taxes, foreign
        exchange adjustments, R\&D, pension expense, and tangible asset revaluations
\ee
\subsection{Describe momentum indicators and their use in valuation.}
\be
    \item Momentum indicators: mkt price/fundamental variable to the time seris of historical/expected
    value. 
    \\-Example: earning surprise, stdized unexpected earnings, relative strengh.
\ee
Common momentum indicators
\be
    \item Unexpected earnings/earnings surprise=reported EPS-expected EPS
    \item Standardized unexpected earnings(SUE):
        SUE=earnings surprise/standard deviation of earnings surprise
    \\Std comes from historical data.
    \item Relative strength indicators: compare a stock's price or return performance
        with is own historical performance.
\ee
\subsection{Explain the use of the arithmetic mean, the weighted harmonic mean,
and the meian to describe the central tendency of a group of multiples}
Use Weighted harmonic mean to calculate Portfolio or index P/E:
$$
WHM = \frac{1}{\sum_i=1^n \frac{w_i}{X_i}}
$$
where $X_i$ is the P/Es, $w$ is the weights.


\section{Reading 33: Residual Income Valuation}
\subsection{Calculate and interpret residual income, economic value added, and market
value added.}
Definitions 
\be
    \item Residual income = economic profit = net income of a firm - a charge that measures stockholder's
opportunity cost of capital.
    \item Residual income explicitly deducts all capital costs.
\ee
EVA and MVA
\be
    \item Econmic value added: EVA = NOPAT - (WACC$\times$total capital) = [EBIT$\times$(1-t)]-\$WACC
        \be
            \item Use beginning year capital
            \item Terms:
                \\NOPAT= Net opearting profit after tax(before substrating interest expenses)
                \\t=marginal tax rate
                \\ \$WACC= dollar cost of capital
                \\ total capial=net working capital + net fixed assets
                    = book value of long-term debt + book value of equity
            \item Adjustments that may needed
                \\-Capitalize and amortize research and development charges, and add them
                back to earnings to get NOPAT.
                \\-Add back charges on strategic investments that will generate returns in the future
                \\-Eliminate deferred taxes an consider only cash taxes as an expense.
                \\-Treat operating leases as capital leases
                \\-Add LIFO reserve to invested capital and add back change in LIFO reserve to 
                NOPAT.
        \ee
    \item Market value added: MVA = market value - total capital=mkt value of 
        equity and long-term debt - book value of invested capital
        \\-Use end-year capital.
\ee
\subsection{Describe the use of residual income models}
For the exam\ra Use residual income models to value equity and goodwill impairment.
\subsection{Calcualte the intrinsic value of a common stock using the residual income model
and compare value recognition in residual income and other present value models}
\be
    \item Equation:
        \\$RI_t=E_t-(r\times B_{t-1})=(ROE-r)\times B_{t-1}$
        \\ $r$= required return on equity; $E_t$=expected EPS for year t.
        \\ $B_{t-1}$=book value per share in year t-1
        \\ ROE=expected return on new investments(expteced return on equity.)
    \item NOTE: Check examples in Notes.
\ee
Residual Income Valuation model
\be
    \item 
        $$
            V_0=B_0+ \frac{RI_1}{(1+r)^1}+\frac{RI_2}{(1+r)^2}+...
        $$
    \item Limitations:
        \\-assumptions needed to calculate the PV of residual incomes.
\ee 
\subsection{Explain fundamental determinants of residual income}
Single-stage residual income valuation model
\be
    \item Equations:
        $$
            V_0=B_0+[\frac{(ROE-r)\times B_0}{r-g}]
        $$
    \item Assumptions:
        \\-Constant dividend/earnings growth rate
        \\-Stock is correctly valued: $V_0=P_0$
    \item Fundamental drivers of residual income
        \\- ROE must be greater than $r$ to get additional value.
    \item Tobin's Q:
        \\Q=(mkt value of debt + mkt value of equity)/(replacement cost of total assets)
        \\=total mkt value of firm/total asset value of firm
\ee
\subsection{Explain the relation between residual income valuation and the justified
price-to-book ratio based on forecasted fundamentals.}
Residual income models are most closely related to the price-to-book value.
\subsection{Calculate and interpret the intrinsic value of a common stock using
single-stage(constant-growth) and multistage residual income models.}
Just check the examples in the Notes. Nothing special here.
\subsection{Calculate the implied growth rate in residual income, given the market
price-to-book ratio and an estimate of the required rate of return on euqity.}
Use equation:
$$
g=r-[\frac{B_0\times(ROE-r)}{V_0-B_0}]
$$
\subsection{Explain continuing residual income and justify an estimate of continuing
residual income at the forecast horizaon, given company and industry prospects}
Continuing residual income:
\be
    \item General summary: forecast short-horizon(5 years) residual income, and then use assumptions
        for long-term residual income.
    \item Persistence factor:$\omega$, the projected rate at which the residual income is going
        to fade over the life cycle of the firm
    \item Assmumptions(use one of below)
        \\-Residual income will persist at its current level forever
        \\-Residual income will drop to zero immediately
        \\-Residual income will decline over time as ROE falls to the cost
        of equity
        \\-Residual income will decline to a long-run average level consistent
        with a mature industry
    \item Eqution:
        $V_0=B_0$+PV of hig-growth RI + PV of continuing residual income
        \\NOTE: PV of continuing residual at the end of year T-1 = RI$_T$/(1+r-$\omega$)
\ee
\be
    \item Assumptions 1: Residual income persist at the current level forever
        \\$\omega=1$
        \\PV of continuing residual income in year T-1 = RI$_T$/r
    \item Assumptions 2: Residual income drops to 0
        \\$\omega=0$
        \\PV of continuing residual income in year T-1 = RI$_T$/(1+r)
    \item Assumptions 3: Residual income declines over time to zero
        \\$\omega\in[0,1]$
    \item Assumptions 4: Residual income declines to long-run level in mature industry.
        \\PV of continuing residual in year T = P$_T$ - B$_T$, where P$_T$=mkt value.
        \\-Estimate P$_T$: P$_T$=B$_T\times$ Price-to-book value
        \\-PV of coninuing residual income in year T-1 = $(P_T-B_T+RI_T)/(1+r)$
        
\ee
\subsection{Comapre residual income models to dividend discount and free cash flow models}
DDM and FCFE: Discounting expected cash flows.
\\RI: starts with a book value.
\subsection{Explain strengths and weaknesses of residual income models and justify
the selection of a residual income model to value a company's common stock.}
\be
    \item Strength:
        \\-Terminal value does not dominant.
        \\-RI use accounting data.
        \\-Applicable to firms that do not pay dividends or do not have positive FCF
        \\-applicable when CF is volatile 
        \\-Focus on economic profitability
    \item Weakness
        \\-rely on accounting data, which may be manipulated
        \\-Need many adjustments
        \\-Assumes that $B_t=B_{t-1}+E_t-D_t$ holds(clean surplus model). However, any accounting charges
        that taken directly to equity accounts will break this relation, like currency
        translation gains/losses.
\ee
When to use RI models or not?
\be
    \item When:
        \\-firms do not pay dividends, or payments too volatile
        \\-Expected FCF are negative
        \\-Terminal forecast is highly uncertain
    \item Not
        \\-Clean surplus model broke.
        \\-Estimates of book value and ROE is uncertain.
\ee
\subsection{Describe accounting issues in applying residual income models.}
Clean surplus violations, delete them
\be
    \item Foreign currency translation gains/losses that go to retained eranings
    \item Certain pension adjustments
    \item Gain/losses on certain hedging instruments
    \item Changes in revaluation surplus for long-lived assets. IFRS only.
    \item Changes in the mkt value of debt and equity that classified as available-for-sale.
\ee
Variations from Fair Value: accrual method may make book value are different from market values
\be
    \item Operating leases should be capitalized rather than expensed.
    \item Special purpose entities' A/L should be consoliated to the parent company.
    \item Reserve and allownaces should be adjusted.
    \item Inventory with LIFO should be adjusted to FIFO.
    \item Pension asset or liability should be adjusted to reflect the funded 
        status of the plan.
    \item Deferred tax liabilities should be eliminated and reported as equity if
        it's not going to reverse.
\ee
Intangible Asset Effects on book value
\be
    \item Intangibles recognized at acquisition
    \item R\&D expenditures: productive R\&D increase ROE and RI.
\ee
Nonrecurring items and Other Aggressive Accounting Practices: Shouldn't include them to
calculate CONTINUING RI.
\\International Accounting Differences.
\subsection{Evaluate whether a stock is overvalued, fairly valued or undervalued
based on a residual income model.}
Market price$>$model price\ra Overvalued.


\section{Reading 34: Private Company Valuation}
\subsection{Compare public and private company valuation}
\be
    \item Company-Specific Factors
        \\-Stage of lifecycle: usually less mature, but sometimes mature firms/bankrupt firms near liquidation.
        \\-Size: less capital, fewer assets, and fewer employees. Riskier.
        \\-Quality and depth of management: maybe bad than public firms
        \\-Management/shareholder overlap: substantial ownership position, long-term perspective
        \\-Short-term investor: public firms usually take a shorter-term view, b/c shareholders
        often focus on short-term performance
        \\-Quality of financial and other information: less information about financial
        disclosures.
        \\-Taxes: private firms will be more concerned with taxes than public firms
    \item Stock-specific fators
        \\-Liquidity: equity has fewer potential owners, less liquid
        \\-Restrictions on marketability: have agreement for not selling shares
        \\-Concentration of control: concentrated in a few shareholders
\ee
\subsection{Describe uses of private business valuation and explain applications
of greatest concern to financial analyst}
3 Reasons: transactions, complicance, and litigation
\be
    \item Transaction-Related Valuation
        \\-Venture capital financing: valuation subject to negotiation and informal
        \\-Initial public offering(IPO): use values of similar public firms as benchmark
        \\-Sales in an acquision
        \\-Bankruptcy proceedings
        \\-Performance-based managerial compensation
    \item Compliance-related valuations
        \\-Financial reporting: often related to goodwill impairment test
        \\-Tax purposes
        \\-Litigation-Related valuation: shareholder suits, damage claims, lost profits claims
        or divorce settlements
\ee
\subsection{Explain various definitions of value and demontrate how different definitions
can lead to different estimates of value}
Definitions of Value: depends on what valuation will be used for
\be
    \item Fair market value: for tax purposes
    \item Fair value for financial reporting
    \item Fair value for litigation
    \item Market value: used for real assets where purchase will be levered.
    \item Investment value: value to a particular buyter
    \item Intrinsic value
\ee
The effect of Value Definition on Estimated Value
\be
    \item The definition of value affects the estimated value of an asset.
    \item The valuation should be used for its intended purpose.
\ee
\subsection{Explain the income, market, and asset-based approaches to private company
valuation and factors relevant to the selection of each approach.}
Three approached:
\be
    \item Income approach: value a firm as PV of expected future income
    \item Market approach: value a firm using the price multiples
    \item Asset-based approach: values a firm's asset - liabilities
\ee
Related issues
\be
    \item Selection: depends on the firm's operations and its lifecycle stage
        \\-Early: asset-based
        \\-Developing: income approach
        \\-Mature: market approach
    \item Firm size: parameters from large benchmark company shouldn't be used in small companies
    \item Should include both Operating/nonoperating assets
\ee
\subsection{Explain cash flow estimation issues related to private companies
and adjustments required to etimate normalized earnings}
Normalized earnings: ``Firm earnings IF the firm were acquired." Adjustments needed are
listed below.
\\Estimating Normalized Earnings
\be
    \item Exclude nonrecurring and unusual items. Adjustments may be needed for highly concentrated
        small firms, like internal transactions between the firm and its owner.
        \\-Example: high management compensation\ra normalized earnings will be larger than reporting earnings
        low management compensation\ra normalized earnings will be smaller than reporting earnings
    \item Company owned real estate: should be separated from the firm's operations
        \\Treatment: remove related income/expense. Use market lease rates.
\ee
Strategic and Nonstrategic Buyers
\be
    \item Strategic buyers: need to consider the perceived synergies related to other assets
    \item Nonstrategic buyers: no synergies
\ee
Estimating Cash Flow
\be
    \item Controlling/uncontrolling equity interests will have different values
    \item Scenarios should be considered if there is significant uncertainty
        \\-Development: sale of the firm, IPO, bankruptcy, continued private operation
        \\-Mature firm: different assumed growth rates
    \item Aware: management may be biased on future estimates
    \item FCFF/FCFE: FCFF is more approprite when significant changes are anticipated.
\ee
\subsection{Calculate the value of a private company using free cash flow, 
capitalized cash flow, and/or excess earnings methods}
ALL of them are based on INCOME APPROACH
\be
    \item Free Cash Flow
        \\-Terminal value: usually five years out. 
        \\-use constant growth model/price multiple approach
        \\-NOTE: double counting. In a high growth industry, PE ratio may resulted in double
        counting growth.
    \item Capitalized Cash Flow
        \\-Good for: no comparables are available, projections are uncertain and stable
        growth is reasonable
        \\-Value of firm = FCFF$_1$/(WACC-g), where FCFF$_1$ is the expected free
        cash flow to the firm over the next year.
        \\-Value of equity = FCFE$_1$/(r-g)
    \item Retained Earnings/Excess Earnings
        \\-Starts with the earnings that SHOULD be generated by working capital and fixed assets
        \\-- Excess earnings= firm earnings - required return of earnings
        \\-- Value of intangible assets = PV of the stream of excess earnings. Using the excess earnings
        and the growing perpetuity formula from Capitalized Cash Flow.
        \\--Check EXAMPLES!
\ee
\subsection{Explain factors that require adjustment when estimating the discount rate
for private companies}
\be
    \item Size premiums: small companies may have finanical distress. Size premiums are added
        to the discount rates for small private companies.
    \item Availability and cost of debt: prt companies have less access to debt, and WACC will be higher
    \item Acquirer versus target: some acquirer will incorrectly use their own cost of capital(low) to 
        get a wrong high value of target company.
    \item Projection risk: information avaiability; inexperienced manager forecasting
    \item Lifecycle stage
\ee
\subsection{Compare models used to estimate the required rate of return to private company,
(for example, the CAPM, the expanded CAPM, and the build-up approach}
\be
    \item CAPM: Beta is estimated from public firm data. Not good for private firms that won't go public
    \item Expanded CAPM: Includes addtiontional premiums for size and firm-specific(unsystematic) risk
    \item Build-up method: Used when it's impossible to find comparable public firms to estimate beta.
        Beginning with the expected return on the mkt, and add premiums for small size, industry
        factors, and company specific factors.
\ee
\subsection{Calculate the value of a private company based on market approach methods and describe
advantages and disadvantages of each method}
Three main methods: Guideline Public Company (GPCM), Guideline Transactions Method (GTM), Prior Transaction Method (PTM)
\\ Marekt approaches are preferred over income and asset approaches.
Basic issues:
\be
    \item When choosing public comparables, consider industry, operations, size, and lifecycle.
    \item Large private firm valuation is usually EBIT or EBITDA multiples, and Market Value of Invested Capital(MVIC) as numerator.
    \item Small private firms: usually use net income multiples. A revenue multiple might be used for extremely 
        small firms.
\ee
GPCM
\be
    \item Use price multiples from trade data for public companies.
    \item Evaluating a controlling equity interest in a private firm, control premium should be considered.
        , becase public trades transactions are for small noncontronlling interests.
        \be
            \item Transaction type: finanancial transactions has a smaller price premium.
            \item Industry conditions: Sometimes, there is a flurry in industry acquisition activitity, which
                drives up acquisition prices. Control premium are somewhat included in such markets.
            \item Type of consideration: Historical Acquisitions may involve stocks, and the value are "bubbled", which
                may overstate the estimation of the control premium.
            \item Resonableness: The estimation of control premiums should be checked.
        \ee Controlling premium only applies to the equity portion of the firm's value. Two way to include:
            \\-Use raw multiple to estimate firm value, estimate the equity portion, and apply the control premium to the equity
            portion.
            \\-Adjust controlling premium: Ajusted premium = (control premium on equity)$\times$(1-DR), where DR = debt-to-asset
            ratio of the private company.
    \item 
\ee
GTM
\be
    \item Prior acquistion values for entirme companies are used, and no additional
        adjustment for a controlling interest is necessary.
    \item Issues related to limitie information of private companies
        \be
            \item Transaction type: need to adjust the historical multiple, if data
                is strategic transaction while the valuated subject is non-strategic
                transition.
            \item Contingent consideration: if part of the acquisition price is contingent
                on the achievement of specific company performance, analysts should be
                pay more attention when comparing transactions without such contingencies.
            \item Type of consideration: stock transactions or cash transactions.
            \item Availability of data: relevance and accurancy of the historical data
            \item Date of data
       \ee
\ee
Price Transaction Method
\be
    \item Use transactions data fro the stock of the actual subject company. Most
        appropriate when valuing minority interests.
\ee
\subsection{Describe the assetbased approach to private company valuation}
\be
    \item Value of the firm =  fair value of its assets - fair value of its liabilities
    \item Usually not used. Because it's easier to find comparable data at the firm level
        than asset level; difficult to find intangible assets data
    \item Results in lowest valuation, bc firm's assets combination together will have more
        value than just simply sum them together\ra 1+1$>$2
    \item Good for
        \\-Firms with minimal profits and little hope for better prospects
        \\-Finance firms such as banks
        \\-Investment companies such as real estate investment trusts and closed-end
        investment companies where the underlying assets values are determined using
        the market or income approaches.
        \\-Small companies or early stage companies with few intagible asests
        \\-Natural resource firms
\ee
\subsection{Explain and evaluate the effects on private company valuations of discounts
}
Adjustments are needed when liquidity or control position of an anquisition differs from 
that of the comparable companies: Comprable data is controlling interest, and subject valuation
is noncontrolling interest. --Lack of Control and Lack of Marketability
\\The Discount for Lack of Control (DLOC)
\be
    \item DLOC = 1 - 1/(1+control premium)
\ee
The Discount for Lack of Marketability (DLOM)
\be
    \item If an interest in a firm cannot be easity sold, DLOM is needed.\
    \item Estimate DLOM:
        \be
            \item Compare the price of restricted shares to the price of the
                publicly traded shares.
            \item Compare the price of pre-IPO shares and post-IPO shares. Issues here:
                post-IPO price are thought to have more certain cash flows and lower risk,
                therefore not purely reflect changes in marketability.
            \item Use the price of put-opiton divided by the stock price, where the put
                used is at the money.
        \ee
    \item TOTAL disount = 1 - [(1-DLOC)(1-DLOM)]
\ee
\subsection{Describe the rold of valuation standards in valuing private companies}
Two stds:
\be
    \item Uniform Standards of Professional Apprasial Practice (USPAP) from US Appraisal Foundation 
    \item International Valuation Standards from International Valuation Standards Committee (IVSC)
\ee
Challenges
\be
    \item Not required to follow these standards
    \item Hard to follow standards
    \item Valuation reports are private. Hard to ensure the compliance of the standards.
\ee

\section{Reading 35: The Term Structure and Interest Rate Dynamics}
\subsection{Describe relationships among spot rates, forward rates, yield to maturity,
expected and realized returns on bonds, and the shape of the yield curve}
\be
    \item Spot Rates: P$_T$=1/(1+S$_T$)$^T$
    \item Forward Rates: annualized interest rate on a loan to be initiated
        at a future period is the forward rate.
    \item Yield to Maturity: included coupon payments, and the spot rate of each
        coupn might be different. YTM is a summary parameter.
    \item Expected and Realized Returns on Bonds: ex-ante holding period return that a 
        bond investor expects to earn.
        \\When: 1. The bond is held to maturity, all payments are mad on time and full, 
        and all coupons are reinvested at the original YTM, the expected return =
        YTM
\ee
\subsection{Describe the forward pricing and forward rate models and calculate forward
and spot prices and rates using those models.}
\be
    \item The Forward Pricing Model: arbitrage-free pricing 
    \item The Forward Rate Model: $[1+S_{j+k}]^{j+k}=(1+S_j)^j[1+f(j,k)]^k$, 
        where $f$ is the forward rate.
\ee
\subsection{Describe how zero-coupon rates (spot rates) may be obtained from the par
curve by bootstrapping.}
Par rate: YTM of a bond trading at par.
\\ Par rate curve: par rates for bonds with different maturities.
\\ Par rate = the coupon rate on the bond.
\subsection{Describe the assumptions concerning the evolution of spot rates in relation 
to forward rates impolicit in active bond portfolio management.}
Relationships between Spot and Forward Rates
\be
    \item Upward-sloping spot curve\ra Forward rate rises as $j$ increases
    \item Upward-sloping spot curve\ra Forward curve will be above the spot curve
\ee
Forward Price Evolution
\be
    \item If future spot rates evolves as forecasted by the forward curve, the forward
        price will remain unchanged.
    \item Spot rate lower than the forward curve\ra forward price up
\ee
\subsection{Describe the strategy of riding the yield curvea}
``Riding the Yield Curve''
\be
    \item Maturity matching: purchasing bonds that have a maturity = investment horizon
    \item "Rolling down the yield curve": If the yield curve remains unchanged, you can buy
        a long-maturity bond, and sell it before its maturity. 
\ee
\subsection{Explain the swap rate curve and why and how market participants use it in
valuation}
The Swap Rate Curve
\be
    \item plain vanilla interest swap: A pays on a fixed rate, B pays on a floating rate.
        \\- the fixed rate\ra swap fixed rate OR swap rate
    \item swap rte curve is preferred as a benchmark interest rate curve.
        \\- reflects the credit risk of commercial banks
        \\- swap mkt is not regulated by government, therefore comparable across
        different countries.
        \\- Has yield quotes at many maturities.
    \item Wholesale banking uses Swap rate; Retail banking uses government bond yield curve
    \item Calculating SFR
        $$
            \sum_{t=1}^T \frac{SFR_T}{(1+S_t)^t}+\frac{100}{(1+S_T)^T}=100
        $$
\ee
\subsection{Calculate and interpret the swap spread for a given maturity}
Swap spread:
\be
    \item Definition: the swap rate exceeds the yield of government bond with the 
        save maturity
    \item swap spread$_t$ = swap rate$_t$ - Treasury yield$_t$
\ee
I-spread:
\be
    \item The amount by which the risky bond exceeds the swap rate for the same 
        maturity. Linear interpolation for missing swap rates.
    \item I-spread reflects compensation for risk.
\ee
Describe the Z-spread
\be
    \item Definition: the spread that, when added to each spot rate on the default-free
        spot curve, makes the present value of a bond's cash flows equal to the bond's
        market price.
    \item zero-volatility: assumption of zero interest rate volatility.
\ee
\subsection{Describe the TED and Libor-OIS spreads}
TED Spread
\be
    \item Definition: T: T-bill, ED: Eurodollar futures contract
        \\The amount by which the interest rate on loans between banks (3-mon LIBOR)
        exceeds the interest rate on short-term US government debt(3-mon T-bills)
    \item Indication of the risk of interbank loans. Captures the risk in the banking system.
\ee
LIBOR-OIS Spread
\be
    \item OIS: Overnight Indexed Swap. Reflects the federal funds rate. Includes minimal credit risk.
    \item LIBOR-OiS spread: LIBOR rate - OIS rate
\ee
\subsection{Explain traditional theories of the term structure of interest rates
and describe the implications of each theory for forward rates and the shape of the
yield curve}
Unbiased Expectation Theory/Pure expectation theory:
\be
    \item Main point: Investor's expectations determine the shape of interest rate
        term structure. 
    \item Example: 
        \\A. Investor hsould earn the same return by investing in a 5-yr bond, or by
        investing in a 3-yr bond and then a 2-yr band.
        \\B. with 3-yr horizon, investors will be indifferent between 3-yr bond or 5-yr bond.
    \item Behind point is: risk neutrality. Investors don't demand a risk-premium for 
        maturity strategies that differ from their investment horizon.
\ee
Local Expectations Theory
\be
    \item Main point: preserves risk-neutrality assumption only for short holding periods.
    \item Not hold.
\ee
Liquidity Preference Theory
\be
    \item Main point: Forward rates reflect investors' expectations + a liquidity premium
        to compensate investors for exposure to interest rate risk.
    \item Implication:  Forward rates  are biased estimates of the market's expectation of
        future rates.
\ee
Segmented Markets Theory
\be
    \item Main Point: The shape of yield curve is determined by the preferences of borrowers
        and lenders -- supply and demand for loans of different maturities.
\ee
Preferred Habitat Theory
\be
    \item Main point: Forward rates represent expected future spot rates plus a premium. The
        premiums are related to supply and demand for funds at various maturities, give incentives
        to investors to shift from their preferred maturity.
\ee
\subsection{Describe modern term structure models and how they are used.}
Equilibrium Term Structure Models: use fundamental economic variables
\be
    \item Cox-Ingersoll-Ross model: interest movements depends on whether people
        choose consumption or investing today
        \be
            \item $ dr=a(b-r)dt+\sigma\sqrt{r}dz $
                \\dr = change in the short-term interest rate
                \\a = speed of mean reversion parameter.
                \\b = long-run value of short-term interest rate
                \\t = time
                \\$\sigma$ = volatility
                \\dz = a small random walk movement
        \ee
    \item The Vasicek Model
        \\$dr = a(b-r)dt + \sigma dz$.
\ee
Arbitrage-free Models: assumes that bonds trading now are correctly priced.
\be
    \item Ho-Lee Model: $dr_t=\theta_t dt + \sigma dz_t$
        \\-$\theta_t=a time-dependent drift term$
        \\-Using market prices to find the drift term
\ee
\subsection{Explain how a bond's exposure to each of the factors driving the yield
curve can be measured and how these exposures can be used to manage yield curve}
Managing Yield Curve Risks: risk to the portfolio value to changes in the yield curve.
\\Sensitivity measures:
\be
    \item Effective Duration: price sensitivity to a small parallel shifts in the yield curve.
    \item Key Rate Duration: the sensitivity of the portfolio value to changes in a single
        par rate. (percentage change in the value in respone to a 100 basis point change
        in the correspoing key rate, all other rates are constant.)
        \\  Example: a portfolio has 3 bond. 1-yr, 5-yr, 25-yr. Key rate durations: $D_1=0.7,
        D_2=3.5,D_{25}=9.5$,
        $\Delta_P/P \approx -D_1\Delta r_1-D_5\Delta r_5 - D_{25}\Delta r_{25}$
    \item Sensitivity to Parallel, Steepness, and Curvature Movements
        \be
            \item Level$\Delta x_L$: parallel increase/decrease
            \item Steepness $\Delta x_S$: Long-term interest rates increase while short-term rates decrease
            \item Curvature $\Delta x_C$: short-term and long-term rates increase, while intermediate
                rates do not change.
        \ee
\ee
\subsection{Explain the maturity structure of yield volatilities and their effect on price
volatility}
Term structure of interest rate volatility: Yield volatility versus maturity.
\\Long-maturity volatility $<$ Short-term volatility
\\Long-maturity volatility\ra uncertainty regarding real economy and inflation. 
\\Short-maturity volatility\ra risks regarding monetary policy.
\end{document}












