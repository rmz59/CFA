\documentclass{article}
\usepackage[top=1.2in, bottom=1in, left=1in, right = 1in]{geometry}
%\usepackage[font=footnotesize,width = 0.8\textwidth]{caption}
\usepackage{titling}
\usepackage{fancyhdr}
\usepackage{float}
\usepackage{graphicx}
\usepackage{subcaption}
\usepackage{indentfirst}
\usepackage{lastpage}
\usepackage{amsmath}
\usepackage{bm}
\usepackage{wrapfig}
\usepackage[inline]{enumitem}   % This allows me to enumerate things inline.
\usepackage{hyperref}

\hypersetup{
    pdftitle={CFA II Notes},
    pdfauthor={Runmin Zhang},
    %pdfsubject={Your subject here},
    %pdfkeywords={keyword1, keyword2},
    bookmarksnumbered=true,     
    bookmarksopen=true,         
    bookmarksopenlevel=1,       
    colorlinks=true,            
    pdfstartview=Fit,           
    pdfpagemode=UseOutlines,    % this is the option you were lookin for
    pdfpagelayout=TwoPageRight
}


\newcommand{\be}{\begin{enumerate}}
\newcommand{\ee}{\end{enumerate}}
\newcommand{\ra}{$\rightarrow$}
\newcommand{\Ra}{$\Rightarrow$}
\setlength\parindent{0pt}
\pagestyle{fancy}
\fancyhf{}
\fancyhead[L]{CFA II Notes}
\fancyhead[R]{Runmin Zhang}
\fancyfoot[R]{\thepage}
\begin{document}
\tableofcontents
\newpage
\section{Reading 9: Correlation and Regressions}
\subsection{Sample covar and sample correlation coefficients}
Sample covariance: $cov_{x,y}=\sum_i \frac{(X_i-\bar X)(Y_i-\bar Y)}{n-1}$
\\Sample correlation coeff: 
$r_{x,y}=\frac{cov_{x,y}}{s_x s_y}$, where $s_x$ is the sample dev of X.
\subsection{Limtations to correlations analysis}
Outliers: The results will be affected by extreme data points.(outliers)\\
Spurious correlation: There might be some non-zero corrlation coeff, but 
acutally they have no correlation at all.\\
Nonlinear relationships: Correlation only describe the linear relastions.
\subsection{Hypothesis: determine if the population 
correlation coefficient is zero}
Two-tailed hypothesis test:
$$
H_0: \rho=0, H_a: \rho \neq 0
$$
Assume that the two populations are {\bf normally} distrubited, then we 
can use t-test:
$$
t=\frac{r\sqrt{n-2}}{1-r^2}
$$:
Reject $H_0$ if  $t>+t_{critical}$ or $t<-t_{critical}$. Here,$r$ is the
sample correlation. Remember, you need to check t-table to find the t-value.
\subsection{Determine dependent/indepedent variables in a linear regression}
{\bf Simple linear regression}: Explain the variation in a dependent variable 
in terms of the variabltion in a single indepedent variable.
{\bf Independent variables} are called explanatory variable, the exogenous 
variable, or the predicting variable. 
{\bf Dependent variable} is also called the explained variable, the endogenous 
variable, or the predicted variable.
\subsection{Assumptions in linear regression and interpret regression coeff.}
\begin{enumerate}
    \item Assumptions of linear regression:
        \begin{enumerate}
            \item Linear relationship must exist.
            \item The indepedent variable is uncorrelated with residuals.
            \item Expected Residual term is value. $E(\epsilon)=0$
            \item variance of the residual term is const. $E(\epsilon_i^2)=
                \sigma_\epsilon^2$. Otherwise, it will be "heteroskedastic"
            \item The residual term is independently distributed. otherwise -- "auto correlation"" 
                $E(\epsilon_i\epsilon_j)=1$    
            \item The residual term is normally distributed.
        \end{enumerate}
    \item Simple Linear Regression Model
        \begin{enumerate}
            \item Model: $Y_i=b_0+b_1X_i+\epsilon_i$, where $i=1...n$, and $Y_i$ is 
     the actual observed data.
            \item The fitted line, the line of best fit
                : $\hat{Y}=\hat{b_0}+\hat{b1}X_i$. Where $\hat{b_0}$
                is the estimated parameter of the model.
            \item How to choose the best fitted line? {\bf Sum of squared errors}
                 is minimum.
                 $$
                    \hat{b_1} = \frac{cov_{x,y}}{sigma_x^2}
                 $$
                 where $X$ is the indepdent variable. $\hat{b_1}$ is 
                 "regression coeffcient".
                 $$
                    \hat{b_0} = \bar Y - \hat{b_1}\bar X
                 $$
                 where $\bar X, \bar Y$ are the mean.
        \end{enumerate}
    \item Interpreting a regression coefficient: Similar to basic ideas of 
        "slope". Keep in mind: any conclusion regarding this parameter needs 
        the statistical significance of the slope coefficient.
\end{enumerate}
\subsection{Standard error of estimate, the coeff. of determination and a 
confidence interval for a regression coefficient.}
\begin{enumerate}
    \item Standard error of estimate (SEE): Standard deviation between $Y_{estimate}$
        and $Y_{actual}$. - Smaller: better
    \item Coefficient of Determination ($R^2$)
        The percentage of the total variance in the dependent variable that is 
        predictable from the indepedent variable. - One indepdent variable: $R^2=r^2$,
        where $r^2$ is the square of correlation coefficient.
    \item Regression Coefficient confidence interval
        \begin{enumerate}
            \item Hypothesis: $H_0: b_1=0 \Leftrightarrow H_a: b_1\neq 0$ 
            \item Confidence interval:
                $\hat{b_1}-(t_c s_{\hat{b_1}})<b_1<\hat{b_1}+(t_c s_{\hat{b_1}})$ 
            $s_{\hat{b}_1}$ is the standard error of the regression coeffi.  
        \end{enumerate}
\end{enumerate}
\subsection{Hypothesis: Determine if $\hat{b}_1=b_1$}
\begin{enumerate}
    \item t-test statistic: $t_{b_1}=\frac{\hat{b}_1 - b_1}{s_{\hat{b}_1}}$
    \item Reject: if $t>+t_{critical}$ or $t<-t_{critical}$
\end{enumerate}
\subsection{Calculate the predicted value for the depedent variable}
If an estimated regression model is known, $\hat{Y}=\hat{b}_0+\hat{b}_1 X_p$
\subsection{Calculate and interpret a confidence interval for the predicted 
value of the depedent variable}
\begin{enumerate}
    \item Eq: $\hat{Y}\pm(t_c s_f)$, where $s_f$ is the {\color{red}std error of the forecast.}
    \item $s_f^2=SEE^2 \left[1+\frac{1}{n}+\frac{(X-\bar X)^2}{(n-1)s_x^2} \right]$
        \begin{enumerate}
            \item $SEE^2=$ variance of the residuals
            \item $s_x^2=$ variance of the indepdent variable
            \item $X =$ value of the independent variable where the forecast was
                made.
        \end{enumerate}
\end{enumerate}
\subsection{ANOVA in regression. Interpret results, and calculate F-statistic}
\begin{enumerate}
    \item Analysis of variance (ANOVA) is used to analyze the total variability 
        of the depedent variable. 
        \begin{enumerate}
            \item Total sum of squares(SST): $SST=\sum_{i=1}^n(Y_i-\bar Y)^2$
                \\SST is the total variation in the depedent variable.
                $Variance=SST/(n-1)$
            \item Regression sum of squares(RSS): $RSS=\sum_{i=1}^n(\hat{Y}_i-\bar Y)^2$
                \\RSS is the explained variation.
            \item Sum of squared errors(SSE): $SSE=\sum_{i=1}^n(Y_i-\hat{Y}_i)^2$
                \\SSE is the unexplained variation.
            \item {\color{red}$SST = RSS+SSE$ I cannot get this equation yet}
                You need to know how to use these squares.
            \item Degree of freedom: i) Regression(Explained): $k=1$, since we only
                estimate one parameters.
                ii) Error(Unexplained) $df=n-k-1=n-2$
                iii) Total variation $df=n-1$
        \end{enumerate}
    \item Calculating $R^2$ and {\bf SEE}
        \begin{enumerate}
            \item $R^2=explained variation/total varn=RSS/SST$
            \item $\bf{SEE}=\sqrt\frac{SSE}{n-2}$ {\bf SEE} is the std deviation of the
                regression error terms.
        \end{enumerate}
    \item The F-Statistic: used to explain whether {\it at least one} indepdent parameter
        can significanly explain the dependent parameter.  
        \begin{enumerate}
            \item F-statistic eq: $F=\frac{MSR}{MSE}=\frac{RSS/k}{SSE/n-k-1}$
                 where $MSR=$ mean regression sum of squares.
                       $MSE=$ mean squared errors. 
                       Note: \color{red}{One tailed test!}
        \end{enumerate}
    \item F-statistic with one independent variable.
        \begin{enumerate}
            \item Hypothesis: $H_0: b1=0 \Leftrightarrow H_a: b1\ne 0$
            \item degree of freedom: $df_{rss}=k=1,df_{sse}=n-k-1$
            \item Decision rule: reject $H_0$ if $F>F_c$
        \end{enumerate}
\end{enumerate}
\subsection{Limitations of regression analysis}
    \begin{enumerate}
        \item Parameter instability: the estimation eq may not be useful for other times.
        \item Limited usefulness: other participants may also use the same eq.
        \item Assumptions does not hold: i) Heteroskedastic, i.e., non-const
            variance of the error terms. ii) autocorrelation, i.e., error terms
            are not independent.
    \end{enumerate}

\section{Reading 10: Multiple Regression and Issues in Regression Analysis}
Some basic ides
    \begin{enumerate}
        \item Model: 
            $Y_i=b_0+b_1 X_{1i}+b_2 X_{2i}+...+b_k X_{ki}+\epsilon_i$
        \item Multiple regression methodology estimates the intercept and 
            slope coefficients so that $\sum_i^n \epsilon_i^2$ is minimized.
    \end{enumerate}
\subsection{Interpret estimated regression coefficients and their p-values.}
They are just simple linear functions with multiple parameters. Ignore.
\subsection{Formulate a null/alternative hypothesis, do correspoding calculations}
\begin{enumerate}
    \item Hypothesis Testing of Regression coefficient. (Multi-parameters). \\
        Use t-statistics to determine if one parameter significantly contribute
        to the model.
        $$
            t=\frac{\hat{b}_j-b_j}{s_{\hat{b}_j}}, df=n-k-1
        $$
        where $k$ is the number of regression coefficients, and $1$ corresponds to
        the intercept term, and $s_{\hat{b}_j}$ is the coefficient standard error
        of $b_j$
    \item Determining statistical significance.
        \\ ``testing statistical significance" $\Rightarrow H_0: b_j=0, H_a: b_j\ne 0$
    \item Interpreting p-values.
        \begin{enumerate}
            \item Def: p-value is {\bf the smallest level of significance for 
                which the null hypothesis can be rejected.}
                \ If the p-value is less than significance level, the null 
        \end{enumerate}
    \item Other Tests of the Regression Coefficients: $H_0: a=$some value
\end{enumerate}
\subsection{Calculate and Interpret a confidence interval for the population 
value of a regression coefficient or a predicted value for the depedent variable
if an estimated regression model.}
\begin{enumerate}
    \item Confidence intervals for a regress. coeff.: $\hat{b}_j\pm(t_c\times s_{\hat{b}_j})$
    \item predicting the depedent variable: 
        $\hat{Y}_i=\hat{b}_0+\hat{b}_1 \hat{X}_{1i}+...+\hat{b}_k \hat{X}_{ki}$
        \\Even if you may conclude that some $b_i$ are not statistally significantly, you cannot
        treat them as $0$ and keep other parameters unchanged. You should use the
        original model, or you can throw $\hat{b}_k$ away and make a new regression model.
\end{enumerate}
\subsection{Assumptions of a multiple regresssion model}
\begin{enumerate}
    \item Linear relationships exist.
    \item The independent variables are not random, and there is no exact linear relation
        between indepdent variables.
    \item $E[\epsilon|X_1,...,X_k]=0$
    \item Variance of $\epsilon=0$, i.e. $E[\epsilon_i]=0$
    \item $E(\epsilon_i\epsilon_j)=0$
    \item $\epsilon$ is normally distributed.
\end{enumerate}
\subsection{Calculate and interpret F-statistic}
F-test: whether at least {\bf one} of the indepedent variables explains a significant
portion of the variation of the depedent variable. F test is a one-tail test.
\begin{enumerate}
    \item $H_0: b1=b2=b3=0 vs H_a:$ at least one $b_j\ne 0$
    \item $F=\frac{MSR}{MSE}=\frac{RSS/k}{SSE/n-k-1}$
    \item Degree of freedom: $df_{numerator}=k, df_{denominator}=n-k-1$
    \item Rules: reject $H_0$ if $F(test-statistic)>F_c(critical value)$
\end{enumerate}
\subsection{Distinguish between $R^2$ and adjusted $R^2$}
\begin{enumerate}
    \item coefficient of determination $R^2$: used to test if a group of indepedent
        variable can explain the depedent variable: 
        \\$R^2=\frac{total variation - unexplained variation}{total variation}
        =\frac{SST-SSE}{SST}=\frac{RSS}{SST}$
        \\{\bf Multiple R} = $\sqrt{R^2}$
    \item Adjusted $R^2$
        \begin{enumerate}
            \item Note: $R^2$: {\color{red}{Overestimating}}: will increase as variables are added to the model.
        Even the marginal contribution of new variables are not statistically significant.
            \item Introduce $R_a^2$: $R_a^2=1-\left[\left(\frac{n-1}{n-k-1}\right)\right](1-R^2)$
        \end{enumerate}
\end{enumerate}
\subsection{Evaluate the quality of a regression model by analyzing the ouput of the equation/ANOVA table}
\begin{enumerate}
    \item ANOVA Tables, some important quantities
        \begin{enumerate}
            \item $R^2=\frac{RSS}{SST}$
            \item $F=\frac{MSR}{MSE}$ with $k$ and $n-k-1$ df
            \item Standard error of estimate:$SEE=\sqrt{MSE}$
        \end{enumerate}
\end{enumerate}
\subsection{Formulate a multiple regression with dummpy variables to represent qualitative factors}
\begin{enumerate}
    \item Def: Some value is quite qualitative. Using dummy values like 0 or 1 
        to describe their impacts.
    \item Note: Pay attention to \# of dummy variables. If $n$ classes, we must use
        $n-1$ dummy variables.
    \item Interpreting the coefficients in a dummy variable regression. We can use F-statistics to
        test a group of parameters, or use t-test to test the individual slope coefficients.
    \item Example of Regression application with dummy variables. See Notes directly.
\end{enumerate}
\subsection{Why multiple regression isn't as easy as it looks?}
Pay attention to the assumptions that have been used. Violations like::
\begin{enumerate}
    \item Heteroskedasticity
    \item Serial correlation (auto-correlation)
    \item Multicollinarity
\end{enumerate}
Any violations on the assumptions will impact the estimation of SEE, and finaly 
change the t-statistic and F-statistic, and change the conclustion of the hypotesis
test.
\subsection{Types of Heteroskedasticity, how heteroskedasiticity and serial correlation affect inference}
\begin{enumerate}
    \item What is Heteroskedasticity?
        \\ {\bf Corresponding assumptions: Variance of the residuals is constant across observations. -- Homoskedasticity} 
        Heteroskedasity means the variance of the residuals is not equal.
        \begin{enumerate}
            \item Unconditional heter: Not related to the level of the indepedent
                variables. Will not systematiclly increase with changes in the value
                of the indepedent variables. {\color{red}Usually will not casue major problems.}
            \item Conditional heter: Related to the level of the independent variables. Eg:
                Conditional heter exists if the variance of the residuals increase with 
                the value of the independent variables increases. {\color{red}Will cause big problems.}
        \end{enumerate}
    \item Effect of Heteroskedasiticity on Regression Analysis
        \begin{enumerate}
            \item Unreliable standard errors.
            \item The coefficient estimates aren't affected.
            \item Will change the t-statistic, and will change the conclusion.
            \item Unreliable F-test
        \end{enumerate}
    \item Detect Heteroskedasicity
        \begin{enumerate}
            \item Scatter plot
            \item Breusch-pagan test: $BP test = n\times R^2_{resid}$ with $df=k$. 
                where $n=$the number of observations, $R^2_{resid}$=$R^2$ from a second
                regression of the squared residuals from the first regression. $k=$the number
                of independent variables. If $R^2$ or BP-test are too large, something is wrong.
        \end{enumerate}
    \item Correcting Heteroskedasticity
        \be
        \item Calculate robust sndard errors(White corrected std errors.). Use them for t-test.
            \item Generalized least squares. 
        \ee
    \item What is serial correlations?
        \be
            \item Def: auto-correlation, in which the residual terms are correlated.
                Common problem with time series data.
                \be
                    \item Positive serial correlation: a postive error in one time period will
                        increase the posibility to observe a positive one next time.
                    \item Negative serial correlation: Just opposite.
                \ee
            \item Effect: positive serial correlation will get small coefficient std errors. 
                Thus, too large t-statistics. therefore, too many Type I errors: reject the null
                hypothesis $H_0$ while it's actually true. 
            \item Detection:
                \be
                    \item Residual plots
                    \item Durbin-Watson statistics: 
                        $$
                        \color{red}DW=\frac{\sum_{t=2}^T (\hat{\varepsilon}_t-\hat{\epsilon}_{t-1})^2}{\sum_{t=1}^T \hat{\epsilon}^2}
                        $$
                        For large samples, $DW\approx 2(1-r)$, where $r$ is the 
                        correlation coefficient between residuals from one period and thoese from the previous period.
                        \\Results: 
                        \be
                            \item $DW=2\Rightarrow$ Homoskedasitic and not serially correlated.
                            \item $DW<2\Rightarrow$ Postively serially correlated.
                            \item $DW>2\Rightarrow$ Negatively serially correlated.

                        \ee
                        Formulated hypothesis with DW-table, upper and lower critical values
                        \be
                            \item Hypothesis: $H_0:$ the regression has {\bf no}
                                positive serial correlation.
                            \item $DW<d_l$: positive serially correlated. Reject null.
                            \item $d_l<DW<d_u$: inconclusive results.
                            \item $DW>d_u$: {\bf There is no evidence that are positive correlated.} 
                        \ee
                \ee
            \item Correcting serial correlation:
            \be
                \item Adjust the coefficient std errors. {\bf recommended.} Using Hansen method.
                    \be
                        \item Serial correlation only: Hansen method. 
                        \item Heteroskedasticity only: White-corrected stand errors.
                        \item Both: Hans methods.
                    \ee
                \item Imporoe the specification of the model.
            \ee
    \ee
\end{enumerate}
\subsection{Multicolinearity and its cause and effects in regression analysis}
Multicollinearity: Indepedent variables or linear combinations of independent variables are highly
correlated.
\begin{enumerate}
    \item Effect of Multicollinearity on Regression Analysis: Will increase the std errors of 
        the slope coefficients. {\color{red} Type II Error:
        A variable is significant, while we conclude it's not.} 
    \item Detecting: Common situation: $t-statistic$ is not significant while $F-test$ is significant.
        This tells us the indepedent variables are highly correlated.
        \\ A simple rule works if there are 2 indepedent variables: when the absolute value of the sample correlation betewen any two 
        indepedent variables in the regression is greater than 0.7.
    \item Correcting: omit one or more of the correlated indepedent variables. 
        THe problame is that it's hard to find the variables that result in the multicolinearity.
\end{enumerate}
\subsection{Model misspecification}
\begin{enumerate}
    \item Defination of {\bf Regression model of specification}: decide which independent variables
        to be included in the model.
    \item Types of misspecification
        \be
            \item The functional form can be misspecified:
                important variables are ommited;variables should be transformed; data is improperly pooled.
            \item Explanatory variables are correlated with error term in time series model:
                A lagged dependent variable is used as an independent variable; a function of the dependent
                variable is used as an independent variable(forecasting the past);
                independent variables are measured with error.
            \item Other time-series misspecification.
        \ee
\end{enumerate}
\subsection{Models with qualitive dependent variables}
Include qualitative dependent variables, like default, bankcrupcy. Cannot use an 
ordinary regression model. Should use other models like {\bf probit and logit models}
or {\bf discriminant models}.
\be
\item Probit: normal distribution, give probability.
\item Logistic: logistic distribution. 
\item Discriminant: result in an overall score or ranking.
\ee
% End Reading 10


%% Reading 11
\section{Reading 11: Time-Series Analysis}
\subsection{Calculate/evaluate the predicted trend value for a time series given 
the estimated trend coefficients}
\begin{enumerate}
    \item Linear Trend Model and Log-linear Trend
        \be
            \item Definition: $y_t=b_0+b_1(t)+\epsilon_t$ Note: $t$ is just time.
            \item Coefficients is determined by OLS. Ordinary least squared regression.
                \\$\hat{y}=\hat{b}_0+\hat{b}_1$
            \item Log-linear Trend Models
            \item Model: $y_t=\exp{b_0+b_1(t)}\Rightarrow \ln{y_t}=b_0+b_1(t)$
        \ee
\end{enumerate}
\subsection{Factors that determine whether a linear or a log-linear model trend 
should be used}
\be
    \item Factors that determine which model is best: plot data.
    \item Limitaions of trend models: 
        \be
            \item residuals are uncorrelated with each other.
        Otherwise, it will cause auto correlation and we should not use the trend model.
            \item For log-linear model, it is not suitable for cases with serial correlations (autocorrelation).
            \item Detect auto correlation: Durbin Watson statistic. $DW=2.0\Rightarrow$ No auto correlation.
        \ee
\ee
\subsection{Autoregressive model, requirements for covariance statinoary}
\be
    \item Autoregressive model:
        \be
            \item Model: $x_t=b_0+b_1 x_{t-1}+\varepsilon_t$
            \item Statistical inferences bse on ordinary least squares estimates doesn't
                apply unless the time series is {\bf covariance stationary}.
            \item Conditions for covariance stationary
                \be
                    \item Constant and finite expected value.
                    \item Constant and finite variance.
                    \item Constant and finite covariance between values at any given lag.
                \ee
        \ee
\ee
\subsection{An autogressive model of order $p$}
\be
    \item Model(order $p$): $x_t=b_0+b_1x_{t-1}+b_2x_{t_2}+...+b_px_{t-p}+\varepsilon_t$ 
    \item Forecasting with an autoregressive model: 
        \be
            \item One-period-ahead forecast for $AR(1)$: $\hat{x}_{t+1}=\hat{b}_0+\hat{b}_1 x_t$
            \item Two-period-ahead forecast for $AR(1)$: $\hat{x}_{t+2}=\hat{b}_0+\hat{b}_1 \hat{x}_{t+1}$
        \ee
\ee
\subsection{How the residuals can be used to test the autogressive model}
\be
    \item The residual should have no {\it serial correlation} if an AR model is correct.
    \item Steps
        \be
            \item Estimate: Start with AR(1)
            \item Calculate: the autocorrelations of he model residuals
            \item Test: whether the autoccorelations are signficantly different from 0.
                \\The standard error is $\frac{1}{\sqrt{T}}$ for $T$ observations. The t-test for each
                observation is $t=\frac{\rho_{\epsilon_t,epsilon_{t-k}}}{1/sqrt{T}}$, with $T-2$ df. 
        \ee
\ee
\subsection{Mean reversion and a mean-reverting level}
\be
    \item Mean reversion: The time series tends to move toward its mean.
    \item Mean-reverting level: $\hat{x}_{t+1}=x_{t}$, where $\hat{x}_t$ is the predicted value. 
    \item All covariance stationary time series has finite mean-reverting level. 
\ee
\subsection{Contrast in-sample and out-of-sample forecasts and the forecasting accuracy of 
different time-series models based on the root mean squared error criterion.}
\be
    \item in-sample, out-of-sample: determined by if the predicted data is in the range of 
        the observations.
    \item RMSE, root mean squared error: used to compare the accurancy. If the accurancy of 
        out-of-sample is better, you should use it for future applications
\ee
\subsection{Explain the instability of coefficients of time-series models}
\be
    \item Instability or nonstationarity. Due to the dynamic econimic conditions, 
        model coefficients will change a lot from period to period. 
    \item Shorter time series are more stable, but longer time series are more reliable.
\ee
\subsection{Random walk processes and their comparisons between covariance stationary processes}
\be
    \item Random walk: $x_t=x_{t-1}+\varepsilon_t$
    \be
        \item $E(\varepsilon_t)=0$: The expected value of each error is zero.
        \item $E(\varepsilon_t^2)=0$: The variance of the error terms is constant.
        \item $E(\varepsilon_i,\varepsilon_j)=0$: There is no serial correlation in the error terms.
    \ee
    \item Random walk with a Drift: $x_t=b_0+b_1 x_{t-1}+\varepsilon_t$, where $b_1$=0
    \item A random walk or a random walk with a drift have no finte mean-reverting level.
        Since $b_1=1,\frac{b_0}{1-b_1}=\frac{b_0}{0}$. Therefore, they are not covariance stationary.
    \item $b_1=1$, they exhibit a unit root. Thus, {\bf the least square regression that been used in 
        AR(1) will not work unless we transfrom the data}.
\ee
\subsection{Things about unit roots: when they will occur, how to test them, how to transform data to apply AR}
\be
    \item Unit root testing for nonstationarity:
        \be
            \item run an AR model and check autocorrelations
            \item perform Dickey Fuller test.
                \be
                    \item Transform: $x_t=b_0+b_1x_1+\varepsilon\Rightarrow x_t-x_{t-1}=b_0+(b_1-1)x_{t-1}+\varepsilon$
                    \item Direct test if $b_1-1=0$ using a modified t-test.    
                \ee
        \ee
    \item First differencing
        \be
            \item For a random walk, transform the data $y_t=x_t-x_{t-1}\Rightarrow y_t=\varepsilon_t$
                then start to use an AR model $y=b_0+b_1 y_{t-1}+\varepsilon$, 
                where $b_0=b_1=0$
            \item $y$ is covariance stationary.
        \ee
\ee
\subsection{How to test and correct for seasonality in a time-series model, and 
calculate and interpret a forecasted value using an AR model with a sesonal lag.}
\be
    \item Detect: special autocorrelation exists for some seasonal lags.
    \item Correction: Add an additional seasonal lag term.
\ee
\subsection{Explain autogressive conditional heteroskedasticity (ARCH) and describe
how ARCH models can be applied to predict the variance of a time series}
\be
    \item ARCH: the variance of the residuals in one period is dependent on the variance of
        the residuals in a previous period.
    \item Using ARCH models:
        \\Example $ARCH(1)$: $\hat{\varepsilon}_t^2=a_0+a_1\hat{\varepsilon}_{t-1}+\mu_t$
        if $a_1$ is significantly different from zero. $\hat{\varepsilon}_t^2$ is the squared residuals.
        \\Note: Things like generalized least squares should me used to 
                correct heteroskedasticity. otherwise, the std errors of the 
                coefficients will be wrong, leading to invalid conclusions.
    \item Predicting the variance of a time series: using ARCH model to predict the variance of 

        future periods: $\hat{\sigma}^2_{t+1}=\hat{a}_0+\hat{a}_1\hat{\varepsilon}_t^2$
\ee
\subsection{Explain How time-series variables should be analyzed for nonstationarity and/or cointegration before
use ain a linear regression}
\be
    \item Cointegration: 
        \be
            \item Two time series are economically linked or follow the 
        same trend and that relationship is not expected to change. -- Error terms 
        from regressing one on the other is covariance stationary and the t-test are reliable.
            \item How to test conintegration: regress $y_t$ on $x_t$ 
                 $y_t=b_0+b_1x_t+\varepsilon$, $y_t, x_t$ are two different time series.
                 Then do a unit root test using the Dickey Fuller test with critical t-values
                 calculated by Engle and Granger. 
                 \\If "A unit root" is rejected $\Rightarrow$ covariance stationary, cointegrated.
        \ee
\ee


\section{Reading 12: Probabilistic Approaches: Scenario Analysis, Decision Trees, and Simulations}
\subsection{Describe steps in a simulatino, Explain three ways to define the probability 
        distributions for a simulation's variable, and describe how to treat correlation accross
    variables in a simulation.}
        \be
            \item Steps in simulations:
                \be
                    \item Determine the probabilistic variables
                    \item Define probability distributions for these variables
                        \be
                            \item Option 1: Historical data
                            \item Option 2: Cross-sectional data: estimate the variable from similar 
                                companies.
                            \item Option 3: Pick a distribution and estimate the parameters.
                        \ee
                    \item Checkk for correlations among variables: Use historical data to deterine
                        whether any ststematically related. Strong relations$\Rightarrow$ 1) Allow only
                        one of the variables can be removed. Or 2) Build the rules of correlations
                        into the simulation.
                    \item Run the simulation.
                \ee

        \ee
\subsection{Describe advantages of using simulations in decision making}
\be
    \item Advantages: 1) Better input quality   2) Provides a distribution of expected value
        rather than a point estimate.
\ee
\subsection{Describe some common constraints introduced into simulations}
\be
    \item Constraints: specific limits imposed by users of simulations.
    \item Types of constraints
        \be
            \item Book value constraints:
                \be
                    \item Regulatory capital requirements: minimum capital requirements
                    \item Negative equity
                \ee
            \item Earnings and cash flow constraints: might be imposed to meet analyst expectations
            \item Market value constraints
        \ee
\ee
\subsection{Describe issues in using simulations in risk assessment}
\be
    \item Limitations of using simulations
        \be
            \item Input quality: garbage in, garbage out
            \item Inappropriate statistical distributions
            \item Non-statinoary distributions: parameters will change
            \item Dynamic correlations: correlations between input variables will change.
        \ee
    \item Risk-adjusted value: cash flows from simulations are not risk-adjusted. 
        SHOULD NOT be discounted at risk-free rate.
\ee
\subsection{Compare scenario analysis, decision trees, and simulations}
\be
    \item Scenario analysis: computes the value of an investment under some specific cases.
        Total probability is less than 1.
    \item Decision trees: good when risk is discrete and sequential. Sum of probability is 1
\ee

\section{Reading 13: Currency Exchange Rates: Determination and Forecasting}
\subsection{Calculate and interpret the bid-ask spread}
\be
    \item Exchange rates
        \be
            \item Important things: exchange rate, spot exchange rate, forward exchange rate.
            \item Bid/offer(ask) rates:
                //Bid: The price that bank will buy. Offer: The price that bank will sell.
            \item Foregin Exchange Spread. Unit: "1 pip"$=1/10000=0.0001$. Spread depend on:
                \be
                    \item Spread in the interbank market. (Currencies, time, market volatility)
                    \item Size of transaction.
                    \item Relationship between the dealer and client.
                \ee
        \ee
\ee
\subsection{Identify a triangular arbitrage opportunity and calculate its profit}
\be
    \item Example: USD/AUD. USD is the price currency, and AUD is the base currency.
        \be
            \item Buy the base currency at the ask $\Rightarrow$ Sell the price currency at the ask
            \item Sell the base currency at the bid $\Rightarrow$ Buy the price currency at the bid
        \ee
    \item For investors,Rule: {\it up-the-bid-and-multiply, down-the-ask-and-divide}
        \be
            \item Convert USD into AUD: going down the quote -- from USD on top 
                to AUD on bottom. Use the ask price for the quote.
            \item Convert AUD into USD: similar. But from bottom to top.
        \ee
    \item Cross Rate: The exchange rate between two currencies with the help by 
        a common third currency.
    \item Cross Rate with bid-ask spreads.
        \be
            \item Rule 1:
            $$
            \left(\frac{A}{C}\right)_{bid}=\left(\frac{A}{B}\right)_{bid}
            \times \left(\frac{B}{C}\right)_{bid};
            \left(\frac{A}{C}\right)_{offer}=\left(\frac{A}{B}\right)_{offer}
            \times \left(\frac{B}{C}\right)_{offer}
            $$
            \item Rule 2:
            $$
            \left(\frac{B}{C}\right)_{bid}=\frac{1}{\left(\frac{C}{B}\right)_{offer}};
            \left(\frac{B}{C}\right)_{offer}=\frac{1}{\left(\frac{C}{B}\right)_{bid}}
            $$
        \ee
    \item Triangular Arbitrage: If the dealer's quote is different from the cross
        rate, arbitrage opportunities may exist. Check it with Notes.
\ee
\subsection{Distinguish between spot and forward rates and calculate the forward
premium/discont for a given currency}
\be
    \item Forward premium relative to a second currency: 
        Forward price > Spot price. Forward premium = $F-S_0$
    \item Calculate the market-to-market value of a forward contract
            $$
                V_T=(FP_T-FP)(contract size)
            $$
            where:
            \be
                \item $V_T$ = value of the forward contract at time $T$, denominated
                    in price currency
                \item $T$ = maturity of the forward contract
                \item $FP$ = forward price locked in at inception to buy base currency
                \item $FP_T$ = forward price to {\bf sell} the same currency at time $T$
            \ee
    \item Value prior to expiration.
    $$
    V_t=\frac{(FP_t-FP)contract size}{1+R(\frac{days}{360})}
    $$
    where 
    \be
        \item $V_t$ is the value of the forward price
        \item $FP_t$: forward price at time $t$
        \item $days$ number of days remaining
        \item $R$ interest rate
    \ee
\ee
\subsection{Explaining international parity relations (covered and uncovered 
interest rate parity, purchasing power parity, and the international Fisher effect)}
\be
    \item Covered interest rate parity: ``Covered" means bound by arbitrage. 
        Investor should earn the same return using either currency.
        $$
            F=\frac{1+R_A(\frac{days}{360})}{1+R_B(\frac{days}{360})}S_0
        $$
    \item Uncovered interest rate parity: Forward currency contract is unavailable, 
        which makes the interest rate not bound by arbitrage.
        For a quote A/B, the base currency is expected to apprecitate
        $$
            E(\%\Delta_S)_(A/B)=R_A-R_B
        $$
        Uncovered interest rate parity can only {\bf forcast} the future spot exchange rate.
    \item Comparing covered and uncovered interest parity:
        \be
            \item Covered interest parity $\Leftrightarrow$ No-arbitrage forward rate
            \item Uncovered interest parity $\Rightarrow$ {\bf Expected} future spot rate
        \ee
    \item International Fisher Relation
        \be
            \item $R_{nominal}=R_{real}+E(inflation)$
            \item Under real interest rate parity, the real interest rate are 
                assumed to converge across different markets.
                $$
                    R_{nominal A}-R{nominal B}=E(inflation_A)-E(inflaction_B)
                $$
        \ee
    \item Purchasing Power Parity: Assumed by one price law.
        \be
            \item Absolute purchasing power parity: The average price of a basket
                of consumption goods.
                $$
                    S(A/B)=CPI(A)/CPI(B)
                $$
                May not hold due to different weights of consumptions.
            \item Relative Purchasing Power Parity: Changes in exchange rates 
                should exactly offset the price effects of any inflation differencital b
                between the two contries.
                $$
                \%\Delta S(A/B)=Inflation_A-Inflation_B=change in spot price(A/B)
                $$
                Not always held in short run.
            \item Ex-Ante Version of Purchasing Power Parity: Similar to relative PPP,
                but Ex-Ante uses expected inflation instead of actuall inflation.
        \ee
\ee
\subsection{Describe the relations among the international parity conditions}
See Notes Page 263, Vol. 2.
\subsection{Evaluate the use of the current spot rate, the forward rate, purchasing
parity and uncovered interest parity to forecast future spot exchange rates}
\be
    \item Real Exchange Rate = $S_t\left[\frac{CPI_B}{CPI_A} \right]$, $S_t$ is 
        the spot rate at time $t$ given as A/B
\ee
\subsection{Explain how flows in the balance of payment accounts affect currency
exchange rates}
\be
    \item Balance of Payments: accounting method to track transactions between a
        country and its international trading partners.
        \be
            \item Incluing government, consumer, and business transactions.
            \item current account + financial account + offcial reserve account = 0
            \item
                \be
                    \item Current account: Exchanges of goods/services, exchanges of
                        investment income and unilateral transfers like gifts.
                        \be
                            \item Surplus: we sell more to other countries, buy less from them
                            \item Deficit: we buy more from the rest, sell less to them
                        \ee
                    \item Financial account/Capital account: Flows of funds for debt
                        and equity investment into/out of a country. Surplus: Money is
                        flowing into the country.
                    \item Offcial resreve: thoes made from the reserves held by the government.
                        Normally doesn't change from year to year.
                \ee
        \ee
    \item Influence of BOP on Exchange Rates
        \be
            \item Current Account
                \be
                    \item Flow mechanism
                        \be
                            \item Deficit: increase the supply of that currency in the market. 
                                Because exporters to our countries need to convert
                                their revenue to their own currency. $\Rightarrow$ Down on the exchange
                                value.
                            \item Depreciation of the currency may rebalance the 
                                current account. Depending on {\bf The initial deficit, the influence
                                of exchange rates on import/export prices, price elasiticity of traded goods.}
                                See Notes P265 for details.
                        \ee
                    \item Portfolio Composion mechanism. Countries with current account surpluses
                        usually have capital account deficits, which typically take the form of investments
                        in countries with current account deficits. As a result of these flows of capital, investor
                        countries may find their portfolios' composition being dominated by few investee currencies. 
                        When investor countries decide to rebalance their investment portfolios, it can have a significant negative
                        impact on the value of those investee country currencies.
                    \item Debt sustainability mechanism: Current account deficit may be balanced
                        by borrowing money from other countries. If the debt too high, lenders may question
                        the security, leading to the deprecitaion of the borrower's currency.
                \ee
            \item Capital Account Influences: Money flow in$\Rightarrow$Demand for 
                my country's currency increases$\Rightarrow$Appreciation. 
                \be
                    \item Good: can help to overcome a shortage of internal savings
                    \item Bad: Too much money can be problematic for emerging markets.
                        \be
                            \item Excessive appreciation of the domestic currency
                            \item Financial asset, real estate bubbles
                            \item Increase in external debt
                            \item Excessive consumption in the domestic market funded by credit
                        \ee
                \ee

        \ee
    \item real exchange rate (A/B) = equilibrium real exchange rate (A/B) + (real interest rate$_B$ - real interest rate$_A$)
        -(risk premium$_B$ - risk premium$_A$) \\
        This eqution is not precise. We cannot use it to calculate the rate.
    \item Taylor Rule
        $$
            R=r_n+\pi+\alpha(\pi-\pi^\ast)+\beta(y-y^\ast)
        $$
        \be
            \item R = Central bank policy rate implied by the Taylor Rule
            \item $r_n$ = Neutral {\bf real} policy interest rate
            \item $\pi$ = Current inflation rate
            \item $\pi^\ast$ = Central bank's target inflation rate
            \item $y$ = log of current level of output
            \item $y^\ast$ = log of central bank's target (sustainable) output
            \item $\alpha, \beta$ = policy response coefficients. (suggested value: 0.5 for both)
        \ee
        $$
        Real interest rate = r = R-\pi=r_n+\alpha(\pi-\pi^\ast)+\beta(y-y^\ast)
        $$
        Substitute the real interest rate equation, we have\\
        Real exchange rate (A/B) = equilibrium real exchange rate(A/B) + differenc
        in neutral real policy interest rate(B-A)+$\alpha$[difference in inflation gap (B-A)]+
        $\beta$[difference in output gap(B-A)]-(risk premium$_B$-risk premium$_A$)
        //Where: Inflation gap = current inflation - target inflation, 
        Output gap = current output - target output
\ee
\subsection{Explain approaches to assessing the long-run fair value of an exchange rate}
\be
    \item The ex-ante version of relative PPP holds $\Rightarrow$ The real exchange rates constant. 
        However, relative PPP does not necessarily hold over the short term. Over long
        term, PPP holds, and the real rate will be near its equilibrium level.
    \item IMF asseses long-term equilibrium real exchanges rate based on
        \be
            \item Macroeconomic balance approach: if the Ex rates need to be adjusted to 
                equalize the expected current account imbalance and the sustainable current
                account imbalance.
            \item External sustainability approach. How rates need to be adjust to force
                a country's external debt relative to GDP towards its sustainable level.
            \item Reduced-form econometric model approach. 
        \ee
\ee
\subsection{Describe the carry trade and its relation to uncovered interest rate
parity and calculate the profit from a carry trade.}
\be
    \item FX carry trade: Invest in a higher yielding funding with the funds borrowed in
        a lower yielding currency. This is due to the uncovered interest rate parity may not
        hold.
    \item Risk of the Carry Trade
        \be
            \item The exchange rate may change abruptly.
            \item The return distribution is not normal. Negative skewness and 
                excess kurtosis (fat tails). $\Rightarrow$ High probabilithy of large loss
        \ee
    \item Risk Management in Carry Trades
        \be
            \item Volatility filter: if volatitlity > certain threshold, close the carry trade.
            \item Valuation filter: valuation band for each currency based on PPP. If the value
                of a currency falls below the band, we will increase its ratio.
        \ee
\ee
\subsection{Describe the Mundell-Fleming model, the monetary approach and the asset
market approach to exchange rate determination.}
\subsection{Forecast the direction of the expected change in an exchange rate based on
balance of payment, Mundell-Fleming, monetary, and asset market approaches to exchange rate determination.}
\subsection{Explain the potential effects of monetary and fiscal policy on exchange rates.}
\be
    \item Mundell-Fleming Model: evaluate the impact of monetary and fiscal policies on interest rates, 
        and therefore on exchange rates.
    \item Flexible Exchange Rate Regimes: rate are determined by markets.
        \be
            \item High Capital Mobility: Expansionalry M and F are likely to have
                opposite effects. Expansionary M will reduce the interest rate, 
                reduce the inflow of  capital investment, reduce the demand for 
                domestic money, depprecation.
            \item Low Capital Mobility: Uncertain
            \item Summary:\\
                \begin{tabular}{c c c}
                \hline
                Monetary/Fiscal & High Capital Mobility & Low Capital Mobility \\
                \hline
                Expan/Expan &   Uncertain   & Depreciation \\
                Expan/Restr &   Depreciation & Uncertain    \\
                Restr/Expan &   Appre       & Uncertain     \\
                Restr/Restr &   Uncertain   & Appreciation  \\
                \hline
                \end{tabular}   
            \item Fixed Ex rate regimes
                \be
                    \item If monetary expansionary (depreciation), governments need to buy money in the 
                        FX market, therefore will reverse the effect from monetary expansionary.
                    \item Fiscal expansionary \ra Appreciation(More money needed) \ra Government need to 
                        sell money to keep Ex rate stable. \ra Fiscal effect on aggregate demand will be reinforced.
                \ee
        \ee
    \item Monetary Approach to Exchange Rate determination\\
        Inflation play no role in exchange rate in Mundell-Fleming model.
        \\Assumptions: 1. Output is fixed.
        \be
            \item Method 1: Pure Monetary model. Assume: PPP holds, output is constant.
            \item Dornbusch overshooting model. Price are inflexible in short term.
                Expan Monetary \ra price increase, interest rate down \ra depreciatoin of currency.
                Therefore, in short term, price sticky, interest rate down too much. \ra depreciation is greater
                than PPP implies.
        \ee
    \item Portfolio Balance Approach to Exchange rate determination.
        \be
            \item It focues on long-term implications of fiscal policy on currency values.
            \item Fisical deficit\ra sell bonds\ra When investors thinks the country is safe, they will continue to buy bonds. 
                If the investors refuse to fund the deficits \ra depreciation
        \ee
    \item In short term, with free capital flows, expan fiscal \ra appreciation\\
        Long term\ra government has to reverse expan fiscal. Otherwise, investor
        will refuse to fund it, then the country have to monetize its debt (print money).\ra depreation

\ee
\subsection{Objectives of central bank intervention and capital controls and describe
the effectiveness of intervention and capital controls.}
See Notes P274. Old version.
\subsection{Describe warning signs of currency crisis.}
\be
    \item Terms of trade deteriorate
    \item Foreign reserve down quickly
    \item Real exchange rate is extremly higher than mean-reverting value.
    \item INflation increases.
    \item Equity markets have a boom-bust cycle.
    \item Money supply relative to bank reserves increases.
    \item Nominal private credit grows.
\ee
\subsection{Technical analysis}
See Notes P275 Old version.


\section{Economic Growth and the Investment Decision}
\subsection{Compare factors favoring and limiting economic growth in developed and developing economies}
Two important factors. 1: GDP per capita. 2: Growth of GDP
\be
    \item Preconditions for Growth
        \be
            \item Saving and investment. Positively correlated with economic development.
            \item Financial markets and intermediaries. Help resources reallocation. 
                However, it may increase leverage, risks.
            \item Political stability, rule of law and property rights.
            \item Investment in human capital. Worker's skills.
            \item Tax and regulatory systems. Lower tax burdens are good.
                Lower regulation levels are good.
            \item Free trade and unrestriced capital flows.
        \ee
\ee
\subsection{Describe the relation between the long-run rate of stock market appreciation and
the sustainable groth rate of economy.}
The growth in the price is related to earnings and GDP: 
$\Delta_P=\Delta_GDP + \Delta(E/GDP)+\Delta(P/E)$.
Over the long-term, $\Delta(E/GDP)=0,\Delta(P/E)=0$. Only GDP growth matters.
\subsection{Explain why potential GDP and its growth rate matter for equity and fixed income investors.}
Higer GDP growth\ra Higher interest rates \ra Higer real asset returns.\\
Higher GDP growth makes people think that future income is increasing, therefore increase consumptions
and reduce savings. To encourage consumers save, higher interest rate is needed.

In short term, actual GDP in excess of potential GDP will result in rising prices \ra inflationary pressure.
\subsection{Distinguish between capital deepening investment and technological progress and explain
how each affects economic growth and labor productivity}
\be
    \item Factor input and Economic growth
        \be
            \item Model: 2-factor aggregate production: Y~F(L,K) at a level of tech T.
                Output Y is a function of labor(L) and capital.
            \item Cobb-Douglas Production: $Y=TK^\alpha L^{(1-\alpha)}$
                \\Dividing both sides by $L$, the output per worker is 
                $$
                    Y/L=T(K/L)^\alpha
                $$
        \ee
\ee
\subsection{Forecase potential GDP based on growth accounting relations}
\be
    \item Growth Accounting Relations
        $$
        \Delta Y/Y=\Delta A/A+\alpha\times(\Delta K/K)+(1-\alpha)(\Delta L/L)
        $$
        i.e. growth rate in potential GDP = long-term growth of tech 
        + $\alpha$ long-term growth rate of capital
         + (1-$\alpha$)*(long-term growth rate of labor)
        \\ The growth of technology is not observable. Can be estimated from previous
        data: ex-growth rate - ex-growth rate from L and K
\ee
\subsection{Explain how natural resources affect economic growth and evaluate the arument
that limited availability of natural resources constrains economic growth}
\be
    \item Access to natural resouces does not require ownership of resources.
    \item Another theory: ownership of natural resources may actually inhibit growth. 
        \ra Dutch disease: global demand for natural resouces  drives up the coutries
        currency, making all other exports more expensive and uncompetitive.
\ee
\subsection{Explain how demographics, immigration, and labor force participation
affect the rate and sustainability of economic growth}
\be
    \item Labor Supply Factors
        \be
            \item Demographics: A countries age distribution. Countries with younger
                age will have a higher potential growth.
            \item Labor force participation.
            \item Immigration: a potential source in developed countries \Ra increase work force
            \item Average hours worked
        \ee
\ee
\subsection{Explain how investment in physical capital, human capital, and technological
development affects economic growth}
\be
    \item Human capital: knowledge and skills that individuals possess. Can be enhanced
        via education.
    \item Physical capital: infrastructure, computers, telecommunications(ICT)
        AND non-ICT capital(machineary, transportation and non-residential construction).
        More investment in physical capital \Ra Good GDP growth.
        \\ MOre investment may enhance the tech improvements.
    \item Technological development. Investment in tech will increase the productivity.
    \item Public infrastructure: like roads, bridges, and municipal facilities. 
        This will enhance total productivity. Because the private investment will not
        invest these public things for their little returns.
\ee
\subsection{Compare classical growth theory, neoclassical growth theory, and endogenous
growth theory}
\be
    \item Classical growth theory: In the long-term, population growth increases whenever
        there are increases in per capita income above subsistence level due to increase
        in capital or tech progress. \Ra Growth in real GDP per capita
        is not permanant. \Ra This is not supported by observed facts.
    \item Neoclassical Growth theory:
        \be
            \item Estimate steady state growth rate. Equilibrium economy is when the output-to-capital
                ratio is constant. When the output-capital ratio is constant, the labor-to-capital
                ratio and output per capita also grow at the equilibrium rate. Check textbooks here.
            \item 
        \ee
\ee

\end{document}
